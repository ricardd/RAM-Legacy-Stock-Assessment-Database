\section*{Results}
\subsection*{Summary}
\noindent
Total number of proper stocks assessments: REF:SQL:TOTNUMASSESSMENT, from REF:SQL:TOTNUMASSESSFISH marine fish populations and REF:SQL:TOTNUMASSESSINVERT
invertebrate populations.


\subsection*{Global fisheries}

\subsection*{Management bodies and geography}
\noindent
Number of assessments from NMFS: REF:SQL:TOTNUMASSESSNMFS.\\
Number of assessments from ICES: REF:SQL:TOTNUMASSESSICES.\\
Number of assessments from MFish: REF:SQL:TOTNUMASSESSMFish.\\
Number of assessments from DFO: REF:SQL:TOTNUMASSESSDFO.\\
Number of assessments from AFMA: REF:SQL:TOTNUMASSESSAFMA.\\
Number of assessments from DETMCM: REF:SQL:TOTNUMASSESSDETMCM.\\


Assessments were available for REF:SQL:TOTNUMLMESPRIMARY LMEs, with the greatest number of
assessed stocks coming from REF:SQL:NUMASSESSLME:1,
REF:SQL:NUMASSESSLME:2, REF:SQL:NUMASSESSLME:3,
REF:SQL:NUMASSESSLME:4, REF:SQL:NUMASSESSLME:5, REF:SQL:NUMASSESSLME:6
and REF:SQL:NUMASSESSLME:7.

\subsection*{Taxonomy}
\noindent

Number of species in FishBase: REF:SQL:NUMSPPFISHBASE\\
Number of species in SAUP: REF:SQL:SAUPNUMSPP\\
Number of species in RAM Legacy: REF:SQL:TOTNUMASSESSBYSPECIES (from REF:SQL:TOTNUMASSESSBYFAMILY families and REF:SQL:TOTNUMASSESSBYORDERS orders) \\
Top 5 taxonomic orders: REF:SQL:NUMASSESSORDERS:1, REF:SQL:NUMASSESSORDERS:2, REF:SQL:NUMASSESSORDERS:3, REF:SQL:NUMASSESSORDERS:4, REF:SQL:NUMASSESSORDERS:5 \\

\subsection*{Timespan}
\noindent
Number of assessments with catch timeseries: REF:SQL:NUMCATCHSERIES.\\
Number of assessments with recruitment timeseries: REF:SQL:NUMRSERIES.\\
Number of assessments with spawning stock biomass timeseries: REF:SQL:NUMSSBSERIES.\\

The median lengths of catch/landings, SSB, and recruitment timeseries
were REF:SQL:MEDCATCHLEN, REF:SQL:MEDSSBLEN, and REF:SQL:MEDRLEN
years, respectively (Figure~\ref{fig:orca}).  The time period covered by 90\% of assessments
is: catch/landings (REF:SQL:CATCHTIMESPAN90), SSB
(REF:SQL:SSBTIMESPAN90), recruitment (REF:SQL:RTIMESPAN90), while that
covered by 50\% of assessments is: catch/landings
(REF:SQL:CATCHTIMESPAN50), SSB (REF:SQL:SSBTIMESPAN50), recruitment
(REF:SQL:RTIMESPAN50)
 
\subsection*{Assessment methodologies and reference points}
\noindent
The three most common assessment methods were
REF:SQL:NUMASSESSMETHOD:1, REF:SQL:NUMASSESSMETHOD:2 and
REF:SQL:NUMASSESSMETHOD:3. Regionally, Virtual Population Analysis
(VPA) is still the most common assessment model for European stocks
(REF:SQL:PERCENTVPAICES\% of REF:SQL:TOTNUMASSESSICES assessments),
Canada (REF:SQL:PERCENTVPADFO\% of REF:SQL:TOTNUMASSESSDFO
assessments) and Argentina (REF:SQL:PERCENTVPACFP\% of
REF:SQL:TOTNUMASSESSCFP assessments), whereas statistical catch-at-age
and -length models are more common for the United States
(REF:SQL:PERCENTSCALNMFS\% of REF:SQL:TOTNUMASSESSNMFS assessments),
Australia (REF:SQL:PERCENTSCALAFMA\% of REF:SQL:TOTNUMASSESSAFMA
assessments) and New Zealand (REF:SQL:PERCENTSCALMFISH\% of
REF:SQL:TOTNUMASSESSMFish assessments).

Biomass- or exploitation-based reference points were available for
REF:SQL:NUMASSESSBIOREF (REF:SQL:PERCENTASSESSBIOREF\%) and
REF:SQL:NUMASSESSEXPLOITREF (REF:SQL:PERCENTASSESSEXPLOITREF\%)
assessments, respectively.

\subsection*{Stock status}
\noindent
Of the
REF:SQL:NUMASSESSFRIEDEGG stocks presented in
Figure~\ref{fig:friedegg}, REF:SQL:NUMASSESSBIOASSESSREF and
REF:SQL:NUMASSESSBIOSCHAEFERREF of the biomass reference points and
REF:SQL:NUMASSESSEXPLOITASSESSREF and
REF:SQL:NUMASSESSEXPLOITSCHAEFERREF of the exploitation reference
points come from assessments and from surplus production model fits,
respectively.

Overall, REF:SQL:PERCENTASSESSMENTSBELOWBMSY\% of stocks are estimated
to be below their biomass-related MSY BRP, that is $B_{curr}<B_{msy}$,
and REF:SQL:PERCENTASSESSMENTSABOVEFMSY\% are estimated to be above
their exploitation-related MSY BRP, $U_{curr}>U_{msy}$
(n=REF:SQL:NUMASSESSFRIEDEGG stocks total; Figure~\ref{fig:friedegg}).
Of the stocks for which biomass is currently estimated to be below
$B_{msy}$, REF:SQL:PERCENTASSESSBELOWBMSYANDBELOWFMSY\% have had their
exploitation rate reduced below $U_{msy}$, suggesting potential for
recovery (Figure~\ref{fig:friedegg}). The remaining
REF:SQL:PERCENTASSESSBELOWBMSYANDABOVEFMSY\% of these stocks however,
still have excessive exploitation rates (Figure~\ref{fig:friedegg}).
On a positive note, REF:SQL:PERCENTASSESSMENTSABOVEBMSY\% of all stocks are
estimated to be above $B_{msy}$, and
REF:SQL:PERCENTASSESSABOVEBMSYANDBELOWFMSY\% of the stocks above
$B_{msy}$ also have $U_{current}$ below $U_{msy}$.
