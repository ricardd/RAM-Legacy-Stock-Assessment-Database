\newpage
\section*{Results}
%\subsection*{Scope of the RAM Legacy database}
\subsection*{The knowledge-base for commercially-exploited marine stocks}
In total, 324 recent stock assessments for
288 marine fish and 36
invertebrate populations are included in the RAM Legacy database
(Version 1.0, 2010; Table S1). Together these comprise time series of
catch/landings were available for 308 stocks (95\%),
SSB estimates for 271 stocks (84\%), and recruitment estimates for
269 stocks (83\%) (Table S1).

\subsubsection*{Management bodies and geography}
Stock assessments are derived from fisheries management bodies in
Europe, the United States, Canada, New Zealand, Australia, Russia,
South Africa and Argentina (Table~\ref{tab:mgmt}). Also included are
assessments conducted by eight Regional Fisheries Management
Organizations (RFMOs), in the Northwest Atlantic, Atlantic, Pacific
and Indian Ocean (Table~\ref{tab:mgmt}). Assessments from the United
States comprise by far the most stocks of any country or region
(n=139); assessments from the European Union's
management body, the International Council for the Exploration of the
Seas (ICES), comprise the the second greatest number of stocks
(n=63).  Whereas nations are responsible for
managing all populations within their EEZs, RFMOs typically focus on a
certain type of species (e.g.  halibut, tunas) or fisheries (e.g.
pelagic high seas) within a given area and hence assess a smaller
number of stocks.

Most assessments come from North America, Europe, Australia, New
Zealand and the High Seas, while there are few from regions such as
Southeast Asia, South America, and the Indian Ocean (outside
Australian waters) (Figure~\ref{fig:lmes}). Assessments were available for 31 LMEs, with the greatest number of
assessed stocks coming from Northeast U.S. Continental Shelf (n=58),
California Current (n=35), New Zealand Shelf (n=29),
Gulf of Alaska (n=26), Celtic-Biscay Shelf (n=26), East Bering Sea (n=22)
and Southeast U.S. Continental Shelf (n=20) (Figure~\ref{fig:lmes}).

%Northeast U.S. Continental Shelf (n=88), the California Current (n=35, the East Bering Sea (n=32), the New Zealand Shelf (n=29), the Gulf of Alaska (n=28), the Celtic-Biscay Shelf (n=24) and the Newfoundland-Labrador Shelf (n=21)

\subsubsection*{Taxonomy}

Assessments for 157 species from
57 families and 20
orders are included in the database (Figure~\ref{fig:taxo:srdb}). Five
taxonomic orders (Gadiformes (n=67),
Perciformes (n=62), Pleuronectiformes (n=53),
Scorpaeniformes (n=41) and Clupeiformes (n=36)) account for
80\% of available stock assessments.  Of these, Perciformes, the most
speciose Order of marine fishes are in fact underrepresented in the
database (46.04\% of all marine fish species vs.  19\% of all marine
fish assessments), while the other four orders are
taxonomically overrepresented: Clupeiformes (2.1\% of marine fishes
vs.  11\% in the database), Gadiformes (3.3\% of marine fishes vs.
21\% in the database), Pleuronectiformes (4.5\% of marine fishes vs.
17\% in the database), Scorpaeniformes (8.5\% of marine fishes vs.
12\% in the database) (Figure~\ref{fig:taxo:threepanel}).

Assessed marine fish stocks in the RAM Legacy database comprise a
relatively small proportion of harvested taxa
(24\% of fish species from the SAUP database)
and an even smaller proportion of marine fish biodiversity
(1\% of fish species in FishBase;
Figure~\ref{fig:taxo:threepanel}). In turn, catches from the SAUP
database, which come from 649 species and
36 orders (Figure~\ref{fig:taxo:threepanel}),
represent only 5\% of the
12339 species and 67\%
of the 54 different orders present in FishBase
(Figure~\ref{fig:taxo:threepanel}). The diversity of harvested marine
invertebrates is clearly underrepresented in the stock assessment
database and likely in stock assessments in general.

%The paucity of marine invertebrate stock assessments means these
%species are more poorly taxonomically represented in the database than
%fishes (Figure 4XX). Only XX\% .....

%\subsubsection*{Global Fisheries}

%Table~\ref{tab:worldfisheries}

\subsubsection*{Ecology}
Assessed species span a range of ecological traits.
288 assessments reported some life-history
information (e.g. growth, maturity, fecundity) for the assessed
species Of these, age at sexual maturity ranged from XX to XX (n=XX,
mean=XX) and . The trophic level of assessed species ranged from XX to
XX with a mean of XX (Figure~\ref{fig:TL}).

%was 288.

%Assessed species in the data span a range of ecological
%traits..... [need trophic level plot here; Figure 6). -assessemnts by
%trophic level: can we make a barplot showing number of stocks by
%trophic level for a) overexploited, and b) not overexploited species
%(i.e. could be on same plot but different hatching for the a vs. b.


\subsubsection*{Timespan }

%Of the 324 stock assessments, time series data of
%catch/landings were available for 308 stocks (95\%),
%of SSB for 271 stocks (84\%), and of recruitment for
%269 stocks (83\%).  

The median lengths of catch/landings, SSB, and recruitment timeseries
were 38, 34, and 33
years, respectively (Figure~\ref{fig:orca}).  The time period covered by 90\% of assessments
is: catch/landings (1967-2007), SSB
(1972-2007), recruitment (1971-2006), while that
covered by 50\% of assessments is: catch/landings
(1983-2004), SSB (1985-2005), recruitment
(1984-2003) (Figure~\ref{fig:orca}).

\subsubsection*{Stock assessment methodologies and BRPs}
%In addition to the 324 assessments in the
%database, indices of relative abundance from scientific surveys are
%available for an additional 26 stocks. 

The three most common assessment methods were
Statistical catch-at-age/length models (n=164), Virtual Population Analyses (n=90) and
Biomass dynamics model (n=45). Regionally, Virtual Population Analysis
(VPA) is still the most common assessment model for ICES and
Argentina's CFP, whereas statistical catch-at-age and -length models
are more common for NMFS, AFMA and MFish. 
% (need to add a sentence here about the regional differences). 

257 (81\%) and
222 (69\%)
assessments reported biomass- or exploitation- based reference points
of some sort, respectively. 


\subsubsection*{Global Fisheries}
Assessments were available for 8 of the 10 largest fisheries for
individual fish stocks globally (Table~\ref{tab:worldfisheries}). Assessments for Peruvian
anchoveta, the world's largest fishery, and for Japanese anchovy in
the East China Sea (the eighth largest species for an individual
stock, and tenth overall) were not accessible. Looking more broadly,
the database contains assessments for 16 of the 30 largest fisheries
for individual fish stocks globally, and 17 of the 40 largest
fisheries globally (including those recorded at lower taxonomic
resolutions) (Table~\ref{tab:worldfisheries}). Many of the fisheries not included in the RAM
Legacy database, especially those recorded in the SAUP database as
``Marine fishes not identified'' (n=7), occur in developing countries
and have no known formal stock assessment conducted for them.  From a
national perspective, assessments are only included for 2 of the top
10 wild-caught marine fisheries producing nations, U.S.A. and Russia
\citep{FAO:sofia}, with only two assessments from the latter. We were unable
to obtain any assessments from the other top 10 countries: China,
Peru, Indonesia, Japan, Chile, India, Thailand, Philippines \citep{FAO:sofia}.

\subsubsection*{The status of commercially exploited marine stocks }
Overall, 58\% of stocks are below
their biomass-related MSY BRP, that is $B_{curr}<B_{msy}$, and
30\% are above their
exploitation-related MSY BRP, $U_{curr}>U_{msy}$
(n=241 stocks total; Figure~\ref{fig:friedegg}).
Of the stocks for which biomass is currently below Bmsy,
53\% have had their
exploitation rate reduced below Umsy, suggesting potential for
recovery (Figure~\ref{fig:friedegg}). The remaining
47\% of these stocks however,
still have excessive exploitation rates (Figure~\ref{fig:friedegg}).
Encouragingly, 42\% of all stocks are
above Bmsy, and 94\% of the
stocks above Bmsy also have Ucurrent below Umsy. 

%There was no significant
%difference in the status of stocks with assessment-derived BRPs (n=62,
%solid dots in Figures ~\ref{fig:friedegg} and ~\ref{fig:friedeggmgmt}) vs. Schaefer-estimated BRPS (n=178,
%open circles in Figures ~\ref{fig:friedegg} and ~\ref{fig:friedeggmgmt; p<XX).).

The status of exploited marine stocks varied widely depending on the
management body (Figure 7). Most European stocks (managed by ICES)
have biomasses less than Bmsy (79\%), and over half of these stocks (61\%)
still have exploitation rates exceeding Umsy. Canadian stocks
(managed by DFO) also had low biomass (79\% $< B_{msy}$), but all but one
of these has had its exploitation rate reduced below Umsy. In
contrast, about half (49\%) of U.S. stocks (managed by NMFS) are
estimated to still be above Bmsy, and of the XX stocks that are below
Bmsy XX\% have exploitation rates below Umsy (Figure 7). New Zealand
and Australian stocks (managed by MFish and AFMA) MORE HERE. Argentinian stocks managed by CFP MORE HERE. South African stocks managed by DETMCM MORE HERE. 

Stocks under international management MORE HERE.


%From these assessments,
%66 report both a biomass-based and an
%exploitation-based BRP and appear as solid dots on
%Figures~\ref{fig:friedegg} and ~\ref{fig:friedeggmgmt}.
%Schaefer-derived BRPs add an additional
%175 assessments, for a total of
%241 assessments used to generate
%Figure~\ref{fig:friedegg}. Overall,
%58\% of assessed stocks are below
%their biomass-related MSY BRP and
%30\% are above their
%exploitation-related MSY BRP. Different management bodies have
%different overall status of current biomass to BRPs
%(Figure~\ref{fig:friedeggmgmt}).


%Status of Assessed Stocks 
%Need to know:
%\% of stocks with biomass below Bmsy
%\% of stocks with
%overall and by management body.
 

