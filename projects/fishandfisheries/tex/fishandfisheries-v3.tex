%% Last modified: Time-stamp: <2010-07-02 11:13:49 (srdbadmin)>
%%\documentclass[letterpaper,12pt]{article}
\documentclass[letterpaper,review,authoryear,12pt]{myelsarticle}
%%\documentclass[preprint,authoryear,12pt]{elsarticle}

\usepackage{amssymb}
\usepackage{pdflscape}
\usepackage{longtable}
\usepackage{graphicx}
\usepackage{paralist}
\usepackage{comment}
\usepackage{color}
\usepackage[left=3cm,top=3cm,right=3cm, bottom=3cm,nohead]{geometry}
\usepackage{booktabs}
\usepackage{url}

%\title{Assessing the knowledge-base for commercially exploited marine fishes and invertebrates with a new global database of stock assessments}

%%%\title{Assessing the knowledge-base for commercially exploited marine fishes and invertebrates with a new global database of stock assessments \\ \vspace{0.5cm} Alternative Title 1: A new global stock assessment database for exploited marine species \\ \vspace{0.5cm} Alternative Title 2: Understanding marine population dynamics using a new global database \\ \vspace{0.5cm} Suggested Running Title: A new global stock assessment database}

%\author{
%Daniel Ricard \thanks{Department of Biology, Dalhousie University, Halifax, NS B3H 4J1, Canada} \and 
%C{\'o}il{\'i}n Minto \and 
%Julia Baum\thanks{Scripps Institution of Oceanography, UCSD, 9500 Gilman Drive, La Jolla, CA 92093-0202, USA} \and 
%Olaf Jensen \thanks{School of Aquatic and Fishery Sciences, University of Washington, Seattle, WA  98195-5020, USA}
%}

\begin{document}

\begin{frontmatter}

%% Title, authors and addresses

%% use the tnoteref command within \title for footnotes;
%% use the tnotetext command for the associated footnote;
%% use the fnref command within \author or \address for footnotes;
%% use the fntext command for the associated footnote;
%% use the corref command within \author for corresponding author footnotes;
%% use the cortext command for the associated footnote;
%% use the ead command for the email address,
%% and the form \ead[url] for the home page:
%%
%% \title{Title\tnoteref{label1}}
%% \tnotetext[label1]{}
%% \author{Name\corref{cor1}\fnref{label2}}
%% \ead{email address}
%% \ead[url]{home page}
%% \fntext[label2]{}
%% \cortext[cor1]{}
%% \address{Address\fnref{label3}}
%% \fntext[label3]{}

\title{Assessing the knowledge-base for commercially exploited marine fishes and invertebrates with a new global database of stock assessments
\\ \vspace{0.5cm} \small Alternative Title 1: A new global stock assessment database for exploited marine species
\\ \vspace{0.5cm} Alternative Title 2: Understanding marine population dynamics using a new global database 
\\ \vspace{0.5cm} \normalsize Suggested Running Title: A new global stock assessment database \vspace{1.5cm}}

%% use optional labels to link authors explicitly to addresses:
%% \author[label1,label2]{<author name>}
%% \address[label1]{<address>}
%% \address[label2]{<address>}

\normalsize
\author[dal]{Daniel Ricard\corref{cor1}}
\ead{ricardd@mathstat.dal.ca}
\author[dal]{C{\'o}il{\'i}n Minto\fnref{fn2}}
\author[scripps]{Julia Baum\fnref{fn3}}
\author[uw]{Olaf Jensen\fnref{fn4}}
\cortext[cor1]{Corresponding author: Tel: 902-494-2146, Fax: 902-494-3736} 
\fntext[fn2]{Current Address: Marine and Freshwater Research Centre, Galway-Mayo Institute of Technology, Dublin Road, Galway, Ireland}
\fntext[fn3]{Current Address: National Center for Ecological Analysis and Synthesis, UCSB, 735 State St. Suite 300, Santa Barbara, CA 93101, USA}
\fntext[fn4]{Current Address: Institute of Marine and Coastal Sciences, Rutgers University, 71 Dudley Road, New Brunswick, NJ 08901-8525, USA}

\address[dal]{Department of Biology, Dalhousie University, Halifax, NS B3H 4J1, Canada}
\address[scripps]{Scripps Institution of Oceanography, UCSD, 9500 Gilman Drive, La Jolla, CA 92093-0202, USA}
\address[uw]{School of Aquatic and Fishery Sciences, University of Washington, Seattle, WA  98195-5020, USA}


%\begin{abstract}
%% Text of abstract
%%%%%%%%%%%%%%%%%%%%%%%%%
%% Abstract
%%%%%%%%%%%%%%%%%%%%%%%%% 
%%\newpage
\section*{Abstract}

%Data used to assess the status of individual fish stocks range from
%very little information on many of the world's artisanal fisheries, to
%commercial landings, research surveys, and sophisticated population
%dynamics models that integrate many sources of information.  Previous
%evaluations of the state of global fisheries have used catch data,
%which may be poor proxies for fish stock abundances. A global
%compilation of stock assessment data in the mid-1990s enabled
%substantial syntheses of stock status; however its focus was on
%stock-recruitment relationships and it is now 15 years out of date. 

To facilitate global analyses of population dynamics and the status of
fished species, we have assembled a new database, the RAM Legacy
Database, of the most intensively studied commercially exploited
marine fish stocks. Results from assessment models, including time
series of total biomass, spawner biomass, recruits, fishing mortality,
and catch; reference points; and ancillary information on the life
history, management, and assessment methods for each stock.  Here, we
present the first overview of this database and use it to evaluate the
knowledge-base for assessed marine species.  Assessments were
assembled for 324 stocks
(288 fish species representing
45 families, and 36
invertebrate species representing 12
families), including 8 of the world's 10 largest fisheries.
Assessments were obtained from 18 national and international
management institutions, with most coming from North America, Europe,
Australia, New Zealand and the high seas. Stocks present in the
database come from 31 Large Marine Ecosystems
and cover the Atlantic, Pacific, Indian, Mediterranean, Arctic and Antarctic
Ocean. Reference points were available or could be calculated for
about 74\% of these stocks. The available
data provide new insight into the status of exploited populations,
57\% of stocks with reference points
were estimated to be below $B_{msy}$, and
29\% had exploitation levels
estimated to be above $U_{msy}$.  Assessed marine fish stocks comprise
a relatively small proportion of harvested taxa (24\%), and an even
smaller proportion of marine fish biodiversity (1\%). We hope that
access to the database will facilitate new research into life
histories, population dynamics and the effects of fishing and
encourage further data contributions from stock assessment scientists.

%extractions from the
%database provide new insight into the status of exploited
%populations


%Globally, stock assessments were found
%for 324 stocks (288 species
%of fishes representing 45 families and
%36 species of invertebrates representing
%12 families), from 19
%national and international
%management institutions.

\noindent \textbf{Keywords}: marine fisheries, meta-analysis, population dynamics models, relational database, stock assessment, synthesis.

%with XX\% coming from north temperate regions (North
%Atlantic, North Pacific)
%\noindent Keywords: marine fisheries, meta-analysis, population dynamics models, relational database, stock assessment, synthesis.
%\newpage

%  Geographic differences in assessment
%methods show that Statistical Catch at Age (SCA) models are widely
%used by the west coast of the U.S. (XX percent of assessments),
%regional fishery management organizations in the Pacific (XX percent
%of assessments), and New Zealand (XX percent of assessments); the east
%coast of the U.S. is transitioning from Virtual Population Analysis
%(VPA) to SCA (XX percent of assessments conducted since 2000 have used
%SCA); while VPA is still the dominant assessment
%technique in western Europe (XX percent of assessments).


%\end{abstract}

%\begin{keyword}
%marine fisheries \sep meta-analysis \sep population dynamics models \sep relational database \sep stock assessment \sep synthesis
%\end{keyword}

\end{frontmatter}


\newpage
%\newpage
\section*{Abstract}

%Data used to assess the status of individual fish stocks range from
%very little information on many of the world's artisanal fisheries, to
%commercial landings, research surveys, and sophisticated population
%dynamics models that integrate many sources of information.  Previous
%evaluations of the state of global fisheries have used catch data,
%which may be poor proxies for fish stock abundances. A global
%compilation of stock assessment data in the mid-1990s enabled
%substantial syntheses of stock status; however its focus was on
%stock-recruitment relationships and it is now 15 years out of date. 

To facilitate global analyses of population dynamics and the status of
fished species, we have assembled a new database, the RAM Legacy
Database, of the most intensively studied commercially exploited
marine fish stocks. Results from assessment models, including time
series of total biomass, spawner biomass, recruits, fishing mortality,
and catch; reference points; and ancillary information on the life
history, management, and assessment methods for each stock.  Here, we
present the first overview of this database and use it to evaluate the
knowledge-base for assessed marine species.  Assessments were
assembled for 324 stocks
(288 fish species representing
45 families, and 36
invertebrate species representing 12
families), including 8 of the world's 10 largest fisheries.
Assessments were obtained from 18 national and international
management institutions, with most coming from North America, Europe,
Australia, New Zealand and the high seas. Stocks present in the
database come from 31 Large Marine Ecosystems
and cover the Atlantic, Pacific, Indian, Mediterranean, Arctic and Antarctic
Ocean. Reference points were available or could be calculated for
about 74\% of these stocks. The available
data provide new insight into the status of exploited populations,
57\% of stocks with reference points
were estimated to be below $B_{msy}$, and
29\% had exploitation levels
estimated to be above $U_{msy}$.  Assessed marine fish stocks comprise
a relatively small proportion of harvested taxa (24\%), and an even
smaller proportion of marine fish biodiversity (1\%). We hope that
access to the database will facilitate new research into life
histories, population dynamics and the effects of fishing and
encourage further data contributions from stock assessment scientists.

%extractions from the
%database provide new insight into the status of exploited
%populations


%Globally, stock assessments were found
%for 324 stocks (288 species
%of fishes representing 45 families and
%36 species of invertebrates representing
%12 families), from 19
%national and international
%management institutions.

\noindent \textbf{Keywords}: marine fisheries, meta-analysis, population dynamics models, relational database, stock assessment, synthesis.

%with XX\% coming from north temperate regions (North
%Atlantic, North Pacific)
%\noindent Keywords: marine fisheries, meta-analysis, population dynamics models, relational database, stock assessment, synthesis.
%\newpage

%  Geographic differences in assessment
%methods show that Statistical Catch at Age (SCA) models are widely
%used by the west coast of the U.S. (XX percent of assessments),
%regional fishery management organizations in the Pacific (XX percent
%of assessments), and New Zealand (XX percent of assessments); the east
%coast of the U.S. is transitioning from Virtual Population Analysis
%(VPA) to SCA (XX percent of assessments conducted since 2000 have used
%SCA); while VPA is still the dominant assessment
%technique in western Europe (XX percent of assessments).

\newpage

\section*{Introduction}

Marine wild capture fisheries provide more than 80 million tons of
fisheries products (both food and industrial) per year and employ 43.5
million people (wild capture and aquaculture, \citep{FAO:sofia}).
At the same time, fishing has been recognized as having one of the
most widespread human impacts in the world's oceans
\citep{Halpern:etal:2008:science}, and the Food and Agricultural
Organization of the United Nations (FAO) estimates that two-thirds of
fish stocks globally are fully exploited or overexploited
\citep{FAO:sofia}.  While many fisheries have reduced exploitation
rates to levels that should in theory promote recovery, overfishing continues to
be a serious global problem \citep{Worm:etal:2009:science}.  Fishery
managers are asked to address multiple competing objectives, including
maximizing yields, ensuring profitability, reducing bycatch, and
minimizing the risk of overfishing.  Given the enormous social and
economic costs \citep{Rice:etal:2003:icescm} and ecosystems
consequences \citep{Frank:etal:2005:science, Myers:etal:2007:science}
of collapsed fisheries, it is imperative that we are able to quickly
learn from successful and failed fisheries from around the world.

Effective management of exploited fish populations generally requires
an understanding of where the current size and harvest rate
lie in relation to the size and rate which maximize
fishery benefits or limit the risk of overfishing.  This process of
quantitative determination of stock status and estimation of reference
points is called stock assessment.  Some fisheries in developing
countries have apparently provided sustainable yields for long periods
of time without formal stock assessment (e.g. many community-managed
fisheries in Oceania, \citep{Johannes:2002:arees}).  This has been achieved by
limiting harvest rates, often through gear restrictions or seasonal or
area closures.  In modern industrialized fisheries, however, where fishing
capacity exceeds the productivity of fished stocks stock
assessment is an integral component of responsible
management \citep{Hilborn:Walters:1992}.

The global databases of fishery landings compiled by FAO
\citep{FAO:fishstat} and synthesized by the Sea Around Us project
\citep{Watson:etal:2004:fandf} have proven to be valuable resources
for understanding the status of fisheries worldwide; however, catch
data alone can be misleading when used as a proxy for stock size.
Many papers have used these data to examine changes in fishery status
\citep{Worm:etal:2006:science, Costello:etal:2008:science}, including
changes in trophic level \citep{Pauly:etal:1998,
  Essington:etal:2006:procnatacadsci, Newton:etal:2007:currentbiol}.
Most of these analyses rely (either explicitly or implicitly) on the
assumption that catch or landings is a reliable index of stock size.
Critics have pointed out that catch can change for a number of reasons
unrelated to stock size, including changes in targeting, fishing
restrictions, or market preferences \citep{deMutsert:etal:2008:pnas,
  Murawski:Methot:Tromble:2007:science, Hilborn:2007:science}.
Standardizing catch by the amount of fishing effort
(catch-per-unit-of-effort, CPUE) is an improvement, particularly when
these data are modeled to account for spatial, temporal, and
operational factors affecting the CPUE, but CPUE can still be an
unreliable index of relative abundance since it is difficult to
account for all relevant factors \citep{Hutchings:Myers:1994:cjfas,
  Harley:etal:2001:cjfas, Walters:2003:cjfas, Polacheck:2006:marpol}.

Stock assessments consider time series of catch along with other
sources of information such as: natural mortality rates, changes in
size or age composition, stock-recruitment relationships, and CPUE
coming from different fisheries and/or from fishery-independent surveys.
Because they integrate across multiple sources of information, stock
assessment models are thought to provide a more accurate picture of
changes in abundance than catch data alone
\citep{Sibert:etal:2006:science}. Yet, without a current and
comprehensive database of stock assessments, scientists wishing to
conduct comparative analyses of marine fish population dynamics and
fishery status have little choice but to use problematic catch data.

The first global database of stock assessment information, the Myers
Stock Recruitment Database, was developed by the late Ransom A. Myers
and colleagues in the mid-1990s \citep{Myers:etal:1995:summary}.
While the database was primarily known for its time series of stock
and recruitment, it did contain time series of fishing mortality rates
for many stocks; biological reference points were however largely
absent. The original release version of the Myers database
\citep{Myers:etal:1995:summary} included spawning stock size and
recruitment time series for 274 stocks representing 92 species as well
as fishing mortality rates time series for 144 stocks. The number of
entered stocks grew to approximately 642 stocks (509 with at least one
SR pair) over the period from 1995-2005. Note that the Salmonidae
comprised 290 assessments in the original database.  The assessment
results collated by Dr. Myers were used to:
\begin{inparaenum}[1\upshape)] \item decisively answer the question of
  whether recruitment shows any relationship to spawning stock size
  \citep{Myers:Barrowman:1996:fishbull}, \item investigate potential
  depensation in stock-recruitment relationships
  \citep{Myers:etal:1995:science, Liermann:Hilborn:1997:cjfas,
    Garvey:etal:2009:cjfas}, \item discover generalities in the annual
  reproductive rates of fishes \citep{Myers:etal:1999:cjfas,
    Myers:etal:2001:cjfas}, \item investigate density-dependence in
  juvenile mortality \citep{Myers:2001:ices, Minto:etal:2008:nature}, \item develop informative Bayesian priors on steepness
  \citep{Myers:etal:1999:cjfas, Myers:etal:2002:najfm,
    Dorn:2002:najfm}, and \item examine patterns of collapse and recovery in exploited fish populations \citep{Hilborn:1997:csiro, Hutchings:2000:nature, Hutchings:2001:jfishb, Hutchings:Baum:2005:philB} \end{inparaenum}.  

%The Myers database has also
%been used for several studies of collapse and recovery of exploited
%fish populations \citep{Hutchings:2000:nature, Hutchings:2001:jfishb,
%  Hilborn:1997:csiro}..

Although the original Myers database \citep{Myers:etal:1995:summary}
was critical for motivating comparative analyses in fisheries science, most of the stocks are now 15 years out of date.
For stocks that were depleted in 1995, the past 15 years
include valuable observations at low stock size or of a recovering
population, both of which are critical for estimating population
dynamics parameters such as the behaviour of the stock-recruitment
relationship near the origin. In addition, there have been numerous improvements in
stock assessment methodologies (including important advances in
statistical catch-at-age or catch-at-length models) and assessments
have been conducted for the first time for many species.

Meta-analyses of fishery status also have been hampered by the
lack of a global assessment database containing biological reference
points (BRPs, e.g., the total/spawning biomass and fishing mortality rate that
produce Maximum Sustainable Yield (MSY), $B_{MSY}$ and $F_{MSY}$).  Knowledge of BRPs
is important if stocks are to be managed for high yields that can be
sustained over time \citep{Mace:1994:cjfas}.  Without information on
reference points, previous analyses of stock assessments or catch data
have been forced to use non-biological thresholds to define fishery
status, such as the greatest 15-year decline
\citep{Hutchings:Reynolds:2004:biosci} or 10 percent of maximum catch
\citep{Worm:etal:2006:science}. Ad hoc reference points based on some
fraction of the maximum of a time series also have undesirable
statistical properties and can result in false collapses when applied
to inherently variable time series of catch or abundance
\citep{Wilberg:Miller:2007:science, branch:2008:marpol}.  Complicating
comparisons of fishery status is the fact that different BRPs are used
in different parts of the world and even the same BRP can be used in a
different manner, for example, as a target or as a limit. 

Here we present a new global database of stock assessments for
commercially exploited marine fish populations.  The database is an
update and extension of that developed by Ransom A. Myers, and is named
the RAM Legacy database in honour of his pioneering contribution.  This
effort is the first global stock assessment database to:
\begin{enumerate}
\item Use a formal relational database structure;
\item Use source control software to organise release versions;
\item Include metadata related to the geographic location of the stock, the type of assessment model used, and the original source document for the assessment data;
\item Include biological reference points, in addition to stock-specific life history information. 
\end{enumerate}


We use the new RAM Legacy database (Version 1.0, 2010) to evaluate the
knowledge-base for commercially exploited marine populations in terms
of institutional contributions, geography, taxonomy, ecology,
timespan, stock assessment methodologies and BRPs. We compare the
database's taxonomic coverage to that of global fisheries catches and to
global fish diversity. We then evaluate the status of assessed stocks
globally, and by management body, referencing all stocks to a
comparable benchmark. Finally, we discuss biases in the knowledge base
for assessed marine species, highlight potential applications of the
database, point out its caveats and limitations, and outline
directions for future development.

% have used the database to assess the knowledge-base for management of marine fish populations and address the following questions:
%\begin{enumerate}
%\item What fraction of world wild-capture fishery landings come from assessed stocks and how does this proportion vary by region?
%\item What is the temporal coverage of stock assessments, i.e. how far back do stock assessments look when reconstructing trends in abundance?
%\item What are the taxonomic and geographic biases, if any, in assessed stocks?
%\item Which stock assessment approaches and biological reference points are used and how does this vary by region?
%%\item How accessible is stock assessment information in different regions?
%\end{enumerate}

\newpage
\section*{Methods} 

\subsection*{The RAM Legacy database}
The RAM Legacy database is a global relational database designed and
developed to store data from all current and accessible population
dynamics model-based fisheries stock assessments for marine fish and
invertebrate populations. Time series of spawning stock biomass (SSB),
total biomass (TB), recruits (R), total catch (TC) or landings (TL),
and fishing mortality (F) from individual stock assessments form the
core of the database.  Apart from catch/landings, these time series
are not raw data, but rather the output of population dynamics models;
depending on the type of assessment model not all of these time series
were available for every stock. The database also contains details
about the time series data, including the age and sex of spawners, age
of recruits, and the ages used to compute the fishing mortality, as
well as BRPs and some life history information (e.g. growth
parameters, age and length at 50\% maturity, natural mortality).
Metadata for each stock assessment consists of taxonomic information
about the species and the geographic location of the stock (detailed in
``Links to related databases''), the management body that conducted
the assessment, the assessment methodology, the reference for the
stock assessment document, the name of the recorder entering the
assessment data, and the date the assessment was entered.  Some
assessments, particularly those for invertebrates, were based only on
CPUE time series rather than population dynamics models. While we
included these in the database (n=26), the database description and
analyses presented here focus on those stocks assessed using
population dynamics models.

Over the past two and a half years, we have employed a variety of
search methods in an attempt to obtain all recent fisheries stock
assessments.  Publicly available stock assessment reports available
from the internet were the primary data source.  These reports were
obtained either from the website of the relevant management agency or
directly from stock assessment scientists.  Other assessments were
obtained from the primary literature and through personal contacts at
fisheries management agencies.  Significant contributions were also
made by the other members of the National Center for Ecological
Analysis and Synthesis (NCEAS) working group ``Finding common ground
in marine conservation and management''. Relevant assessment data were
first transferred into a standardized spreadsheet template by a number
of recorders, including ourselves, assessment authors, our NCEAS
collaborators, and associated graduate students and postdoctoral
researchers, and then uploaded into the relational database management
system by the first author.

\subsection*{Database structure and advantages}
The database is implemented in the Open Source PostgreSQL relational
database management system (RDBMS) \citep{postgresql:2009}, and includes tables for the assessment metadata,
time series values, time series units, and biometrics (a catch-all term
for data, such as life history characteristics or BRPs, that are not
part of a time series). The entity relationship diagram of the database and its component tables can be found in the Supplementary Materials.
%tables is shown in the  (Figure~\ref{fig:ERD}).

RDBMSs form the server back-end to many applications of
interest to ecologists, including web-clients and GIS software, and
have a number of advantages over spreadsheet or flat text file data
compilations.  First, housing stock assessments in an RDBMS allows
multiple users to concurrently access and extract subsets of data in
an efficient and reproducible manner. Second, with the development of
Application Programming Interfaces (APIs) that allow analytical
softwares to directly communicate and extract data from the database,
a common data environment is established, independent of one's choice
of analytical software (e.g., SAS:SAS ACCESS, Matlab: Matlab/Database,
R:RDBI/RODBC, Perl:DBI, etc.).  Users familiar with Structured Query
Language (SQL) can also query the database directly from their
analytical software of choice and the same SQL query will extract the
same data through each of these applications.  Third, data products
tailored to specific projects can be generated and stored as dynamic
(i.e. continually updated) ``views'' within the database.  These are
typically rectangular, spreadsheet-like results of an expansive query
of the relevant tables that can be readily read into all commonly-used
analytical software.  The use of views is advantageous over
manipulating spreadsheets or flat text files for importing into a
specific analytic software, which runs the risk of losing data
integrity (e.g. multiple copies) and becomes impractical with large,
non-rectangular datasets and multiple users.

\subsection*{Data integrity and quality control}
We have employed several mechanisms to ensure that the database is of
high quality. During the data recording process, assessment authors
often were contacted to clarify aspects of the assessment or to obtain
more detailed data. Time series data presented only in assessment
report figures were, for example, only entered into the database if
the exact numbers could be obtained from the assessment or its
authors. In cases where multiple models were presented in an
assessment without a preferred or ~best~ model being denoted, we
attempted to ascertain which model was preferred by the stock
assessment scientist, but included all model results whenever this was
not possible.  Once uploaded into the database all stock assessments
underwent an additional ~Quality Assurance/Quality Control~ (QA/QC)
step, to ensure that the entered data replicated that of the original
assessment document exactly. This process consisted of creating a
QA/QC summary document for each assessment, containing summary details
of the stock, a selection of biometrics and ratios for comparison
(e.g. current status relative to BRP), and time series plots of the
biomass, recruitment, and exploitation trajectories. QA/QC documents
were then returned to assessment recorders and an electronic trail of
subsequent correspondence was captured using a bug tracking system.
Recorders were responsible for checking, and where necessary
correcting, their QA/QC documents, after which all corrections were
transmitted back to the operational database and a quality controlled
flag was inserted to signify the assessment had passed the check. Only
assessments that have passed this QA/QC step are available for
subsequent analyses.

% The only exception to this is Indian Ocean bigeye tuna, a
%stock assessed by the Indian Ocean Tuna Commission (IOTC) for which we
%could not obtain the model output from the assessment authors.

\subsection*{Links to related databases}


To facilitate integration of the RAM Legacy database with related
databases, such as Fishbase \citep{Froese:Pauly:2009:fishbase} and the
Sea Around Us Project's (SAUP) global landings database
\citep{Watson:etal:2004:fandf}, each species present in the RAM Legacy
database was assigned a matching FishBase species name and species
code, a matching SAUP taxon code, and taxonomic information from the
Integrated Taxonomic Information System (ITIS) (http://www.itis.gov).
Additionally, each stock was assigned to a primary (and in some cases
secondary and tertiary) Large Marine Ecosystem (LME)
\citep{NOAA:LME64:1998}.  LMEs encompass the continental shelves of
the world's oceans and represent the most productive areas of the
oceans.  Open ocean areas beyond the continental shelves are, however,
not included in the LME classification. Large, highly migratory
oceanic species such as tuna were therefore assigned to new categories
``Atlantic High Seas'', ``Pacific High Seas'', ``Indian High Seas'',
and ``Subantarctic High Seas''.

\subsection*{Assessing the knowledge-base for commercially exploited stocks}
We assess the knowledge-base for commercially exploited stocks, as
represented by the RAM Legacy database, using a variety of metrics. To
evaluate the taxonomic scope of the database, we compare the taxonomy
of assessed stocks with the diversity of i) all marine fishes (as
represented by FishBase), and ii) marine fishes in global fisheries
catches (as represented by the species available from the SAUP
database), and discuss taxonomic biases in species included in catch
data and in populations assessed using stock assessments. We evaluate
the ecological scope of assessed stocks in terms of age at sexual
maturity as reported in the assessments, and trophic level of those
assessed stocks as reported in FishBase. We overview the types of
assessment models used, and BRPs estimated, for all stock assessments
and by management body. To determine what fraction of world
wild-capture fisheries landings come from assessed stocks, we used the
SAUP's average global fisheries catches from the most recent ten years
of available data (1995-2004); we also discuss limitations to
obtaining assessments for some of the world's major fisheries.
Comparisons between assessments and catch data at a regional level are
hampered by the geographic mismatch between stocks and FAO statistical
areas or the SAUP's Large Marine Ecosystems.

%; we
%also evaluate how the frequencies of different assessment model types
%and BRPs has changed over time

%Next, we briefly overview the types of asssessment models and BRPs
%contained within the stock assessments, and the frequencies of
%different assessment methods and BRPs overall and by management body.

\subsection*{Assessing the status of commercially exploited marine stocks}
We evaluate the status of assessed stocks overall and by management
body, using standard reference points so that all
stocks are referenced to a comparable benchmark.  Following
\cite{Froese:Proelss:2010:fandf} and \cite{Worm:etal:2009:science}, we
compare the current biomass and exploitation rate of stocks relative
to their MSY reference points, $B_{MSY}$ and $U_{MSY}$, respectively. 

We do not advocate the use of MSY targets for management, but still
report MSY-related BRPs because they are the most commonly estimated
BRP and can be used to compare multiple stocks.  For those assessments
that did not contain MSY reference points, but did include total catch
($TC_{i,s}$, $i \in {1,\ldots,n_{s}}$) and total biomass ($TB_{i,s}$,
$i \in {1,\ldots,n_{s}}$) time series data, we used a Schaefer surplus
production model to estimate total biomass and exploitation rate at
MSY ($TB_{MSY_{s}}$ and $u_{MSY}$, respectively). Surplus production
of stock $s$ in year $t$, $P_{s,t}$, is a commonly used measure of
stock productivity, representing the amount of catch that can be taken
while maintaining the biomass at a constant size, and can be
calculated as:

\begin{equation}
P_{s,t}  = TB_{s,t+1} - TB_{s,t} + TC_{s,t}
\end{equation}
where,
\begin{description}
\item $TB_{s,t}$ is the total biomass of stock $s$ in year $t$
\item $TC_{s,t}$ is the total catch of stock $s$ in year $t$
\end{description}

We fit a Schaefer surplus-production model, which is based on a logistic model of population
growth to the catch and total biomass time series data. The predicted surplus production in each
year in the Schaefer model is given by:

\begin{equation}
\hat{P_{s,t}} = \frac{4mTB_{s,t}}{K} - 4m\left(\frac{TB_{s,t}}{K} \right)^{2}
\end{equation}
where,
\begin{description}
\item $m$ is the maximum sustainable yield, equal to $rK/4$
\item $K$ is the carrying capacity or equilibrium total biomass in the absence of fishing \citep{Hilborn:Walters:1992}
\end{description}

We estimated the model parameters ($m$ and $K$) using maximum
likelihood in AD Model Builder \citep{admb} assuming that the
residuals $\epsilon_{s,t}=P_{s,t} - \hat{P_{s,t}}$ were normally
distributed.  For the Schaefer model, $B_{MSY}$ is simply $0.5K$, and the
harvest rate that results in maximum sustainable yield, $u_{MSY}$, is
$m/B_{MSY}$. Carrying capacity was constrained to be less than twice the
maximum observed total biomass.


% for the stock (s) as follows:
%\begin{equation}
%TB_{i+1,s} = TB_{i,s}*\left(1+\frac{r_{s}}{K_{s}}\right) - TC_{i,s}
%\end{equation}
%where,
%\begin{description}
%\item $r_{s}$ is the intrinsic growth rate of stock $s$
%\item $K_{s}$ is the carrying capacity of stock $s$
%\end{description}

%BRPs are obtained from: 
%\begin{equation}
%TB_{MSY_{s}} = K_{s}/2 % \left( 1 - \frac{}{r_{s}}\right)
%\end{equation}

%\begin{equation}
%MSY_{s} = r_{s}K_{s}/4
%\end{equation}

%\begin{equation}
%U_{MSY_{s}} = MSY_{s}/TB_{MSY_{s}}
%\end{equation}
%\citep{Quinn:Deriso:1999}.



%For assessments where both a biomass-based and an
%exploitation-based reference point are available, we calculate the ratio of the
%current biomass and fishing mortality to its corresponding BRP.

%We also evaluate how stock status relates to a species trophic level (obtained from FishBase).

Finally, we discuss potential applications of the database, point out
its limitations and caveats about its use, and outline directions for
future development.

Statistical analyses and plot generation were conducted with the R
Environment for Statistics and Graphics \citep{R} using the packages
RODBC \citep{R:RODBC}, KernSmooth \citep{R:KernSmooth}, xtable
\citep{R:xtable}, ape \citep{R:ape}, gsubfn \citep{R:gsubfn}, IDPmisc
\citep{R:IDPmisc}, and doBy \citep{R:doBy}.  Figure~\ref{fig:lmes} was
generated using the Generic Mapping Tools \citep{gmt}.

%% , beanplot \citep{R:beanplot} gregmisc \citep{R:gregmisc},
%%%%%%%%%%%%%%%%%%%%%%%%%
%% Results
%%%%%%%%%%%%%%%%%%%%%%%%% 
\newpage
\section*{Results}
%\subsection*{Scope of the RAM Legacy database}

%The database presented here contains ... \citep{Worm:etal:2009:science, Hutchings:etal::2010:cjfas}

\subsection*{The knowledge-base for commercially-exploited marine stocks}
In total, 324 recent stock assessments for
288 marine fish and 36
invertebrate populations are included in the RAM Legacy database
(Version 1.0, 2010; Table S1). Together these comprise time series of
catch/landings for 307 stocks (95\%),
SSB estimates for 274 stocks (85\%), and recruitment estimates for
270 stocks (83\%) (Table S1).

\subsubsection*{Management bodies and geography}
Stock assessments are derived from fisheries management bodies in
Europe, the United States, Canada, New Zealand, Australia, Russia,
South Africa and Argentina (Table~\ref{tab:mgmt}). Also included are
assessments conducted by eight Regional Fisheries Management
Organizations (RFMOs), in the Northwest Atlantic, Atlantic, Pacific
and Indian Ocean (Table~\ref{tab:mgmt}). Assessments from the United
States constitute by far the most stocks of any country or region
(n=139); assessments from the European Union's
management body, the International Council for the Exploration of the
Seas (ICES), constitute the the second greatest number of stocks
(n=63).  Whereas nations are responsible for
managing all populations within their EEZs, RFMOs typically focus on a
certain type of species (e.g.  halibut, tunas) or fisheries (e.g.
pelagic high seas) within a given area and hence assess a smaller
number of stocks.

Most assessments come from North America, Europe, Australia, New
Zealand and the high seas, while there are few from regions such as
Southeast Asia, South America, and the Indian Ocean (outside
Australian waters) (Figure~\ref{fig:lmes}). Assessments were available for 31 LMEs, with the greatest number of
assessed stocks coming from Northeast U.S. Continental Shelf (n=58),
California Current (n=35), New Zealand Shelf (n=29),
Gulf of Alaska (n=26), Celtic-Biscay Shelf (n=26), East Bering Sea (n=22)
and Southeast U.S. Continental Shelf (n=20) (Figure~\ref{fig:lmes}).

%Northeast U.S. Continental Shelf (n=88), the California Current (n=35, the East Bering Sea (n=32), the New Zealand Shelf (n=29), the Gulf of Alaska (n=28), the Celtic-Biscay Shelf (n=24) and the Newfoundland-Labrador Shelf (n=21)

\subsubsection*{Taxonomy}

Assessments for 159 species from
57 families and 20
orders are included in the database (Figure~\ref{fig:taxo:srdb}). Five
taxonomic orders (Gadiformes (n=67),
Perciformes (n=62), Pleuronectiformes (n=53),
Scorpaeniformes (n=40) and Clupeiformes (n=36)) account for
80\% of available stock assessments.  Of these, Perciformes, the most
speciose Order of marine fishes are in fact underrepresented in the
database (46\% of all marine fish species vs.  19\% of all marine
fish assessments), while the other four orders are
taxonomically overrepresented: Clupeiformes (2.1\% of marine fishes
vs.  11\% in the database), Gadiformes (3.3\% of marine fishes vs.
21\% in the database), Pleuronectiformes (4.5\% of marine fishes vs.
17\% in the database), Scorpaeniformes (8.5\% of marine fishes vs.
12\% in the database) (Figure~\ref{fig:taxo:threepanel}).

Assessed marine fish stocks in the RAM Legacy database constitute a
relatively small proportion of harvested taxa
(25\% of fish species from the SAUP database)
and an even smaller proportion of marine fish biodiversity
(1\% of fish species in FishBase;
Figure~\ref{fig:taxo:threepanel}). In turn, catches from the SAUP
database, which come from 649 species and
36 orders (Figure~\ref{fig:taxo:threepanel}),
represent only 5\% of the
12339 species and 67\%
of the 54 different orders present in FishBase
(Figure~\ref{fig:taxo:threepanel}). The diversity of harvested marine
invertebrates is clearly underrepresented in the stock assessment
database and likely in stock assessments in general.

%The paucity of marine invertebrate stock assessments means these
%species are more poorly taxonomically represented in the database than
%fishes (Figure 4XX). Only XX\% .....

%\subsubsection*{Global Fisheries}

%Table~\ref{tab:worldfisheries}

\subsubsection*{Ecology}
Assessed species span a range of ecological traits. Some life-history
information (e.g. growth, maturity, fecundity) is available for
288 of the collated assessments. In some cases, this
information is derived from biological studies, while in other cases
life-history parameters represent model assumptions (e.g., natural
mortality = 0.2) or model estimates. 

%The trophic level of assessed
%species ranged from 2 to 4.5 with a mean of
%3.7, with no apparent relationship between trophic level and stock status (Figure~\ref{fig:TL}).

%%Of these, age at sexual maturity ranged from XX to XX (n=XX, mean=XX) and . 
%was 288.

%Assessed species in the data span a range of ecological
%traits..... [need trophic level plot here; Figure 6). -assessemnts by
%trophic level: can we make a barplot showing number of stocks by
%trophic level for a) overexploited, and b) not overexploited species
%(i.e. could be on same plot but different hatching for the a vs. b.


\subsubsection*{Timespan }

%Of the 324 stock assessments, time series data of
%catch/landings were available for 307 stocks (95\%),
%of SSB for 274 stocks (85\%), and of recruitment for
%270 stocks (83\%).  

The median lengths of catch/landings, SSB, and recruitment timeseries
were 38, 34, and 33
years, respectively (Figure~\ref{fig:orca}).  The time period covered by 90\% of assessments
is: catch/landings (1967-2007), SSB
(1972-2007), recruitment (1971-2006), while that
covered by 50\% of assessments is: catch/landings
(1983-2004), SSB (1985-2005), recruitment
(1984-2003) (Figure~\ref{fig:orca}).

\subsubsection*{Stock assessment methodologies and BRPs}
%In addition to the 324 assessments in the
%database, indices of relative abundance from scientific surveys are
%available for an additional 26 stocks. 

The three most common assessment methods were
Statistical catch-at-age/length models (n=164), Virtual Population Analyses (n=91) and
Biomass dynamics model (n=44). Regionally, Virtual Population Analysis
(VPA) is still the most common assessment model for European stocks
(71\% of 63 assessments),
Canada (59\% of 22
assessments) and Argentina (83\% of
6 assessments), whereas statistical catch-at-age
and -length models are more common for the United States
(66\% of 139 assessments),
Australia (81\% of 16
assessments) and New Zealand (76\% of
29 assessments).
% (need to add a sentence here about the regional differences). 

Biomass- or exploitation-based reference points were available for
256 (81\%) and
221 (69\%)
assessments, respectively. The most commonly reported biomass-based
BRPs relate to biomass at MSY (e.g. $B_{msy}$), to ``limit'' biomass
(e.g. $B_{lim}$, a biomass level above which stocks should be
maintained) and to ``precautionary approach'' biomass (e.g.  $B_{pa}$,
a biomass level which provides an additional buffer to account for
uncertainty). Biomass and exploitation of United States' stocks under
the management of NMFS must follow MSY-based reference points whereas
other fisheries agencies use different BRPs.

\subsubsection*{Global Fisheries}
Assessments were available for 9 of the 10 largest fisheries for
individual fish stocks globally (Table~\ref{tab:worldfisheries}).
Assessments for Japanese anchovy in the East China Sea (the eighth largest species
for an individual stock, and tenth overall) were not publicly
accessible. Looking more broadly, the database contains assessments
for 17 of the 30 largest fisheries for individual fish stocks
globally, and 18 of the 40 largest fisheries globally (including those
recorded at lower taxonomic resolutions)
(Table~\ref{tab:worldfisheries}). Many of the fisheries not included
in the RAM Legacy database, especially those recorded in the SAUP
database as ``Marine fishes not identified'' (n=7), occur in
developing countries and have no known formal stock assessment
conducted for them.  From a national perspective, assessments are only
included for 2 of the top 10 wild-caught marine fisheries producing
nations, U.S.A. and Russia \citep{FAO:sofia}, with only two
assessments from the latter. We were unable to obtain any assessments
from the other top 10 yield-producing countries: China, Peru,
Indonesia, Japan, Chile, India, Thailand, Philippines
\citep{FAO:sofia}. 
%Including stock assessments from these countries would greatly benefit the database's spatial and taxonomic coverage. 


% cushion by using "estimates"

\subsubsection*{The status of commercially exploited marine stocks }
To evaluate stock status, we single out stocks for which both a biomass
BRP and an exploitation BRP are available. Of the
240 stocks presented in
Figure~\ref{fig:friedegg}, 112 and
128 of the biomass reference points and
83 and
157 of the exploitation reference
points come from assessments and from surplus production model fits,
respectively.  To identify potential biases arising from using BRPs
derived from surplus production models we computed a contingency table
of status classification for stocks that have both assessment- and
Schaefer-derived BRPs (Table S2). Surplus production models correctly
classified ratios of current biomass to BRPs in
69\% of cases (for 67
of 97 assessments) and 62\%
of cases for exploitation BRPs (for 37 of
60 assessments).

%For the stocks where both are available, we found a correlation of 0.67 and 0.61 between assessment BRP and surplus production model BRP for biomass (n=97) and exploitation (n=60), respectively (Supplementary Figure S1). 

% BRPs derived from surplus production models tended to underestimate $B/B_{msy}$ and overestimate $U/U_{msy}$.

Overall, 57\% of stocks are estimated
to be below their biomass-related MSY BRP, that is $B_{curr}<B_{msy}$,
and 30\% are estimated to be above
their exploitation-related MSY BRP, $U_{curr}>U_{msy}$
(n=240 stocks total; Figure~\ref{fig:friedegg}).
Of the stocks for which biomass is currently estimated to be below
$B_{msy}$, 55\% have had their
exploitation rate reduced below $U_{msy}$, suggesting potential for
recovery (Figure~\ref{fig:friedegg}). The remaining
45\% of these stocks however,
still have excessive exploitation rates (Figure~\ref{fig:friedegg}).
On a positive note, 43\% of all stocks are
estimated to be above $B_{msy}$, and
91\% of the stocks above
$B_{msy}$ also have $U_{current}$ below $U_{msy}$.

% put contingency table instead of the correlation

% surplus production model systematically provide 

%There was no significant
%difference in the status of stocks with assessment-derived BRPs (n=62,
%solid dots in Figures ~\ref{fig:friedegg} and ~\ref{fig:friedeggmgmt}) vs. Schaefer-estimated BRPS (n=178,
%open circles in Figures ~\ref{fig:friedegg} and ~\ref{fig:friedeggmgmt; p<XX).).

The status of exploited marine stocks, as estimated from biomass- and
exploitaion-BRPs, varied widely depending on the management body
(Figure~\ref{fig:friedeggmgmt}). Most European stocks (managed by
ICES) have biomasses less than $B_{msy}$
(79\%), and over half of these
stocks (61\%) still
have exploitation rates exceeding $U_{msy}$. Canadian stocks (managed
by DFO) also had low biomass (79\%
$< B_{msy}$), but all but one of these has had its exploitation rate
reduced below $U_{msy}$. In contrast, about half
(21\%) of U.S. stocks (managed by
NMFS) are estimated to still be above $B_{msy}$, and of the
40 stocks that are below $B_{msy}$
65\% have exploitation
rates below $U_{msy}$ (Figure~\ref{fig:friedeggmgmt}). In the New
Zealand and Australian waters, stocks managed by MFish and AFMA are
above $B_{msy}$ in 21\% and
42\% of cases, respectively. For
the stocks grouped as ``Atlantic'' in Figure~\ref{fig:friedeggmgmt} we
found that 6 of the
10 ICCAT stocks and
6 of the
10 of NAFO stocks were below $B_{msy}$ .

%Species under international management include tuna stocks in the
%Atlantic, Pacific, Indian Oceans. 


%From these assessments,
%REF:SQL:NUMASSESSBIOANDEXPLOITREF report both a biomass-based and an
%exploitation-based BRP and appear as solid dots on
%Figures~\ref{fig:friedegg} and ~\ref{fig:friedeggmgmt}.
%Schaefer-derived BRPs add an additional
%REF:SQL:NUMADDITIONALASSESSSCHAEFER assessments, for a total of
%240 assessments used to generate
%Figure~\ref{fig:friedegg}. Overall,
%57\% of assessed stocks are below
%their biomass-related MSY BRP and
%30\% are above their
%exploitation-related MSY BRP. Different management bodies have
%different overall status of current biomass to BRPs
%(Figure~\ref{fig:friedeggmgmt}).


%Status of Assessed Stocks 
%Need to know:
%\% of stocks with biomass below Bmsy
%\% of stocks with
%overall and by management body.
 




\newpage
\section*{Discussion}
\subsection*{The knowledge-base and status of commercially exploited marine stocks}
The RAM Legacy Database provides detailed time series and point data
from available stock assessments for the world's industrial marine
fisheries, thus providing a basis for evaluating the existing
knowledge-base and current status of these fisheries. Accessible stock
assessments are predominantly from developed nations in
north-temperate regions, and tend to cover only the past few decades,
typically a significantly shorter period than that for which the stock
has been exploited. The taxonomic makeup of available assessments is a
very limited subset of the accepted taxonomic coverage of marine
species worldwide, and of globally exploited species. Most notably
(with the exception of tunas), assessment-based knowledge is not
available for coral reef and other tropical fishes. Inshore (e.g.
estuarine species) and anadramous populations are also noteworthy in
their absence (as a result of our focus on federally or
internationally managed marine species) and, as such, any assessment
of global status of exploited populations must be interpreted only for
that subset of exploited species for which assessments are present in
the database.
% [need to briefly discuss the �Stock Status� results]


Note that BRPs derived from surplus production models are to be
interpreted with great care. For stocks where both were avaialble, we
compared the values of assessment BRPs and Schaefer-derived BRPs
(Figure S2) and found correlations of XX\% between $B_{msy}$ BRPs and
XX\% between $U_{msy}$ BRPs.


\subsection*{Biases in the knowledge-base for commercially exploited marine stocks}

\subsubsection*{Geographic bias}
Bias in the geographic scope of the RAM Legacy database (relative to
that of all fisheries globally) may arise for several reasons, all of
which vary geographically in their prevalence: 1. an assessment is not
conducted on a stock; 2. it is not possible to access the assessment;
or 3. the non-exhaustive collation we undertook overlooked the
assessment. Whether an assessment is conducted for a given stock
depends upon a multitude of factors, including the economic value of
the stock, the availability of fiscal resources to collect the data
required for an assessment (which frequently includes conducting
fisheries-independent research surveys) and the expertise to conduct
assessments. In general, conducting stock assessments is a costly
endeavour that is restricted to wealthy fishing nations. The legal
context where fisheries are prosecuted can also strongly influence the
requirement for conducting stock assessments. In the United States,
the Magnuson-Stevens Act defines which stocks are to be monitored and
managed, hence a large number of the assessments in the RAM Legacy
database are under the jurisdiction of the US National Marine
Fisheries Services. How accessible assessments are for entry depends
upon the transparency and access policies of the relevant management
agencies, which also varies geographically. Our incomplete search for
assessments could also give rise to geographic biases, as concerted
collation efforts have only been conducted in those known
assessment-rich regions.  It is hoped that readers of this article can
assist in correcting these biases by participating in future updates
of the RAM Legacy database, and that the development of this database
will encourage greater transparency amongst fishing nations.


\subsubsection*{Taxonomic bias}
Related to geographic bias is the taxonomic bias in those species that
are known, caught and assessed. At a broad level the Gadiformes and
Clupeiformes occupy disproportionate taxonomic representation in the
catch compared to overall species occurrence (Figure~\ref{fig:taxo:threepanel}, panels a and
b). Taxonomic biases at this level may reflect behavioural tendencies
of the over-represented species in the catch to form large aggregated
populations in temperate regions that are readily accessible to
fishing. Consumer preferences may also be an important determinant of
what taxonomic groups are more likely to be caught.  The
over-representation of the Gadiformes and, to a lesser degree, the
Clupeiformes, continues when caught and assessed taxa are compared
(Figure~\ref{fig:taxo:threepanel}, panels b and c). Historical economic importance as well as
the geographic distribution of the taxa in relation to mandated
assessments may play important roles in determining what fished taxa
are assessed.  Even in developed countries, however, not all stocks
are assessed. For example, in 2007, of the 528 fish and invertebrate
stocks recognized by the National Marine Fisheries Service (NMFS),
only 179, or slightly over one-third, were fully assessed \citep{NMFS:2008:status}. An assessment by the European
Environment Agency (EEA) in 2006 indicated that the percentage of
commercial landings obtained from assessed stocks ranged between 66-97
percent in northern European waters and 30-77 percent in the
Mediterranean \citep{eea:2009:status}. The New Zealand
Ministry of Fisheries reports the status of 117 stocks or sub-stocks
out of a total of 628 stocks managed under New Zealand's Quota
Management System \citep{NZMF:2009}. In
Australia, 98 federally managed stocks have been assessed
\citep{Wilson:etal:2009:status} out of an unknown total. The extent to
which stocks are assessed elsewhere in the world is currently unknown.

\subsubsection*{Temporal bias}
Most of the assessments in the RAM Legacy database contain time series
of 30 years or less whereas industrial fishing began long before this.
Dominant age-structured assessment methodologies typically rely on
catch-at-age data, which are often available for considerably shorter
periods of time than total catch unless significant reconstruction
efforts are made.  Such historical reconstructions of catch-at-age
data are highly uncertain and in many cases the ``base case'' models
used for management are based only on more reliable recent catch data.
For assessments used in a tactical sense and for short-term
projection, e.g., to understand whether a particular quota level will
result in an increase or decrease in stock size, using only reliable
recent catch data may be preferable.  This is particularly true for
backward projection methods (e.g., VPA), which may converge on
parameter estimates within the more reliable recent period and
potentially benefit little from reaching further back in time.
Nevertheless, a focus on only the recent history of a fishery can be
seriously misleading for strategic decisions about goals and BRPs.
Put simply, if we don't know what's historically possible (in terms of
stock size), it's hard to know where we should set our goals.  This
``shifting baseline'' problem has been widely recognized
\citep{Pauly:1995:tree, SaenzArroyo:etal::2005:procB}, but is still apparent in the
relatively short time series of most assessments.

%.)  -what are the
%implications of this? Shifting baselines..(recent Callum Roberts 2010
%North Sea Nature paper; Jennings and Blanchard 2004 paper)

\subsubsection*{Future applications of the RAM Legacy database}
We anticipate that this new database will be of utility for fisheries
scientists, ecologists, and marine conservation biologists interested
in conducting comparative analyses of global fisheries status,
collapse and recovery patterns, fisheries productivity or marine
population dynamics. In addition to the initial aim of providing
reliable access to time series information about stocks, we hope to
also stimulate research in the relationships of life-history
characteristics and their relation to exploitation. The RAM Legacy
database contains the corresponding species codes to the Sea Around Us
Project and FishBase, thus facilitating researchers' use of a global
fisheries data ``toolkit'' to address questions on the relationships
between life history attributes and resulting population dynamics in
an exploited setting.

\subsubsection*{Caveats and limitations}
Stock assessment outputs (e.g. biomass time series), which comprise the
majority of the new RAM Legacy database are model estimates, not raw
data. The uncertainty associated with these estimates should be
carried forth in subsequent analyses. Although the database structure
allows for inclusion of estimates of uncertainty (standard errors,
95\% credible/confidence intervals), because these estimates were
typically missing from assessments, either because they weren�t
produced by the assessment model (e.g. non-bootstrapped VPA
assessments) or the focus of the assessment document was on central
tendency (e.g. mean biomass) not the associated uncertainty, they have
not been included in this first version of the database. Note that
this view of assessment uncertainty is changing with the advent of
MCMC approaches to Bayesian inference for assessments, bootstrap
methods, statistical catch-at-age models \citep{admb} and a
general focus on uncertainty \citep{Walters:Maguire:1996:reviews}. As with any
analysis, clearer inference on the strength of a signal is available
when all uncertainty in the data is carried forth. This represents a
difficulty for synthetic analyses of fisheries data in that in an
ideal world one would access the raw data for each stock and carry
forth the uncertainty at all levels of the analysis. In the case of
assessments, the raw data is typically catch-at-age matrices and
potentially survey indices. To understand the fleet characteristics
and survey stratification schema for each stock in a potentially
global meta-analysis would be extremely time consuming and
error-prone. Instead, the expert opinion of those researchers most
familiar with the data, stock assessment authors, is used, while
recognizing that without accompanying uncertainty estimates the
strength of conclusions drawn may be weakened.  

The original database developed by Ransom A. Myers was used to address
a variety of ecological questions derived from stock-recruit
relationships. This was possible because the VPA-type assessment
models that comprised most of that database generated time series of
stock and recruitment with relatively few a priori assumptions.
Forward projection methods generally specify the form of the
stock-recruit relationship, and in many cases even fix parameters
(infinitely dense point prior) such as steepness.  Stock-recruitment
``data'' from such models, are clearly inappropriate for
straightforward meta-analysis.

More generally, meta-analysis may become the victim of its own
success.  As more assessments incorporate some type of prior
information from other stocks or species
\citep{Hilborn:Liermann:1998:reviews}, there is less stock-specific
information available for future meta-analysis.  One solution is for
stock assessments to report not only best estimates of parameters
based on all available data, but also stock-specific parameter
estimates that do not incorporate prior information from other stocks
or species.  Similarly, state-space models represent a significant
advance over observation error models.  Yet, variability in
state-space model outputs such as biomass time series often reflects
assumptions about the relative contributions of process and
observation error.

\subsubsection*{Future development}

We anticipate that the RAM Legacy database will continue to grow with
hitherto unentered stocks e.g. freshwater and anadramous populations,
particularly the Salmonidae that comprised 45\% of the stocks in the
original Myers Stock Recruitment Database, and updated assessments for
already included stocks. Future versions of the database will also
include timelines of management actions per stock, as well as
age-varying and length-varying data such as maturity ogives and
age-disaggregated natural mortality. Depending on availability,
subsequent releases of the database could also include estimates of
assessment uncertainty. Future database products will include
management-agency-level reports containing summaries of all stocks
within their remit.  The development of a standard for assessment
reporting at the management agency level would greatly assist in the
acquisition of new assessments, and hence to ensure that the database
remains current.  For example, ICES assessments have a very regular
standard, including agreed-upon reference points and regular estimate
reporting. This makes the process of data collation much more routine
than unstandardized documents where the recorder trawls through a
report for the relevant information. ICES also has a central database
of assessments for stocks of the region. Certainly different stocks
and regions require different formats but basic output tables,
consisting of total and spawning biomass, recruitment, catch/landings,
estimated fishing mortality over vulnerable age groups, associated
measures of uncertainty, and commonly-used reference points would
streamline the process immensely. A process whereby the assessment
spreadsheets are filled out at each assessment meeting would
facilitate the process even further and be the least error prone
method. In return, the assessment scientists can access results for a
global collation of assessments to further their own research
initiatives in population assessment and management. The ultimate goal
is to provide a comprehensive stock assessment database for
researchers to use results from multiple regions to assist in their
own applied and fundamental research in population ecology, fisheries
science, and conservation biology.


%

\newpage

\section*{Availability of the database} 
Contributions or corrections to the existing database, as well as
requests to use the database (subject to standard ``Fair Use''
policies), should be directed to the corresponding author. 


\section*{Acknowledgments }
We sincerely thank all of the fisheries scientists whose assessments form
the basis of this new global database. We are also grateful for the
database contributions, advice, and support of Trevor Branch, Jeremy
Collie, Laurence Fauconnet, Mike Fogarty, Rainer Froese, Ray Hilborn,
Jeff Hutchings, Simon Jennings, Heike Lotze, Pamela Mace, Michael
Melnychuk, Ana Parma, Ren\'{e}e Pr\'{e}fontaine, Kate Stanton, Reg Watson,
Boris Worm, Dirk Zeller, and the financial support of the National
Science Foundation through an NCEAS Working Group, the Natural
Sciences and Engineering Research Council (NSERC) of Canada, the
Canadian Foundation for Innovation, the David H. Smith Conservation
Research Fellowship, the Schmidt Research Vessel Institute, and the
Census of Marine Life (CoML/FMAP).


\newpage
%\bibliographystyle{plain}
%\bibliographystyle{authordate1}
\bibliographystyle{fishandfisheriesBST}

\bibliography{./fishfisheries}

%\section*{Tables}

%\noindent Number of assessments included in the RAM Legacy database by
%country and ocean basin, with associated national management bodies
%and regional fisheries management organizations (RFMOs).\\

\begin{table}
\caption{Number of assessments included in the RAM Legacy database by
country and ocean basin, with associated national management bodies
and regional fisheries management organizations (RFMOs).}
\begin{tabular}{| l | p{7cm} | l | r |}\label{tab:mgmt}
\textit{Country/Ocean} & \textit{Management Body} & \textit{Acronym} & \textit{No. stocks} \\
\hline \hline
Argentina & Consejo Federal Pesquero & CFP & 6 \\ \hline
Australia & Australian Fisheries Management Authority & AFMA & 16 \\ \hline
Canada & Department of Fisheries and Oceans & DFO & 22 \\ \hline
Europe & International Council for the Exploration of the Sea & ICES & 63 \\ \hline
New Zealand & Ministry of Fisheries & MFish & 29 \\ \hline
Russia & Russian Federal Fisheries Agency & RFFA & 2 \\ \hline
South Africa & Department of Environment and Tourism, Marine and Coastal Management & DETMCM & 14 \\ \hline
USA & National Marine Fisheries Service & NMFS & 137 \\ \hline
USA & US state-level management & US State & 3 \\ \hline
Atlantic Ocean & International Commission for the Conservation of Atlantic Tunas & ICCAT & 10 \\ \hline
 & Northwest Atlantic Fisheries Organization & NAFO & 8 \\ \hline
Indian Ocean & Indian Ocean Tuna Commission & IOTC & 1 \\ \hline
Pacific Ocean & Inter-American Tropical Tuna Commission & IATTC & 2 \\ \hline
 & International Pacific Halibut Commission & IPHC & 1 \\ \hline
 & South Pacific Regional Fisheries Management Organization & SPRFMO & 1 \\ \hline
 & Western and Central Pacific Fisheries Commission & WCPFC & 4 \\ \hline
Antarctic & Commission for the Conservation of Antarctic Marine Living Resources & CCAMLR & 1 \\ \hline
\end{tabular}
\end{table}

%\noindent The world's forty largest wild-caught fisheries (comprising
%less than 41\% of total global catches, based on average catches
%1995-2004 in SAUP database), and the thirty largest fisheries of
%individual stocks (i.e. fisheries identified to the species level;
%comprising more than 32\% of total global catches), including their
%LME, whether or not stock assessments for them are included in the RAM
%Legacy database, and the reason if not included (e.g. 1= no known
%assessment, 2=assessment is not based on a population dynamics model,
%3=assessment inaccessible).

%\begin{table}
\begin{longtable}{p{2cm} | p{2cm} | p{5cm} | l | p{2cm} | p{2cm}}
  \bottomrule \\ \multicolumn{2}{c}{Continued on next page} \endfoot \endlastfoot
\textit{Stock Rank} & \textit{Stock Number} & \textit{Species (Common name, Latin name) or higher taxonomic unit} & \textit{LME} & \textit{In Database?} & \textit{Reason if not included} \\ \midrule \endhead
1 & 1& Peruvian anchoveta, \textit{Engraulis ringens} & Humboldt Current & no & 3\\
 & 2& Marine fishes not identified & South China Sea & no & 1\\
 & 3& Marine fishes not identified & Bay of Bengal & no & 1\\ \hline
2 & 4& Alaska pollock, \textit{Theragra chalcogramma} & Okhotsk Sea & yes & \\ \hline
3 & 5& \textit{Ammodytes} & North Sea & yes & \\ \hline
4 & 6& Atlantic herring, \textit{Clupea harengus} & Norwegian Sea & yes & \\ \hline
5 & 7& Alaska pollock, \textit{Theragra chalcogramma} & East Bering Sea & yes & \\ \hline
6 & 8& Capelin, \textit{Mallotus villosus} & Iceland Shelf/Sea & yes & \\ \hline
7 & 9& European pilchard, \textit{Sardina pilchardus} & Canary Current & yes & \\ \hline
8 & 10& Japanese anchovy, \textit{Engraulis japonicus} & East China Sea & no & 3\\ \hline
9 & 11& Inca scad, \textit{Trachurus murphyi} & Humboldt Current & yes & \\ \hline
 & 12& Marine fishes not identified & East China Sea & no & 1\\ \hline
10 & 13& Gulf menhaden, \textit{Brevoortia patronus} & Gulf of Mexico & yes & \\ 
 & 14& Marine fishes not identified & Yellow Sea & no & 1\\
 & 15& Marine fishes not identified & Indonesian Sea & no & 1\\ \hline
11 & 16& Alaska pollock, \textit{Theragra chalcogramma} & Gulf ofAlaska & yes & \\ \hline
12 & 17& Argentinean short-finned squid, \textit{Illex argentinus} & Patagonian Shself & no & 1\\ \hline
13 & 18& Argentine hake, \textit{Merluccius hubbsi} & Patagonian Shelf & yes & \\ \hline
14 & 19& Japanese anchovy, \textit{Engraulis japonicus} & South China Sea & no & 1\\ \hline
15 & 20& Araucanian herring, \textit{Strangomera bentincki} & Humboldt Current & no & \\ \hline
16 & 21& Atlantic cod, \textit{Gadus morhua} & Barents Sea & no & \\ \hline
17 & 22& European sprat, \textit{Sprattus sprattus} & Baltic Sea & yes & \\ \hline
18 & 23& Atlantic herring, \textit{Clupea harengus} & North Sea & yes & \\ \hline
19 & 24& Alaska pollock, \textit{Theragra chalcogramma} & Arctic Ocean & no & \\ \hline
 & 25& Marine fishes not identified & Gulf of Thailand & no & \\
20 & 26& Atlantic herring, \textit{Clupea harengus} & Baltic Sea & yes & \\ \hline
21 & 27& Cape horse mackerel, \textit{Trachurus capensis} & Benguela Current & yes & \\ \hline
22 & 28& Largehead hairtail, \textit{Trichiurus lepturus} & East China Sea & no & \\ \hline
23 & 29& Japanese anchovy, \textit{Engraulis japonicus} & Yellow Sea & no & \\ \hline
24 & 30& European anchovy, \textit{Engraulis encrasicolus} & Black Sea & no & \\ \hline
25 & 31& Chub mackerel, \textit{Scomber japonicus} & East China Sea & no & \\ \hline
26 & 32& Indian oil sardine, \textit{Sardinella longiceps} & Arabian Sea & no & 1\\ \hline
 & 33& \textit{Decapterus} & South China Sea & no & \\
 & 34& \textit{Sciaenidae} & Arabian Sea & no & \\
27 & 35& Atlantic mackerel, \textit{Scomber scombrus} & North Sea & yes & \\ \hline
28 & 36& Largehead hairtail, \textit{Trichiurus lepturus} & Yellow Sea & no & \\ \hline
 & 37& \textit{Merluccius} & Benguela Current & yes & \\
 & 38& Marine fishes not identified & Kuroshio Current & no & \\
29 & 39& Alaska pollock, \textit{Theragra chalcogramma} & Sea of Japan & no & \\ \hline
30 & 40& Round sardinella, \textit{Sardinella aurita} & Canary Current & no & \\ \hline
\caption{The world's forty largest wild-caught fisheries (comprising less than 41\% of total global catches, based on average catches 1995-2004 in SAUP database), and the thirty largest fisheries of individual stocks (i.e. fisheries identified to the species level; comprising more than 32\% of total global catches), including their LME, whether or not stock assessments for them are included in the RAM Legacy database, and the reason if not included (e.g. 1= no known assessment, 2=assessment is not based on a population dynamics model, 3=assessment inaccessible).}\\
\label{tab:worldfisheries}
\end{longtable}
%\end{tabular}
%\end{table}






\section*{Tables}

% latex table generated in R 2.9.1 by xtable 1.5-6 package
% Mon May 31 22:01:26 2010
\begin{table}[ht]
\begin{center}
\caption{Number of assessments included in the RAM Legacy database}
\label{tab:mgmt}
\begin{tabular}{p{3cm}p{5cm}cc}
\textit{Country/Ocean} & \textit{Management Body} & \textit{Acronym} & \textit{No. stocks} \\ \midrule
Argentina & Consejo Federal Pesquero & CFP &   6 \\ 
  Australia & Australian Fisheries Management Authority & AFMA &  16 \\ 
  Canada & Department of Fisheries and Oceans & DFO &  22 \\ 
  Multinational & Indian Ocean Tuna Commission & IOTC &   1 \\ 
  Multinational & Unknown management body & UNKNOWN &   1 \\ 
  Multinational & International Commission for the Conservation of Atlantic Tunas & ICCAT &  10 \\ 
  Multinational & Inter-American Tropical Tuna Commission & IATTC &   2 \\ 
  Multinational & International Pacific Halibut Commission & IPHC &   1 \\ 
  Multinational & International Council for the Exploration of the Sea & ICES &  63 \\ 
  Multinational & South Pacific Regional Fisheries Management Organization & SPRFMO &   1 \\ 
  Multinational & Commission for the Conservation of Antarctic Marine Living Resources & CCAMLR &   1 \\ 
  Multinational & Northwest Atlantic Fisheries Organization & NAFO &   9 \\ 
  Multinational & Western and Central Pacific Fisheries Commission & WCPFC &   4 \\ 
  New Zealand & Ministry of Fisheries & MFish &  29 \\ 
  Russia & Russian Federal Fisheries Agency & RFFA &   2 \\ 
  South Africa & South African national management & DETMCM &  14 \\ 
  USA & US state-level management & US State &   3 \\ 
  USA & National Marine Fisheries Service & NMFS & 139 \\ 
   \hline
\end{tabular}
\end{center}
\end{table}


\begin{landscape}
\begin{longtable}{p{2cm} | p{2cm} | p{5cm} | l | p{2cm} | p{2cm}}
\textit{Stock Rank} & \textit{Stock Number} & \textit{Species (Common name, Latin name) or higher taxonomic unit} & \textit{LME} & \textit{In Database?} & \textit{Reason if not included} \endhead
\hline
1 & 1& Peruvian anchoveta, \textit{Engraulis ringens} & Humboldt Current & no & 3\\
 & 2& Marine fishes not identified & South China Sea & no & 1\\
 & 3& Marine fishes not identified & Bay of Bengal & no & 1\\ \hline
2 & 4& Alaska pollock, \textit{Theragra chalcogramma} & Okhotsk Sea & yes & \\ \hline
3 & 5& \textit{Ammodytes} & North Sea & yes & \\ \hline
4 & 6& Atlantic herring, \textit{Clupea harengus} & Norwegian Sea & yes & \\ \hline
5 & 7& Alaska pollock, \textit{Theragra chalcogramma} & East Bering Sea & yes & \\ \hline
6 & 8& Capelin, \textit{Mallotus villosus} & Iceland Shelf/Sea & yes & \\ \hline
7 & 9& European pilchard, \textit{Sardina pilchardus} & Canary Current & yes & \\ \hline
8 & 10& Japanese anchovy, \textit{Engraulis japonicus} & East China Sea & no & 3\\ \hline
9 & 11& Inca scad, \textit{Trachurus murphyi} & Humboldt Current & yes & \\ 
 & 12& Marine fishes not identified & East China Sea & no & 1\\ \hline
10 & 13& Gulf menhaden, \textit{Brevoortia patronus} & Gulf of Mexico & yes & \\ 
 & 14& Marine fishes not identified & Yellow Sea & no & 1\\
 & 15& Marine fishes not identified & Indonesian Sea & no & 1\\ \hline
11 & 16& Alaska pollock, \textit{Theragra chalcogramma} & Gulf ofAlaska & yes & \\ \hline
12 & 17& Argentinean short-finned squid, \textit{Illex argentinus} & Patagonian Shself & no & 1\\ \hline
13 & 18& Argentine hake, \textit{Merluccius hubbsi} & Patagonian Shelf & yes & \\ \hline
14 & 19& Japanese anchovy, \textit{Engraulis japonicus} & South China Sea & no & 1\\ \hline
15 & 20& Araucanian herring, \textit{Strangomera bentincki} & Humboldt Current & no & \\ \hline
16 & 21& Atlantic cod, \textit{Gadus morhua} & Barents Sea & no & \\ \hline
17 & 22& European sprat, \textit{Sprattus sprattus} & Baltic Sea & yes & \\ \hline
18 & 23& Atlantic herring, \textit{Clupea harengus} & North Sea & yes & \\ \hline
19 & 24& Alaska pollock, \textit{Theragra chalcogramma} & Arctic Ocean & no & \\ \hline
 & 25& Marine fishes not identified & Gulf of Thailand & no & \\
20 & 26& Atlantic herring, \textit{Clupea harengus} & Baltic Sea & yes & \\ \hline
21 & 27& Cape horse mackerel, \textit{Trachurus capensis} & Benguela Current & yes & \\ \hline
22 & 28& Largehead hairtail, \textit{Trichiurus lepturus} & East China Sea & no & \\ \hline
23 & 29& Japanese anchovy, \textit{Engraulis japonicus} & Yellow Sea & no & \\ \hline
24 & 30& European anchovy, \textit{Engraulis encrasicolus} & Black Sea & no & \\ \hline
25 & 31& Chub mackerel, \textit{Scomber japonicus} & East China Sea & no & \\ \hline
26 & 32& Indian oil sardine, \textit{Sardinella longiceps} & Arabian Sea & no & 1\\ \hline
 & 33& \textit{Decapterus} & South China Sea & no & \\
 & 34& \textit{Sciaenidae} & Arabian Sea & no & \\
27 & 35& Atlantic mackerel, \textit{Scomber scombrus} & North Sea & yes & \\ \hline
28 & 36& Largehead hairtail, \textit{Trichiurus lepturus} & Yellow Sea & no & \\ \hline
 & 37& \textit{Merluccius} & Benguela Current & yes & \\
 & 38& Marine fishes not identified & Kuroshio Current & no & \\
29 & 39& Alaska pollock, \textit{Theragra chalcogramma} & Sea of Japan & no & \\ \hline
30 & 40& Round sardinella, \textit{Sardinella aurita} & Canary Current & no & \\ \hline
\caption{The world's forty largest wild-caught fisheries (comprising less than 41\% of total global catches, based on average catches 1995-2004 in SAUP database), and the thirty largest fisheries of individual stocks (i.e. fisheries identified to the species level; comprising more than 32\% of total global catches), including their LME, whether or not stock assessments for them are included in the RAM Legacy database, and the reason if not included (e.g. 1= no known assessment, 2=assessment is not based on a population dynamics model, 3=assessment inaccessible).}\\
\label{tab:worldfisheries}
\end{longtable}
\end{landscape}


\include{Figures}
%\appendix
%\section*{Supporting Information}

%\subsection*{Entity relationship diagram}
%\begin{figure}
%\begin{center}
%\includegraphics[width=15cm]{/home/srdbadmin/SQLpg/srdb/trunk/doc/srdb-ERD.pdf}
%\end{center}
%\caption{Entity relationship diagram of the RAM legacy database.}\label{fig:ERD}
%\end{figure}

%%% latex table generated in R 2.9.1 by xtable 1.5-6 package
% Thu Jun 10 10:21:47 2010
\begin{longtable}{p{1.8cm}p{4cm}p{4cm}ccccp{1.9cm}c}
  \hline
mgmt & stock & scientificname & timespan & currentyear & Bratio & Uratio & fromassessment & ref \\ 
  \hline
AFMA & Deepwater flathead Southeast Australia & \textit{Platycephalus conatus} & 1978-2007 & 2006 & 1.33 & 0.61 & no & \cite{NA} \\ 
  AFMA & common gemfish Southeast Australia & \textit{Rexea solandri} & 1966-2007 & 2007 & 0.10 & 0.39 & no & \cite{NA} \\ 
  AFMA & Jackass morwong Southeast Australia & \textit{Nemadactylus macropterus} & 1913-2007 & 2007 & 0.25 & 1.80 & no & \cite{NA} \\ 
  AFMA & New Zealand ling Eastern half of Southeast Australia & \textit{Genypterus blacodes} & 1968-2007 & 2007 & 0.60 & 2.20 & no & \cite{NA} \\ 
  AFMA & Orange roughy Cascade Plateau & \textit{Hoplostethus atlanticus} & 1987-2006 & 2006 & 1.76 & 0.34 & no & \cite{CSIRO-Cascade-Plateau-Stock-Assessment-2006.pdf} \\ 
  AFMA & Orange roughy Southeast Australia & \textit{Hoplostethus atlanticus} & 1978-2007 & 2006 & 0.89 & 0.29 & no & \cite{NA} \\ 
  AFMA & Silverfish Southeast Australia & \textit{Seriolella punctata} & 1978-2006 & 2006 & 1.15 & 0.79 & no & \cite{NA} \\ 
  AFMA & School whiting Southeast Australia & \textit{Sillago flindersi} & 1945-2007 & 2007 & 1.10 & 0.82 & no & \cite{NA} \\ 
  AFMA & Tiger flathead Southeast Australia & \textit{Neoplatycephalus richardsoni} & 1913-2006 & 2006 & 1.05 & 0.00 & no & \cite{NA} \\ 
  AFMA & Blue Warehou Eastern half of Southeast Australia & \textit{Seriolella brama} & 1984-2006 & 2006 & 0.17 & 0.84 & no & \cite{NA} \\ 
  AFMA & Blue Warehou Western half of Southeast Australia & \textit{Seriolella brama} & 1984-2006 & 2006 & 0.62 & 2.04 & no & \cite{NA} \\ 
  AFMA & Tasmanian giant crab Tasmania & \textit{Pseudocarcinus gigas} & 1990-2007 & 2007 & 0.50 & 1.71 & no & \cite{JENSEN_TASGIANTCRAB_2008.pdf} \\ 
  CCAMLR & Antarctic toothfish Ross Sea & \textit{Dissostichus mawsoni} & 1995-2007 & 2007 & 1.76 & 1.09 & no & \cite{NA} \\ 
  CFP & Argentine anchoita Northern Argentina & \textit{Engraulis anchoita} & 1989-2007 & 2007 & 1.37 & 0.17 & yes & \cite{Hansen-ANCHOVY-N-2007.pdf} \\ 
  CFP & Argentine anchoita Southern Argentina & \textit{Engraulis anchoita} & 1992-2007 & 2007 & 3.13 & 0.04 & yes & \cite{Hansen-ANCHOVY-S-2007.pdf} \\ 
  CFP & Argentine hake Northern Argentina & \textit{Merluccius hubbsi} & 1985-2007 & 2007 & 0.16 & 1.26 & yes & \cite{Irusta-hake-N-2007.pdf} \\ 
  CFP & Argentine hake Southern Argentina & \textit{Merluccius hubbsi} & 1985-2008 & 2008 & 0.34 & 1.49 & yes & \cite{Renzi-hake-S-2009.pdf} \\ 
  CFP & Patagonian grenadier Southern Argentina & \textit{Macruronus magellanicus} & 1983-2006 & 2006 & 1.82 & 0.60 & yes & \cite{Giussi-hoki-2007.pdf} \\ 
  DETMCM & Anchovy South Africa & \textit{Engraulis encrasicolus} & 1984-2006 & 2006 & 0.97 & 0.36 & no & \cite{.pdf} \\ 
  DETMCM & Shallow-water cape hake South Africa & \textit{Merluccius capensis} & 1917-2008 & 2008 & 1.16 & 0.40 & no & \cite{SA-Mcapensis-2008_IWS_DEC08_H_5.pdf} \\ 
  DETMCM & Cape horse mackerel South Africa South coast & \textit{Trachurus capensis} & 1950-2007 & 2007 & 1.47 & 0.76 & no & \cite{Johnston-SAHorseMackerel-2007.pdf} \\ 
  DETMCM & Kingklip South Africa & \textit{Engraulis encrasicolus} & 1932-2008 & 2008 & 1.13 & 0.55 & no & \cite{Branch-SA-Kingklip-2008.pdf} \\ 
  DETMCM & Sardine South Africa & \textit{Sardinops sagax} & 1984-2006 & 2006 & 0.75 & 0.55 & no & \cite{deMoorSASardineAssessment-Sep07.pdf} \\ 
  DETMCM & Southern spiny lobster South Africa South coast & \textit{Palinurus gilchristi} & 1973-2008 & 2008 & 0.51 & 1.50 & no & \cite{Johnston-SASouthRockLobster-2008.pdf} \\ 
  DFO & Atlantic cod NAFO 5Zjm & \textit{Gadus morhua} & 1978-2003 & 2002 & 0.34 & 0.45 & no & \cite{NAFO-COD5Zjm-2003.pdf} \\ 
  DFO & Haddock NAFO-4X5Y & \textit{Melanogrammus aeglefinus} & 1960-2003 & 2003 & 0.85 & 0.33 & no & \cite{NAFO-HAD4X5Y-2003.pdf} \\ 
  DFO & Haddock NAFO-5Zejm & \textit{Melanogrammus aeglefinus} & 1969-2003 & 2002 & 1.00 & 0.65 & no & \cite{NAFO-HAD5Zejm-2003.pdf} \\ 
  DFO & Atlantic cod NAFO 2J3KL inshore & \textit{Gadus morhua} & 1959-2006 & 2005 & 1.60 & 0.14 & no & \cite{DFO-COD2J3KLIS-2006.pdf} \\ 
  DFO & Atlantic cod NAFO 3Ps & \textit{Gadus morhua} & 1959-2004 & 2004 & 0.49 & 0.41 & no & \cite{DFO-COD3Ps-2004.pdf} \\ 
  DFO & English sole Hecate Strait & \textit{Parophrys vetulus} & 1944-2001 & 2001 & 1.23 & 0.37 & no & \cite{Flat99.pdf} \\ 
  DFO & Pacific herring Central Coast & \textit{Clupea pallasii} & 1951-2007 & 2007 & 0.30 & 0.11 & no & \cite{NA} \\ 
  DFO & Pacific herring Prince Rupert District & \textit{Clupea pallasii} & 1951-2007 & 2007 & 0.16 & 0.32 & no & \cite{NA} \\ 
  DFO & Pacific herring Queen Charlotte Islands & \textit{Clupea pallasii} & 1951-2007 & 2007 & 0.20 & 0.00 & no & \cite{NA} \\ 
  DFO & Pacific herring Straight of Georgia & \textit{Clupea pallasii} & 1951-2007 & 2007 & 0.91 & 0.40 & no & \cite{NA} \\ 
  DFO & Pacific herring West Coast of Vancouver Island & \textit{Clupea pallasii} & 1951-2007 & 2007 & 0.03 & 0.00 & no & \cite{NA} \\ 
  DFO & Pacific cod Hecate Strait & \textit{Gadus macrocephalus} & 1956-2005 & 2004 & 0.37 & 0.25 & no & \cite{NA} \\ 
  DFO & Pacific cod West Coast of Vancouver Island & \textit{Gadus macrocephalus} & 1956-2002 & 2001 & 0.28 & 0.61 & no & \cite{NA} \\ 
  DFO & Rock sole Hecate Strait & \textit{Lepidopsetta bilineata} & 1945-2001 & 2001 & 1.03 & 0.45 & no & \cite{Flat99.pdf} \\ 
  DFO & Pollock NAFO-4VWX5Zc & \textit{Pollachius virens} & 1974-2007 & 2006 & 0.56 & 0.30 & no & \cite{NAFO-POLL4VWX5Zc-2006.pdf} \\ 
  DFO & Atlantic cod NAFO 3Pn4RS & \textit{Gadus morhua} & 1964-2007 & 2006 & 0.09 & 0.79 & no & \cite{DFO-COD3Pn4Rs-2007.pdf} \\ 
  DFO & Atlantic cod NAFO 4TVn & \textit{Gadus morhua} & 1965-2007 & 2006 & 0.17 & 0.32 & no & \cite{NAFO-COD4TVn-2007.pdf} \\ 
  ICCAT & Albacore tuna North Atlantic & \textit{Thunnus alalunga} & 1929-2005 & 2005 & 0.81 & 1.49 & yes & \cite{2007-ALB-STOCK-ASSESS-REP.pdf} \\ 
  ICCAT & Bluefin tuna Eastern Atlantic & \textit{Thunnus thynnus} & 1969-2007 & 2007 & 0.34 & 9.38 & yes & \cite{2008-BFT-STOCK-ASSESS-REP.pdf} \\ 
  ICCAT & Bluefin tuna Western Atlantic & \textit{Thunnus thynnus} & 1969-2007 & 2007 & 0.57 & 1.33 & yes & \cite{2008-BFT-STOCK-ASSESS-REP.pdf} \\ 
  ICCAT & Bigeye tuna Atlantic & \textit{Thunnus obesus} & 1950-2005 & 2005 & 0.90 & 0.86 & no & \cite{JENSEN-BIGEYEATL-2008.pdf} \\ 
  ICCAT & Skipjack tuna Eastern Atlantic & \textit{Katsuwonus pelamis} & 1950-2006 & 2006 & 1.71 & 0.27 & no & \cite{JENSEN_YFINATL-2008.pdf} \\ 
  ICCAT & Skipjack tuna Western Atlantic & \textit{Katsuwonus pelamis} & 1952-2006 & 2006 & 1.72 & 0.27 & no & \cite{JENSEN_YFINATL-2008.pdf} \\ 
  ICCAT & Swordfish Mediterranean Sea & \textit{Xiphias gladius} & 1968-2006 & 2006 & 0.94 & 1.27 & yes & \cite{ICCAT-Mediterranean-Xiphiasgladius-2007.pdf} \\ 
  ICCAT & Swordfish North Atlantic & \textit{Xiphias gladius} & 1978-2007 & 2005 & 1.03 & 0.82 & no & \cite{JENSEN_SWORDSATL-2007.pdf} \\ 
  ICCAT & Swordfish South Atlantic & \textit{Xiphias gladius} & 1950-2005 & 2005 & 1.18 & 0.69 & no & \cite{JENSEN_SWORDSATL-2007.pdf} \\ 
  ICCAT & Yellowfin tuna Atlantic & \textit{Thunnus albacares} & 1970-2006 & 2006 & 1.07 & 0.81 & yes & \cite{JENSEN-YFINATL-2008.pdf} \\ 
  ICES & Capelin Barents Sea & \textit{Mallotus villosus} & 1965-2007 & 2006 & 0.17 & 0.00 & no & \cite{ICES-AFWG-2007.pdf} \\ 
  ICES & Atlantic cod coastal Norway & \textit{Gadus morhua} & 1982-2006 & 2006 & 0.27 & 2.17 & no & \cite{ICES-AFWG-2007.pdf} \\ 
  ICES & Atlantic cod Northeast Arctic & \textit{Gadus morhua} & 1943-2006 & 2006 & 0.56 & 1.42 & no & \cite{AFWG-NEAR-Gadusmorhua-2007.pdf} \\ 
  ICES & Greenland halibut Northeast Arctic & \textit{Reinhardtius hippoglossoides} & 1959-2007 & 2006 & 0.36 & 1.20 & no & \cite{ICES-AFWG-2007.pdf} \\ 
  ICES & Golden Redfish Northeast Arctic & \textit{Sebastes norvegicus} & 1986-2006 & 2006 & 0.29 & 2.65 & no & \cite{ICES-AFWG-2007.pdf} \\ 
  ICES & Haddock Northeast Arctic & \textit{Melanogrammus aeglefinus} & 1947-2006 & 2006 & 1.10 & 1.06 & no & \cite{ICES-AFWG-2007.pdf} \\ 
  ICES & Pollock Northeast Arctic & \textit{Pollachius virens} & 1957-2006 & 2006 & 1.70 & 0.60 & no & \cite{ICES-AFWG-2007.pdf} \\ 
  ICES & Herring ICES 22-24-IIIa & \textit{Clupea harengus} & 1991-2006 & 2006 & 0.73 & 1.02 & no & \cite{ICES-HAWG-2007.pdf} \\ 
  ICES & Herring Northern Irish Sea & \textit{Clupea harengus} & 1960-2006 & 2006 & 0.72 & 0.34 & no & \cite{ICES-HAWG-2007.pdf} \\ 
  ICES & Herring North Sea & \textit{Clupea harengus} & 1960-2007 & 2006 & 0.65 & 1.32 & no & \cite{ICES-HAWG-2007.pdf} \\ 
  ICES & Herring ICES VIa & \textit{Clupea harengus} & 1957-2006 & 2006 & 0.18 & 1.59 & no & \cite{ICES-HAWG-2007.pdf} \\ 
  ICES & Herring ICES VIa-VIIb-VIIc & \textit{Clupea harengus} & 1969-2000 & 2000 & 0.50 & 1.04 & no & \cite{ICES-HAWG-2007.pdf} \\ 
  ICES & Capelin Iceland & \textit{Mallotus villosus} & 1977-2007 & 2006 & 0.49 & 0.85 & no & \cite{ICES-NWWG-2007.pdf} \\ 
  ICES & Atlantic cod Faroe Plateau & \textit{Gadus morhua} & 1959-2006 & 2006 & 0.26 & 1.52 & no & \cite{ICES-NWWG-2007.pdf} \\ 
  ICES & Atlantic cod Iceland & \textit{Gadus morhua} & 1952-2006 & 2006 & 0.46 & 1.17 & no & \cite{ICES-NWWG-2007.pdf} \\ 
  ICES & Haddock Faroe Plateau & \textit{Melanogrammus aeglefinus} & 1955-2006 & 2006 & 0.85 & 1.07 & no & \cite{ICES-NWWG-2007.pdf} \\ 
  ICES & Haddock Iceland & \textit{Melanogrammus aeglefinus} & 1977-2007 & 2007 & 0.98 & 1.23 & no & \cite{ICES-NWWG-2007.pdf} \\ 
  ICES & Pollock Faroe Plateau & \textit{Pollachius virens} & 1958-2006 & 2006 & 0.99 & 1.52 & no & \cite{ICES-NWWG-2007.pdf} \\ 
  ICES & Atlantic cod Baltic Areas 22 and 24 & \textit{Gadus morhua} & 1969-2007 & 2006 & 0.36 & 1.43 & no & \cite{ICES-WGBFAS-2007.pdf} \\ 
  ICES & Atlantic cod Baltic Areas 25-32 & \textit{Gadus morhua} & 1964-2007 & 2006 & 0.16 & 1.46 & no & \cite{ICES-WGBFAS-2007.pdf} \\ 
  ICES & Atlantic cod Kattegat & \textit{Gadus morhua} & 1970-2006 & 2006 & 0.19 & 0.31 & no & \cite{ICES-WGBFAS-2007.pdf} \\ 
  ICES & Herring ICES 25-32 & \textit{Clupea harengus} & 1973-2006 & 2006 & 0.69 & 0.79 & no & \cite{ICES-WGBFAS-2007.pdf} \\ 
  ICES & Herring ICES 30 & \textit{Clupea harengus} & 1972-2007 & 2006 & 1.19 & 1.10 & no & \cite{ICES-WGBFAS-2007.pdf} \\ 
  ICES & Herring ICES 31 & \textit{Clupea harengus} & 1979-2006 & 2006 & 0.29 & 1.60 & no & \cite{ICES-WGBFAS-2007.pdf} \\ 
  ICES & Herring Iceland (Summer spawners) & \textit{Clupea harengus} & 1983-2007 & 2006 & 1.00 & 0.79 & no & \cite{ICES-NWWG-2007.pdf} \\ 
  ICES & Herring ICES 28 & \textit{Clupea harengus} & 1976-2007 & 2006 & 1.21 & 0.87 & no & \cite{ICES-WGBFAS-2007.pdf} \\ 
  ICES & common European sole ICES Kattegat and Skagerrak & \textit{Solea vulgaris} & 1982-2007 & 2006 & 1.25 & 0.54 & no & \cite{ICES-WGBFAS-2007.pdf} \\ 
  ICES & Sprat ICES Baltic Areas 22-32 & \textit{Sprattus sprattus} & 1973-2007 & 2006 & 1.13 & 1.27 & no & \cite{ICES-WGBFAS-2007.pdf} \\ 
  ICES & Fourspotted megrim ICES VIIIc-IXa & \textit{Lepidorhombus boscii} & 1986-2006 & 2006 & 0.70 & 1.01 & no & \cite{ICES-WGHMM-2007.pdf} \\ 
  ICES & Hake Northeast Atlantic North & \textit{Merluccius merluccius} & 1977-2007 & 2006 & 1.04 & 0.74 & no & \cite{ICES-WGHMM-2007.pdf} \\ 
  ICES & Megrim ICES VIIIc-IXa & \textit{Lepidorhombus whiffiagonis} & 1985-2007 & 2006 & 0.43 & 1.07 & no & \cite{ICES-WGHMM-2007.pdf} \\ 
  ICES & common European sole Bay of Biscay & \textit{Solea vulgaris} & 1982-2006 & 2006 & 0.76 & 1.00 & no & \cite{ICES-WGHMM-2007.pdf} \\ 
  ICES & Mackerel ICES Northeast Atlantic & \textit{Scomber scombrus} & 1972-2007 & 2006 & 0.98 & 0.73 & no & \cite{ICES-WGMHSA07.pdf} \\ 
  ICES & Whiting Northeast Atlantic & \textit{Micromesistius poutassou} & 1980-2007 & 2006 & 0.67 & 1.66 & no & \cite{ICES-WGNPBW-2007.pdf} \\ 
  ICES & Atlantic cod Irish Sea & \textit{Gadus morhua} & 1968-2006 & 2006 & 0.15 & 0.56 & no & \cite{ICES-WGNSDS-2007.pdf} \\ 
  ICES & Atlantic cod West of Scotland & \textit{Gadus morhua} & 1977-2006 & 2006 & 0.12 & 0.42 & no & \cite{ICES-WGNSDS-2007.pdf} \\ 
  ICES & Haddock West of Scotland & \textit{Melanogrammus aeglefinus} & 1977-2006 & 2006 & 0.58 & 0.73 & no & \cite{ICES-WGNSDS-2007.pdf} \\ 
  ICES & European Plaice Irish Sea & \textit{Pleuronectes platessa} & 1962-2006 & 2006 & 1.07 & 0.23 & no & \cite{ICES-WGNSDS-2007.pdf} \\ 
  ICES & common European sole Irish Sea & \textit{Solea vulgaris} & 1968-2006 & 2006 & 0.36 & 1.16 & no & \cite{ICES-WGNSDS-2007.pdf} \\ 
  ICES & Atlantic cod North Sea & \textit{Gadus morhua} & 1962-2007 & 2006 & 0.19 & 0.80 & no & \cite{ICES-WGNSSK-2007.pdf} \\ 
  ICES & Haddock ICES IIIa and North Sea & \textit{Melanogrammus aeglefinus} & 1963-2006 & 2006 & 0.62 & 0.25 & no & \cite{ICES-WGNSSK-2007.pdf} \\ 
  ICES & Haddock Rockall Bank & \textit{Melanogrammus aeglefinus} & 1990-2007 & 2006 & 1.10 & 0.52 & no & \cite{ICES-WGNSDS-2007.pdf} \\ 
  ICES & Norway pout North Sea & \textit{Trisopterus esmarkii} & 1983-2007 & 2006 & 0.90 & 0.33 & no & \cite{ICES-WGNSSK-2007.pdf} \\ 
  ICES & Pollock ICES IIIa, VI and North Sea & \textit{Pollachius virens} & 1964-2006 & 2006 & 0.57 & 0.97 & no & \cite{ICES-WGNSSK-2007.pdf} \\ 
  ICES & Sandeel North Sea & \textit{Ammodytes marinus} & 1983-2007 & 2007 & 0.92 & 0.24 & no & \cite{ICES-WGNSSK-2007.pdf} \\ 
  ICES & Whiting ICES IIIa, VIId and North Sea & \textit{Merlangius merlangus} & 1979-2006 & 2006 & 0.33 & 1.04 & no & \cite{ICES-WGNSSK-2007.pdf} \\ 
  ICES & Haddock ICES VIIb-k & \textit{Melanogrammus aeglefinus} & 1993-2006 & 2006 & 1.37 & 0.41 & no & \cite{ICES-WGSSDS-2007.pdf} \\ 
  ICES & European Plaice ICES VIIf-g & \textit{Pleuronectes platessa} & 1976-2006 & 2006 & 0.65 & 0.41 & no & \cite{ICES-WGSSDS-2007.pdf} \\ 
  ICES & European Plaice ICES VIIe & \textit{Pleuronectes platessa} & 1975-2006 & 2006 & 0.51 & 1.39 & no & \cite{ICES-WGSSDS-2007.pdf} \\ 
  ICES & common European sole Celtic Sea & \textit{Solea vulgaris} & 1970-2006 & 2006 & 0.90 & 0.95 & no & \cite{ICES-WGSSDS-2007.pdf} \\ 
  ICES & common European sole Western English Channel & \textit{Solea vulgaris} & 1968-2006 & 2006 & 0.51 & 1.74 & no & \cite{ICES-WGSSDS-2007.pdf} \\ 
  ICES & Whiting ICES VIIe-k & \textit{Merlangius merlangus} & 1982-2007 & 2006 & 0.44 & 1.25 & no & \cite{ICES-WGSSDS-2007.pdf} \\ 
  IOTC & Bigeye tuna Indian Ocean & \textit{Thunnus obesus} & 1957-2006 & 2004 & 1.23 & 0.97 & yes & \cite{JENSEN_BIGEYEIO-2007.pdf} \\ 
  IPHC & Pacific halibut North Pacific & \textit{Hippoglossus stenolepis} & 1988-2009 & 2008 & 0.54 & 2.01 & no & \cite{hare-clark08.pdf} \\ 
  MFish & Australian salmon New Zealand & \textit{Arripis trutta} & 1975-2006 & 2006 & 1.64 & 0.33 & yes & \cite{NA} \\ 
  MFish & Orange roughy New Zealand Mid East Coast & \textit{Hoplostethus atlanticus} & 1981-2004 & 2004 & 1.20 & 0.35 & yes & \cite{NA} \\ 
  MFish & Black oreo West end of Chatham Rise & \textit{Allocyttus niger} & 1973-2007 & 2007 & 0.99 & 0.82 & yes & \cite{NA} \\ 
  MFish & Smooth oreo Chatham Rise & \textit{Pseudocyttus maculatus} & 1979-2006 & 2006 & 2.25 & 0.38 & yes & \cite{NA} \\ 
  MFish & Smooth oreo West end of Chatham Rise & \textit{Pseudocyttus maculatus} & 1973-2004 & 2004 & 1.25 & 0.53 & yes & \cite{NA} \\ 
  MFish & Hoki Eastern New Zealand & \textit{Macruronus novaezelandiae} & 1972-2007 & 2007 & 1.11 & 0.33 & no & \cite{FAR0804hok07.pdf} \\ 
  MFish & Hoki Western New Zealand & \textit{Macruronus novaezelandiae} & 1972-2007 & 2007 & 0.51 & 0.57 & no & \cite{FAR0804hok07.pdf} \\ 
  MFish & New Zealand snapper New Zealand Area 8 & \textit{Chrysophrys auratus} & 1931-2005 & 2005 & 0.35 & 2.50 & yes & \cite{NA} \\ 
  MFish & Trevally New Zealand Areas TRE 7 & \textit{Pseudocaranx dentex} & 1944-2005 & 2005 & 1.44 & 0.83 & yes & \cite{NA} \\ 
  MFish & Red rock lobster New Zealand area CRA1 & \textit{Jasus edwardsii} & 1945-2001 & 2001 & 1.14 & 0.88 & no & \cite{NA} \\ 
  MFish & Red rock lobster New Zealand area CRA2 & \textit{Jasus edwardsii} & 1945-2001 & 2001 & 0.53 & 2.12 & no & \cite{NA} \\ 
  MFish & Red rock lobster New Zealand area CRA4 & \textit{Jasus edwardsii} & 1945-2005 & 2005 & 0.67 & 1.33 & no & \cite{NA} \\ 
  MFish & Red rock lobster New Zealand area CRA5 & \textit{Jasus edwardsii} & 1945-2002 & 2002 & 0.59 & 1.68 & no & \cite{NA} \\ 
  MFish & Red rock lobster New Zealand area CRA7 & \textit{Jasus edwardsii} & 1976-2005 & 2005 & 0.73 & 0.43 & no & \cite{NA} \\ 
  MFish & Red rock lobster New Zealand area CRA8 & \textit{Jasus edwardsii} & 1976-2005 & 2005 & 0.69 & 0.49 & no & \cite{NA} \\ 
  MFish & common gemfish New Zealand & \textit{Rexea solandri} & 1952-2007 & 2006 & 1.64 & 0.43 & yes & \cite{NA} \\ 
  MFish & New Zealand ling New Zealand Areas LIN 3 and 4 & \textit{Genypterus blacodes} & 1972-2007 & 2007 & 3.07 & 0.09 & yes & \cite{NA} \\ 
  MFish & New Zealand ling New Zealand Areas LIN 5 and 6 & \textit{Genypterus blacodes} & 1972-2007 & 2007 & 3.96 & 0.10 & yes & \cite{NA} \\ 
  MFish & New Zealand ling New Zealand Area LIN 6b & \textit{Genypterus blacodes} & 1980-2006 & 2006 & 2.19 & 0.11 & yes & \cite{NA} \\ 
  MFish & New Zealand ling New Zealand Area LIN 72 & \textit{Genypterus blacodes} & 1972-2007 & 2007 & 2.49 & 0.32 & yes & \cite{NA} \\ 
  MFish & New Zealand ling New Zealand Area LIN 7WC - WCSI & \textit{Genypterus blacodes} & 1972-2008 & 2008 & 2.21 & 0.13 & yes & \cite{NA} \\ 
  MFish & Southern blue whiting Campbell Island Rise & \textit{Micromesistius australis} & 1979-2006 & 2006 & 1.15 & 0.92 & yes & \cite{NA} \\ 
  MFish & Southern hake Chatham Rise & \textit{Merluccius australis} & 1975-2006 & 2006 & 1.77 & 0.12 & yes & \cite{NA} \\ 
  MFish & Southern hake Sub-Antarctic & \textit{Merluccius australis} & 1975-2007 & 2007 & 2.91 & 0.11 & yes & \cite{NA} \\ 
  MFish & New Zealand abalone species New Zealand Area PAU 5A & \textit{Haliotis iris} & 1964-2006 & 2006 & 0.72 & 2.83 & no & \cite{07-09-FAR.pdf} \\ 
  MFish & New Zealand abalone species New Zealand Area PAU 5B & \textit{Haliotis iris} & 1963-2007 & 2007 & 1.02 & 0.59 & no & \cite{08-05-FAR.pdf} \\ 
  MFish & New Zealand abalone species New Zealand Area PAU 5D & \textit{Haliotis iris} & 1964-2006 & 2006 & 0.44 & 2.10 & no & \cite{07-09-FAR.pdf} \\ 
  MFish & New Zealand abalone species New Zealand Area PAU 7 & \textit{Haliotis iris} & 1964-2008 & 2008 & 0.87 & 0.94 & no & \cite{NA} \\ 
  NAFO & American Plaice NAFO-3LNO & \textit{Hippoglossoides platessoides} & 1955-2007 & 2006 & 0.08 & 0.77 & no & \cite{NAFO-GrandBanks-AmPlaice-2007.pdf} \\ 
  NAFO & American Plaice NAFO-3M & \textit{Hippoglossoides platessoides} & 1960-2007 & 2007 & 0.13 & 0.00 & no & \cite{NAFO-AMPL3M-2008.pdf} \\ 
  NAFO & Atlantic cod NAFO 3NO & \textit{Gadus morhua} & 1953-2007 & 2006 & 0.02 & 0.27 & no & \cite{NAFO-3NO-COD-2007.pdf} \\ 
  NAFO & Greenland halibut NAFO 23KLMNO & \textit{Reinhardtius hippoglossoides} & 1960-2006 & 2006 & 0.39 & 1.73 & no & \cite{NAFO-GHAL23KLMNO-2007.pdf} \\ 
  NAFO & Redfish species NAFO 3LN & \textit{Redfish species} & 1959-2007 & 2006 & 1.95 & 0.01 & no & \cite{NAFO-RED3LN-2007.pdf} \\ 
  NAFO & Redfish species NAFO 3LN & \textit{Redfish species} & 1959-2008 & 2008 & 1.88 & 0.04 & yes & \cite{NAFO-3LN-Redfishspp-2008.pdf} \\ 
  NAFO & Redfish species NAFO 3M & \textit{Redfish species} & 1989-2006 & 2006 & 0.93 & 0.15 & no & \cite{NAFO-RED3M-2007.pdf} \\ 
  NAFO & Yellowtail Flounder NAFO 3LNO & \textit{Limanda ferruginea} & 1960-2009 & 2007 & 1.62 & 0.15 & no & \cite{NAFO-YELL3LNO-2008.pdf} \\ 
  NMFS & Alaska plaice Bering Sea and Aleutian Islands & \textit{Pleuronectes quadrituberculatus} & 1972-2008 & 2008 & 2.46 & 0.07 & yes & \cite{AFSC-ALPLAICBSAI-2008-Alaska plaice BSAI.pdf} \\ 
  NMFS & Arrowtooth flounder Bering Sea and Aleutian Islands & \textit{Reinhardtius stomias} & 1970-2008 & 2008 & 1.28 & 0.31 & no & \cite{AFSC-ARFLOUNDBSAI-2007-Arrowtooth flounder BSAI.pdf} \\ 
  NMFS & Arrowtooth flounder Gulf of Alaska & \textit{Reinhardtius stomias} & 1958-2010 & 2007 & 1.33 & 0.28 & no & \cite{2008_SAFE_GOAatf.pdf} \\ 
  NMFS & Atka mackerel Bering Sea and Aleutian Islands & \textit{Pleurogrammus monopterygius} & 1976-2009 & 2008 & 1.50 & 0.55 & no & \cite{2008_SAFE_BSAIatka.pdf} \\ 
  NMFS & Cabezon Northern California & \textit{Scorpaenichthys marmoratus} & 1916-2005 & 2004 & 0.89 & 0.99 & no & \cite{2005-SAFE-WCcabezon.pdf} \\ 
  NMFS & Cabezon Southern California & \textit{Scorpaenichthys marmoratus} & 1932-2005 & 2004 & 0.81 & 0.53 & no & \cite{2005_SAFE_Wccabezon.pdf} \\ 
  NMFS & Dusky rockfish Gulf of Alaska & \textit{Sebastes variabilis} & 1973-2008 & 2007 & 1.54 & 0.54 & yes & \cite{AFSC-DUSROCKGA-2008-Dusky rockfish GA.pdf} \\ 
  NMFS & Flathead sole Bering Sea and Aleutian Islands & \textit{Hippoglossoides elassodon} & 1977-2008 & 2008 & 1.66 & 0.18 & no & \cite{2008_SAFE_BSAIflathead.pdf} \\ 
  NMFS & Greenland turbot Bering Sea and Aleutian Islands & \textit{Reinhardtius hippoglossoides} & 1960-2009 & 2009 & 1.48 & 0.05 & yes & \cite{2008_SAFE_BSAIturbot.pdf} \\ 
  NMFS & Northern rockfish Bering Sea and Aleutian Islands & \textit{Sebastes polyspinis} & 1974-2009 & 2008 & 1.07 & 0.13 & no & \cite{2008_SAFE_BSAInorthern.pdf} \\ 
  NMFS & Northern rockfish Gulf of Alaska & \textit{Sebastes polyspinis} & 1959-2008 & 2008 & 1.50 & 0.66 & yes & \cite{AFSC-NROCKGA-2008-Northern rockfish GA.pdf} \\ 
  NMFS & Northern rock sole Eastern Bering Sea and Aleutian Islands & \textit{Lepidopsetta polyxystra} & 1971-2008 & 2007 & 3.02 & 0.21 & yes & \cite{2008_SAFE_BSAIrocksole.pdf} \\ 
  NMFS & Pacific cod Bering Sea and Aleutian Islands & \textit{Gadus macrocephalus} & 1964-2008 & 2007 & 0.89 & 0.93 & no & \cite{AFSC-PCODBSAI-2008-Pacific cod BSAI.pdf} \\ 
  NMFS & Pacific cod Gulf of Alaska & \textit{Gadus macrocephalus} & 1964-2008 & 2007 & 1.07 & 0.84 & no & \cite{AFSC-PCODGA-2008-Pacific cod GA.pdf} \\ 
  NMFS & Pacific Ocean perch Eastern Bering Sea and Aleutian Islands & \textit{Sebastes alutus} & 1974-2009 & 2008 & 1.70 & 0.26 & no & \cite{2008_SAFE_BSAIpop.pdf} \\ 
  NMFS & Pacific ocean perch Gulf of Alaska & \textit{Sebastes alutus} & 1959-2008 & 2008 & 1.16 & 0.73 & yes & \cite{AFSC-POPERCHGA-2008-Pacific ocean perch GA.pdf} \\ 
  NMFS & Red king crab Bristol Bay & \textit{Paralithodes camtschaticus} & 1960-2008 & 2008 & 1.27 & 1.05 & yes & \cite{CRABSAFE2008.pdf} \\ 
  NMFS & Sablefish Eastern Bering Sea / Aleutian Islands / Gulf of Alaska & \textit{Anoplopoma fimbria} & 1956-2008 & 2008 & 1.05 & 0.66 & yes & \cite{AFSC-SABLEFEBSAIGA-2008-Sablefish EBS AI GA.pdf} \\ 
  NMFS & Snow crab Bering Sea & \textit{Chionoecetes opilio} & 1979-2008 & 2008 & 0.35 & 1.49 & no & \cite{CRABSAFE2008.pdf} \\ 
  NMFS & Walleye pollock Aleutian Islands & \textit{Theragra chalcogramma} & 1976-2008 & 2008 & 0.86 & 0.02 & yes & \cite{AFSC-WPOLLAI-2008-Walleye pollock AI.pdf} \\ 
  NMFS & Walleye pollock Eastern Bering Sea & \textit{Theragra chalcogramma} & 1963-2008 & 2008 & 0.68 & 0.85 & no & \cite{AFSC-WPOLLEBS-2008-Walleye pollock EBS.pdf} \\ 
  NMFS & Yellowfin sole Bering Sea and Aleutian Islands & \textit{Limanda aspera} & 1959-2008 & 2008 & 1.94 & 0.62 & yes & \cite{AFSC-YSOLEBSAI-2008-Yellowfin sole BSAI.pdf} \\ 
  NMFS & Atlantic croaker Mid-Atlantic Coast & \textit{Micropogonias undulatus} & 1973-2002 & 2002 & 1.42 & 0.27 & yes & \cite{2004_ASMFC_AtlCroak.pdf} \\ 
  NMFS & Northern shrimp Gulf of Maine & \textit{Pandalus borealis} & 1960-2009 & 2008 & 1.58 & 0.56 & no & \cite{2008ShrimpAssessment.pdf} \\ 
  NMFS & American Plaice NAFO-5YZ & \textit{Hippoglossoides platessoides} & 1960-2008 & 2007 & 0.55 & 0.30 & no & \cite{ .pdf} \\ 
  NMFS & Bluefish Atlantic Coast & \textit{Pomatomus saltatrix} & 1981-2007 & 2007 & 0.81 & 1.25 & no & \cite{final-2005-SAW-41-assessment.pdf} \\ 
  NMFS & Black sea bass Mid-Atlantic Coast & \textit{Centropristis striata} & 1968-2007 & 2007 & 1.21 & 0.67 & no & \cite{DataPoorReviewPanelReportFinal-1-20-09.pdf} \\ 
  NMFS & Atlantic cod Georges Bank & \textit{Gadus morhua} & 1960-2008 & 2007 & 0.18 & 0.72 & no & \cite{NMFS-GB-Gadusmorhua-2008.pdf} \\ 
  NMFS & Atlantic cod Gulf of Maine & \textit{Gadus morhua} & 1893-2008 & 2007 & 1.46 & 0.29 & no & \cite{NMFS-GOM-Gadusmorhua-2008.pdf} \\ 
  NMFS & Haddock NAFO-5Y & \textit{Melanogrammus aeglefinus} & 1956-2008 & 2007 & 0.36 & 1.21 & no & \cite{NMFS-GOM-Melanogrammusaeglefinus-2008.pdf} \\ 
  NMFS & Monkfish Gulf of Maine / Northern Georges Bank & \textit{Lophius americanus} & 1964-2006 & 2006 & 1.73 & 0.38 & no & \cite{crd0721.pdf} \\ 
  NMFS & Monkfish Southern Georges Bank / Mid-Atlantic & \textit{Lophius americanus} & 1964-2006 & 2006 & 1.72 & 0.30 & no & \cite{Monkfish2007NEFSCAssessment.pdf} \\ 
  NMFS & Sea scallop Georges Bank & \textit{Placopecten magellanicus} & 1964-2006 & 2006 & 1.59 & 0.78 & no & \cite{SeaScallop2007.pdf} \\ 
  NMFS & Sea scallop Mid-Atlantic Coast & \textit{Placopecten magellanicus} & 1964-2006 & 2006 & 1.00 & 0.36 & no & \cite{SeaScallop2007.pdf} \\ 
  NMFS & Spiny dogfish Atlantic Coast & \textit{Squalus acanthias} & 1962-2006 & 2005 & 1.61 & 0.15 & no & \cite{spinydogfish2006.pdf} \\ 
  NMFS & Atlantic surfclam Mid-Atlantic Coast & \textit{Spisula solidissima} & 1965-2008 & 1994 & 1.85 & 0.00 & no & \cite{Surfclam2007.pdf} \\ 
  NMFS & Tilefish Mid-Atlantic Coast & \textit{Lopholatilus chamaeleonticeps} & 1973-2008 & 2005 & 0.72 & 0.61 & no & \cite{Tilefish2005.pdf} \\ 
  NMFS & Weakfish Atlantic Coast & \textit{Cynoscion regalis} & 1980-2008 & 2008 & 0.06 & 1.49 & no & \cite{NEFSC-Weakfish-2009.pdf} \\ 
  NMFS & White hake Georges Bank / Gulf of Maine & \textit{Urophycis tenuis} & 1963-2007 & 2007 & 0.35 & 0.80 & yes & \cite{WhiteHake2008.pdf} \\ 
  NMFS & Winter Flounder NAFO-5Z & \textit{Pseudopleuronectes americanus} & 1982-2007 & 2006 & 1.41 & 0.25 & no & \cite{garm3k.pdf} \\ 
  NMFS & Winter Flounder Southern New England-Mid Atlantic & \textit{Pseudopleuronectes americanus} & 1940-2007 & 2007 & 0.17 & 1.10 & no & \cite{NMFS-SNEMATL-Pseudopleuronectesamercianus-2008.pdf} \\ 
  NMFS & Witch Flounder NAFO-5Y & \textit{Glyptocephalus cynoglossus} & 1982-2008 & 2007 & 0.30 & 1.45 & yes & \cite{NA} \\ 
  NMFS & Yellowtail flounder Cape Cod / Gulf of Maine & \textit{Limanda ferruginea} & 1935-2008 & 2007 & 0.25 & 1.73 & yes & \cite{NMFS-CCGOM-Limandaferruginea-2008.pdf} \\ 
  NMFS & Yellowtail flounder Georges Bank & \textit{Limanda ferruginea} & 1935-2008 & 2007 & 0.22 & 1.14 & yes & \cite{NMFS-GB-Limandaferruginea-2008.pdf} \\ 
  NMFS & Yellowtail Flounder Southern New England-Mid Atlantic & \textit{Limanda ferruginea} & 1935-2008 & 2007 & 0.13 & 1.61 & yes & \cite{NMFS-SNEMATL-Limandaferruginea-2008.pdf} \\ 
  NMFS & Atlantic menhaden Atlantic & \textit{Brevoortia tyrannus} & 1940-2005 & 2005 & 0.47 & 0.97 & no & \cite{Atl.Menhaden-ASMFC-2006.pdf} \\ 
  NMFS & Pacific sardine North Pacific & \textit{Sardinops sagax} & 1981-2008 & 2006 & 1.73 & 0.37 & no & \cite{2008 pac sardine.pdf} \\ 
  NMFS & Arrowtooth flounder Pacific Coast & \textit{Reinhardtius stomias} & 1916-2007 & 2007 & 3.81 & 0.21 & yes & \cite{NWFSC-ARFLOUNDPCOAST-2007-Arrowtooth flounder.pdf} \\ 
  NMFS & Blackgill rockfish  Pacific Coast & \textit{Sebastes melanostomus} & 1950-2005 & 2004 & 0.80 & 1.20 & no & \cite{2005-SAFE-Wcblackgill.pdf} \\ 
  NMFS & Black rockfish Northern Pacific Coast & \textit{Sebastes melanops} & 1914-2006 & 2006 & 1.37 & 0.57 & no & \cite{NWFSC-BLACKROCKNPCOAST-2007-Black rockfish NOR WA.pdf} \\ 
  NMFS & Black rockfish Southern Pacific Coast & \textit{Sebastes melanops} & 1915-2007 & 2007 & 2.23 & 0.33 & yes & \cite{NWFSC-BLACKROCKSPCOAST-2007-Black rockfish OR CA.pdf} \\ 
  NMFS & Blue rockfish California & \textit{Sebastes mystinus} & 1916-2007 & 2007 & 0.75 & 1.19 & yes & \cite{NWFSC-BLUEROCKCAL-2007-Blue rockfish CA.pdf} \\ 
  NMFS & Bocaccio Southern Pacific Coast & \textit{Sebastes paucispinis} & 1951-2006 & 2006 & 0.32 & 0.10 & yes & \cite{NWFSC-BOCACCSPCOAST-2007 Bocaccio.pdf} \\ 
  NMFS & Chilipepper Southern Pacific Coast & \textit{Sebastes goodei} & 1892-2007 & 2006 & 1.43 & 0.04 & no & \cite{NWFSC-CHILISPCOAST-2007-Chilipepper CA OR.pdf} \\ 
  NMFS & Cowcod Southern California & \textit{Sebastes levis} & 1900-2007 & 2007 & 0.09 & 0.07 & yes & \cite{NWFSC-COWCODSCAL-2007-Cowcod CA.pdf} \\ 
  NMFS & Canary rockfish Pacific Coast & \textit{Sebastes pinniger} & 1916-2007 & 2007 & 0.85 & 0.02 & yes & \cite{NWFSC-CROCKPCOAST-2007-Canary.pdf} \\ 
  NMFS & Darkblotched rockfish Pacific Coast & \textit{Sebastes crameri} & 1928-2007 & 2007 & 0.73 & 0.31 & yes & \cite{NWFSC-DKROCKPCOAST-2008-Darkblotched rockfish.pdf} \\ 
  NMFS & English sole Pacific Coast & \textit{Parophrys vetulus} & 1876-2007 & 2006 & 2.06 & 0.14 & no & \cite{NWFSC-ESOLEPCOAST-2007-EnglishSole.pdf} \\ 
  NMFS & Longnose skate Pacific Coast & \textit{Raja rhina} & 1915-2007 & 2007 & 1.56 & 0.40 & no & \cite{NWFSC-LNOSESKAPCOAST-2008-Longnose skate.pdf} \\ 
  NMFS & Longspine thornyhead Pacific Coast & \textit{Sebastolobus altivelis} & 1962-2005 & 2005 & 2.65 & 0.23 & yes & \cite{2005-SAFE-Longspine.pdf} \\ 
  NMFS & Pacific hake Pacific Coast & \textit{Merluccius productus} & 1966-2008 & 2008 & 1.61 & 0.73 & yes & \cite{NWFSC-PHAKEPCOAST-2008-Pacific-Hake-US-Canada.pdf} \\ 
  NMFS & Pacific ocean perch Pacific Coast & \textit{Sebastes alutus} & 1953-2007 & 2007 & 0.69 & 0.00 & yes & \cite{NWFSC-POPERCHPCOAST-2007-Pacific ocean perch.pdf} \\ 
  NMFS & Petrale sole Northern Pacific Coast & \textit{Eopsetta jordani} & 1910-2005 & 2004 & 0.96 & 1.26 & no & \cite{2004_SAFE_WCpetralesole.pdf} \\ 
  NMFS & Petrale sole Southern Pacific Coast & \textit{Eopsetta jordani} & 1874-2005 & 2004 & 0.63 & 0.61 & no & \cite{2004-SAFE-WCpetralesole.pdf} \\ 
  NMFS & Sablefish Pacific Coast & \textit{Anoplopoma fimbria} & 1900-2007 & 2007 & 0.84 & 1.09 & no & \cite{NWFSC-SABLEFPCOAST-2007-Sablefish.pdf} \\ 
  NMFS & Shortspine thornyhead Pacific Coast & \textit{Sebastolobus alascanus} & 1901-2005 & 2004 & 1.55 & 0.93 & no & \cite{2005-SST-assessment.pdf} \\ 
  NMFS & Widow rockfish Pacific Coast & \textit{Sebastes entomelas} & 1955-2006 & 2006 & 0.91 & 0.07 & no & \cite{NWFSC-WROCKPCOAST-2007-widow.pdf} \\ 
  NMFS & Yelloweye rockfish Pacific Coast & \textit{Sebastes ruberrimus} & 1923-2006 & 2006 & 0.38 & 0.34 & no & \cite{NWFSC-YEYEROCKPCOAST-2007-yelloweye.pdf} \\ 
  NMFS & Yellowtail rockfish Northern Pacific Coast & \textit{Sebastes flavidus} & 1967-2005 & 2005 & 0.75 & 0.51 & no & \cite{2005_SAFE_yellowtail.pdf} \\ 
  NMFS & Gag Gulf of Mexico & \textit{Mycteroperca microlepis} & 1963-2004 & 2004 & 1.00 & 2.44 & yes & \cite{JENSEN_GAGGM_2007.pdf} \\ 
  NMFS & Gag Southern Atlantic coast & \textit{Mycteroperca microlepis} & 1962-2005 & 2005 & 0.94 & 1.31 & yes & \cite{JENSEN_GAGSATLC_2006.pdf} \\ 
  NMFS & Greater amberjack Gulf of Mexico & \textit{Seriola dumerili} & 1986-2004 & 2004 & 0.46 & 1.52 & no & \cite{JENSEN_GRAMBERGM_2006.pdf} \\ 
  NMFS & King mackerel Gulf of Mexico & \textit{Scomberomorus cavalla} & 1992-2001 & 2001 & 1.51 & 0.44 & no & \cite{JENSEN_KMACKGMSATLC_2004.pdf} \\ 
  NMFS & King mackerel Southern Atlantic Coast & \textit{Scomberomorus cavalla} & 1981-2001 & 2001 & 1.38 & 0.56 & no & \cite{JENSEN_KMACKGMSATLC_2004.pdf} \\ 
  NMFS & Gulf menhaden Gulf of Mexico & \textit{Brevoortia patronus} & 1964-2004 & 2004 & 1.08 & 0.48 & no & \cite{GILROY-MENHADENGM-2007.pdf} \\ 
  NMFS & Red grouper Gulf of Mexico & \textit{Epinephelus morio} & 1986-2005 & 2005 & 1.00 & 0.88 & no & \cite{JENSEN_RGROUPGM-2006.pdf} \\ 
  NMFS & Red porgy Southern Atlantic coast & \textit{Pagrus pagrus} & 1972-2004 & 2004 & 0.61 & 0.39 & yes & \cite{JENSEN_RPORGYSATLC_2006.pdf} \\ 
  NMFS & Snowy grouper Southern Atlantic coast & \textit{Epinephelus niveatus} & 1961-2002 & 2002 & 0.19 & 3.08 & yes & \cite{2004-SEDAR-deepwatersnappergrouper.pdf} \\ 
  NMFS & Spanish mackerel Southern Atlantic Coast & \textit{Scomberomorus maculatus} & 1950-2008 & 2007 & 0.38 & 0.91 & yes & \cite{JENSEN_SPANMACKSATLC_2008.pdf} \\ 
  NMFS & Tilefish Southern Atlantic coast & \textit{Lopholatilus chamaeleonticeps} & 1961-2002 & 2002 & 0.94 & 1.55 & yes & \cite{2004-SEDAR-deepwatersnappergrouper.pdf} \\ 
  NMFS & Vermilion snapper Southern Atlantic coast & \textit{Rhomboplites aurorubens} & 1946-2008 & 2007 & 0.86 & 1.27 & yes & \cite{2008_SEDAR_VermillionSnapper_Satl.pdf} \\ 
  NMFS & Yellowtail snapper Southern Atlantic Coast and Gulf of Mexico & \textit{Ocyurus chrysurus} & 1962-2001 & 2001 & 1.14 & 0.61 & yes & \cite{2003_SEDAR_Yellowtailsnapper.pdf} \\ 
  NMFS & Dover sole Pacific Coast & \textit{Microstomus pacificus} & 1910-2005 & 2004 & 1.33 & 0.45 & no & \cite{2005-SAFE-WCdover.pdf} \\ 
  NMFS & Gopher rockfish Southern Pacific Coast & \textit{Sebastes carnatus} & 1965-2005 & 2004 & 0.88 & 0.62 & no & \cite{2005-SAFE-Wcgopher.pdf} \\ 
  NMFS & Pacific sardine Pacific Coast & \textit{Sardinops sagax} & 1981-2007 & 2006 & 1.36 & 0.41 & no & \cite{NOAA-TM-NMFS-SWFSC-413.pdf} \\ 
  NMFS & Starry flounder Northern Pacific Coast & \textit{Platichthys stellatus} & 1970-2005 & 2004 & 0.68 & 0.33 & no & \cite{2005-SAFE-WCstarryflounder.pdf} \\ 
  NMFS & Starry flounder Southern Pacific Coast & \textit{Platichthys stellatus} & 1970-2005 & 2004 & 1.15 & 0.12 & no & \cite{2005-SAFE-WCstarryflounder.pdf} \\ 
  RFFA & Walleye pollock Northern Sea of Okhotsk & \textit{Theragra chalcogramma} & 1985-1994 & 1992 & 1.11 & 0.87 & no & \cite{WPOLLNSO-1997-JENSEN.pdf} \\ 
  RFFA & Walleye pollock Western Bering Sea & \textit{Theragra chalcogramma} & 1994-2004 & 2004 & 2.16 & 0.26 & no & \cite{WPOLLWBS-2004-JENSEN.pdf} \\ 
  SPRFMO & Chilean jack mackerel Chilean EEZ and offshore & \textit{Trachurus murphyi} & 1975-2007 & 2006 & 0.52 & 1.20 & no & \cite{JENSEN-JACKMACKCH-2008.pdf} \\ 
  US State & American lobster Rhode Island & \textit{Homarus americanus} & 1959-2007 & 2006 & 0.53 & 0.67 & no & \cite{NA} \\ 
  US State & Tautog Rhode Island & \textit{Tautoga onitis} & 1959-2007 & 2006 & 0.84 & 0.59 & no & \cite{NA} \\ 
  US State & Winter flounder Rhode Island & \textit{Pseudopleuronectes americanus} & 1959-2007 & 2006 & 0.25 & 1.59 & no & \cite{NA} \\ 
  WCPFC & Albacore tuna South Pacific Ocean & \textit{Thunnus alalunga} & 1959-2006 & 2006 & 2.46 & 0.90 & yes & \cite{JENSEN_ALBWPO_2008.pdf} \\ 
  WCPFC & Bigeye tuna Western Pacific Ocean & \textit{Thunnus obesus} & 1952-2006 & 2006 & 1.06 & 1.38 & yes & \cite{SC4-SA-WP1-rev1-bigeye-tuna.pdf} \\ 
  WCPFC & Skipjack tuna Central Western Pacific & \textit{Katsuwonus pelamis} & 1972-2006 & 2006 & 4.38 & 0.30 & yes & \cite{SC4-SA-WP4-SKJ-Assessment-rev1-skipjack.pdf} \\ 
  WCPFC & Yellowfin tuna Central Western Pacific & \textit{Thunnus albacares} & 1952-2005 & 2005 & 1.22 & 0.80 & yes & \cite{WCPFC-SC3-SA-SWG-WP-01.pdf} \\ 
   \hline
\hline
\caption{Summary of the assessments used in this analysis and their estimated ratios of current biomass to the biomass at maximum sustainable yield and current harvest rate to the harvest rate that results is maximum sustainable yield. The estimated ratios were preferentially obtained directly from the assessment document or derived from surplus production model fits. When both an SSBmsy and Bmsy reference points are available, the SSB is chosen preferentially.}
\label{tab:crosshair}
\end{longtable}

%\begin{landscape}
%% latex table generated in R 2.9.1 by xtable 1.5-6 package
% Thu Jun 10 10:21:47 2010
\begin{longtable}{p{1.8cm}p{4cm}p{4cm}ccccp{1.9cm}c}
  \hline
mgmt & stock & scientificname & timespan & currentyear & Bratio & Uratio & fromassessment & ref \\ 
  \hline
AFMA & Deepwater flathead Southeast Australia & \textit{Platycephalus conatus} & 1978-2007 & 2006 & 1.33 & 0.61 & no & \cite{NA} \\ 
  AFMA & common gemfish Southeast Australia & \textit{Rexea solandri} & 1966-2007 & 2007 & 0.10 & 0.39 & no & \cite{NA} \\ 
  AFMA & Jackass morwong Southeast Australia & \textit{Nemadactylus macropterus} & 1913-2007 & 2007 & 0.25 & 1.80 & no & \cite{NA} \\ 
  AFMA & New Zealand ling Eastern half of Southeast Australia & \textit{Genypterus blacodes} & 1968-2007 & 2007 & 0.60 & 2.20 & no & \cite{NA} \\ 
  AFMA & Orange roughy Cascade Plateau & \textit{Hoplostethus atlanticus} & 1987-2006 & 2006 & 1.76 & 0.34 & no & \cite{CSIRO-Cascade-Plateau-Stock-Assessment-2006.pdf} \\ 
  AFMA & Orange roughy Southeast Australia & \textit{Hoplostethus atlanticus} & 1978-2007 & 2006 & 0.89 & 0.29 & no & \cite{NA} \\ 
  AFMA & Silverfish Southeast Australia & \textit{Seriolella punctata} & 1978-2006 & 2006 & 1.15 & 0.79 & no & \cite{NA} \\ 
  AFMA & School whiting Southeast Australia & \textit{Sillago flindersi} & 1945-2007 & 2007 & 1.10 & 0.82 & no & \cite{NA} \\ 
  AFMA & Tiger flathead Southeast Australia & \textit{Neoplatycephalus richardsoni} & 1913-2006 & 2006 & 1.05 & 0.00 & no & \cite{NA} \\ 
  AFMA & Blue Warehou Eastern half of Southeast Australia & \textit{Seriolella brama} & 1984-2006 & 2006 & 0.17 & 0.84 & no & \cite{NA} \\ 
  AFMA & Blue Warehou Western half of Southeast Australia & \textit{Seriolella brama} & 1984-2006 & 2006 & 0.62 & 2.04 & no & \cite{NA} \\ 
  AFMA & Tasmanian giant crab Tasmania & \textit{Pseudocarcinus gigas} & 1990-2007 & 2007 & 0.50 & 1.71 & no & \cite{JENSEN_TASGIANTCRAB_2008.pdf} \\ 
  CCAMLR & Antarctic toothfish Ross Sea & \textit{Dissostichus mawsoni} & 1995-2007 & 2007 & 1.76 & 1.09 & no & \cite{NA} \\ 
  CFP & Argentine anchoita Northern Argentina & \textit{Engraulis anchoita} & 1989-2007 & 2007 & 1.37 & 0.17 & yes & \cite{Hansen-ANCHOVY-N-2007.pdf} \\ 
  CFP & Argentine anchoita Southern Argentina & \textit{Engraulis anchoita} & 1992-2007 & 2007 & 3.13 & 0.04 & yes & \cite{Hansen-ANCHOVY-S-2007.pdf} \\ 
  CFP & Argentine hake Northern Argentina & \textit{Merluccius hubbsi} & 1985-2007 & 2007 & 0.16 & 1.26 & yes & \cite{Irusta-hake-N-2007.pdf} \\ 
  CFP & Argentine hake Southern Argentina & \textit{Merluccius hubbsi} & 1985-2008 & 2008 & 0.34 & 1.49 & yes & \cite{Renzi-hake-S-2009.pdf} \\ 
  CFP & Patagonian grenadier Southern Argentina & \textit{Macruronus magellanicus} & 1983-2006 & 2006 & 1.82 & 0.60 & yes & \cite{Giussi-hoki-2007.pdf} \\ 
  DETMCM & Anchovy South Africa & \textit{Engraulis encrasicolus} & 1984-2006 & 2006 & 0.97 & 0.36 & no & \cite{.pdf} \\ 
  DETMCM & Shallow-water cape hake South Africa & \textit{Merluccius capensis} & 1917-2008 & 2008 & 1.16 & 0.40 & no & \cite{SA-Mcapensis-2008_IWS_DEC08_H_5.pdf} \\ 
  DETMCM & Cape horse mackerel South Africa South coast & \textit{Trachurus capensis} & 1950-2007 & 2007 & 1.47 & 0.76 & no & \cite{Johnston-SAHorseMackerel-2007.pdf} \\ 
  DETMCM & Kingklip South Africa & \textit{Engraulis encrasicolus} & 1932-2008 & 2008 & 1.13 & 0.55 & no & \cite{Branch-SA-Kingklip-2008.pdf} \\ 
  DETMCM & Sardine South Africa & \textit{Sardinops sagax} & 1984-2006 & 2006 & 0.75 & 0.55 & no & \cite{deMoorSASardineAssessment-Sep07.pdf} \\ 
  DETMCM & Southern spiny lobster South Africa South coast & \textit{Palinurus gilchristi} & 1973-2008 & 2008 & 0.51 & 1.50 & no & \cite{Johnston-SASouthRockLobster-2008.pdf} \\ 
  DFO & Atlantic cod NAFO 5Zjm & \textit{Gadus morhua} & 1978-2003 & 2002 & 0.34 & 0.45 & no & \cite{NAFO-COD5Zjm-2003.pdf} \\ 
  DFO & Haddock NAFO-4X5Y & \textit{Melanogrammus aeglefinus} & 1960-2003 & 2003 & 0.85 & 0.33 & no & \cite{NAFO-HAD4X5Y-2003.pdf} \\ 
  DFO & Haddock NAFO-5Zejm & \textit{Melanogrammus aeglefinus} & 1969-2003 & 2002 & 1.00 & 0.65 & no & \cite{NAFO-HAD5Zejm-2003.pdf} \\ 
  DFO & Atlantic cod NAFO 2J3KL inshore & \textit{Gadus morhua} & 1959-2006 & 2005 & 1.60 & 0.14 & no & \cite{DFO-COD2J3KLIS-2006.pdf} \\ 
  DFO & Atlantic cod NAFO 3Ps & \textit{Gadus morhua} & 1959-2004 & 2004 & 0.49 & 0.41 & no & \cite{DFO-COD3Ps-2004.pdf} \\ 
  DFO & English sole Hecate Strait & \textit{Parophrys vetulus} & 1944-2001 & 2001 & 1.23 & 0.37 & no & \cite{Flat99.pdf} \\ 
  DFO & Pacific herring Central Coast & \textit{Clupea pallasii} & 1951-2007 & 2007 & 0.30 & 0.11 & no & \cite{NA} \\ 
  DFO & Pacific herring Prince Rupert District & \textit{Clupea pallasii} & 1951-2007 & 2007 & 0.16 & 0.32 & no & \cite{NA} \\ 
  DFO & Pacific herring Queen Charlotte Islands & \textit{Clupea pallasii} & 1951-2007 & 2007 & 0.20 & 0.00 & no & \cite{NA} \\ 
  DFO & Pacific herring Straight of Georgia & \textit{Clupea pallasii} & 1951-2007 & 2007 & 0.91 & 0.40 & no & \cite{NA} \\ 
  DFO & Pacific herring West Coast of Vancouver Island & \textit{Clupea pallasii} & 1951-2007 & 2007 & 0.03 & 0.00 & no & \cite{NA} \\ 
  DFO & Pacific cod Hecate Strait & \textit{Gadus macrocephalus} & 1956-2005 & 2004 & 0.37 & 0.25 & no & \cite{NA} \\ 
  DFO & Pacific cod West Coast of Vancouver Island & \textit{Gadus macrocephalus} & 1956-2002 & 2001 & 0.28 & 0.61 & no & \cite{NA} \\ 
  DFO & Rock sole Hecate Strait & \textit{Lepidopsetta bilineata} & 1945-2001 & 2001 & 1.03 & 0.45 & no & \cite{Flat99.pdf} \\ 
  DFO & Pollock NAFO-4VWX5Zc & \textit{Pollachius virens} & 1974-2007 & 2006 & 0.56 & 0.30 & no & \cite{NAFO-POLL4VWX5Zc-2006.pdf} \\ 
  DFO & Atlantic cod NAFO 3Pn4RS & \textit{Gadus morhua} & 1964-2007 & 2006 & 0.09 & 0.79 & no & \cite{DFO-COD3Pn4Rs-2007.pdf} \\ 
  DFO & Atlantic cod NAFO 4TVn & \textit{Gadus morhua} & 1965-2007 & 2006 & 0.17 & 0.32 & no & \cite{NAFO-COD4TVn-2007.pdf} \\ 
  ICCAT & Albacore tuna North Atlantic & \textit{Thunnus alalunga} & 1929-2005 & 2005 & 0.81 & 1.49 & yes & \cite{2007-ALB-STOCK-ASSESS-REP.pdf} \\ 
  ICCAT & Bluefin tuna Eastern Atlantic & \textit{Thunnus thynnus} & 1969-2007 & 2007 & 0.34 & 9.38 & yes & \cite{2008-BFT-STOCK-ASSESS-REP.pdf} \\ 
  ICCAT & Bluefin tuna Western Atlantic & \textit{Thunnus thynnus} & 1969-2007 & 2007 & 0.57 & 1.33 & yes & \cite{2008-BFT-STOCK-ASSESS-REP.pdf} \\ 
  ICCAT & Bigeye tuna Atlantic & \textit{Thunnus obesus} & 1950-2005 & 2005 & 0.90 & 0.86 & no & \cite{JENSEN-BIGEYEATL-2008.pdf} \\ 
  ICCAT & Skipjack tuna Eastern Atlantic & \textit{Katsuwonus pelamis} & 1950-2006 & 2006 & 1.71 & 0.27 & no & \cite{JENSEN_YFINATL-2008.pdf} \\ 
  ICCAT & Skipjack tuna Western Atlantic & \textit{Katsuwonus pelamis} & 1952-2006 & 2006 & 1.72 & 0.27 & no & \cite{JENSEN_YFINATL-2008.pdf} \\ 
  ICCAT & Swordfish Mediterranean Sea & \textit{Xiphias gladius} & 1968-2006 & 2006 & 0.94 & 1.27 & yes & \cite{ICCAT-Mediterranean-Xiphiasgladius-2007.pdf} \\ 
  ICCAT & Swordfish North Atlantic & \textit{Xiphias gladius} & 1978-2007 & 2005 & 1.03 & 0.82 & no & \cite{JENSEN_SWORDSATL-2007.pdf} \\ 
  ICCAT & Swordfish South Atlantic & \textit{Xiphias gladius} & 1950-2005 & 2005 & 1.18 & 0.69 & no & \cite{JENSEN_SWORDSATL-2007.pdf} \\ 
  ICCAT & Yellowfin tuna Atlantic & \textit{Thunnus albacares} & 1970-2006 & 2006 & 1.07 & 0.81 & yes & \cite{JENSEN-YFINATL-2008.pdf} \\ 
  ICES & Capelin Barents Sea & \textit{Mallotus villosus} & 1965-2007 & 2006 & 0.17 & 0.00 & no & \cite{ICES-AFWG-2007.pdf} \\ 
  ICES & Atlantic cod coastal Norway & \textit{Gadus morhua} & 1982-2006 & 2006 & 0.27 & 2.17 & no & \cite{ICES-AFWG-2007.pdf} \\ 
  ICES & Atlantic cod Northeast Arctic & \textit{Gadus morhua} & 1943-2006 & 2006 & 0.56 & 1.42 & no & \cite{AFWG-NEAR-Gadusmorhua-2007.pdf} \\ 
  ICES & Greenland halibut Northeast Arctic & \textit{Reinhardtius hippoglossoides} & 1959-2007 & 2006 & 0.36 & 1.20 & no & \cite{ICES-AFWG-2007.pdf} \\ 
  ICES & Golden Redfish Northeast Arctic & \textit{Sebastes norvegicus} & 1986-2006 & 2006 & 0.29 & 2.65 & no & \cite{ICES-AFWG-2007.pdf} \\ 
  ICES & Haddock Northeast Arctic & \textit{Melanogrammus aeglefinus} & 1947-2006 & 2006 & 1.10 & 1.06 & no & \cite{ICES-AFWG-2007.pdf} \\ 
  ICES & Pollock Northeast Arctic & \textit{Pollachius virens} & 1957-2006 & 2006 & 1.70 & 0.60 & no & \cite{ICES-AFWG-2007.pdf} \\ 
  ICES & Herring ICES 22-24-IIIa & \textit{Clupea harengus} & 1991-2006 & 2006 & 0.73 & 1.02 & no & \cite{ICES-HAWG-2007.pdf} \\ 
  ICES & Herring Northern Irish Sea & \textit{Clupea harengus} & 1960-2006 & 2006 & 0.72 & 0.34 & no & \cite{ICES-HAWG-2007.pdf} \\ 
  ICES & Herring North Sea & \textit{Clupea harengus} & 1960-2007 & 2006 & 0.65 & 1.32 & no & \cite{ICES-HAWG-2007.pdf} \\ 
  ICES & Herring ICES VIa & \textit{Clupea harengus} & 1957-2006 & 2006 & 0.18 & 1.59 & no & \cite{ICES-HAWG-2007.pdf} \\ 
  ICES & Herring ICES VIa-VIIb-VIIc & \textit{Clupea harengus} & 1969-2000 & 2000 & 0.50 & 1.04 & no & \cite{ICES-HAWG-2007.pdf} \\ 
  ICES & Capelin Iceland & \textit{Mallotus villosus} & 1977-2007 & 2006 & 0.49 & 0.85 & no & \cite{ICES-NWWG-2007.pdf} \\ 
  ICES & Atlantic cod Faroe Plateau & \textit{Gadus morhua} & 1959-2006 & 2006 & 0.26 & 1.52 & no & \cite{ICES-NWWG-2007.pdf} \\ 
  ICES & Atlantic cod Iceland & \textit{Gadus morhua} & 1952-2006 & 2006 & 0.46 & 1.17 & no & \cite{ICES-NWWG-2007.pdf} \\ 
  ICES & Haddock Faroe Plateau & \textit{Melanogrammus aeglefinus} & 1955-2006 & 2006 & 0.85 & 1.07 & no & \cite{ICES-NWWG-2007.pdf} \\ 
  ICES & Haddock Iceland & \textit{Melanogrammus aeglefinus} & 1977-2007 & 2007 & 0.98 & 1.23 & no & \cite{ICES-NWWG-2007.pdf} \\ 
  ICES & Pollock Faroe Plateau & \textit{Pollachius virens} & 1958-2006 & 2006 & 0.99 & 1.52 & no & \cite{ICES-NWWG-2007.pdf} \\ 
  ICES & Atlantic cod Baltic Areas 22 and 24 & \textit{Gadus morhua} & 1969-2007 & 2006 & 0.36 & 1.43 & no & \cite{ICES-WGBFAS-2007.pdf} \\ 
  ICES & Atlantic cod Baltic Areas 25-32 & \textit{Gadus morhua} & 1964-2007 & 2006 & 0.16 & 1.46 & no & \cite{ICES-WGBFAS-2007.pdf} \\ 
  ICES & Atlantic cod Kattegat & \textit{Gadus morhua} & 1970-2006 & 2006 & 0.19 & 0.31 & no & \cite{ICES-WGBFAS-2007.pdf} \\ 
  ICES & Herring ICES 25-32 & \textit{Clupea harengus} & 1973-2006 & 2006 & 0.69 & 0.79 & no & \cite{ICES-WGBFAS-2007.pdf} \\ 
  ICES & Herring ICES 30 & \textit{Clupea harengus} & 1972-2007 & 2006 & 1.19 & 1.10 & no & \cite{ICES-WGBFAS-2007.pdf} \\ 
  ICES & Herring ICES 31 & \textit{Clupea harengus} & 1979-2006 & 2006 & 0.29 & 1.60 & no & \cite{ICES-WGBFAS-2007.pdf} \\ 
  ICES & Herring Iceland (Summer spawners) & \textit{Clupea harengus} & 1983-2007 & 2006 & 1.00 & 0.79 & no & \cite{ICES-NWWG-2007.pdf} \\ 
  ICES & Herring ICES 28 & \textit{Clupea harengus} & 1976-2007 & 2006 & 1.21 & 0.87 & no & \cite{ICES-WGBFAS-2007.pdf} \\ 
  ICES & common European sole ICES Kattegat and Skagerrak & \textit{Solea vulgaris} & 1982-2007 & 2006 & 1.25 & 0.54 & no & \cite{ICES-WGBFAS-2007.pdf} \\ 
  ICES & Sprat ICES Baltic Areas 22-32 & \textit{Sprattus sprattus} & 1973-2007 & 2006 & 1.13 & 1.27 & no & \cite{ICES-WGBFAS-2007.pdf} \\ 
  ICES & Fourspotted megrim ICES VIIIc-IXa & \textit{Lepidorhombus boscii} & 1986-2006 & 2006 & 0.70 & 1.01 & no & \cite{ICES-WGHMM-2007.pdf} \\ 
  ICES & Hake Northeast Atlantic North & \textit{Merluccius merluccius} & 1977-2007 & 2006 & 1.04 & 0.74 & no & \cite{ICES-WGHMM-2007.pdf} \\ 
  ICES & Megrim ICES VIIIc-IXa & \textit{Lepidorhombus whiffiagonis} & 1985-2007 & 2006 & 0.43 & 1.07 & no & \cite{ICES-WGHMM-2007.pdf} \\ 
  ICES & common European sole Bay of Biscay & \textit{Solea vulgaris} & 1982-2006 & 2006 & 0.76 & 1.00 & no & \cite{ICES-WGHMM-2007.pdf} \\ 
  ICES & Mackerel ICES Northeast Atlantic & \textit{Scomber scombrus} & 1972-2007 & 2006 & 0.98 & 0.73 & no & \cite{ICES-WGMHSA07.pdf} \\ 
  ICES & Whiting Northeast Atlantic & \textit{Micromesistius poutassou} & 1980-2007 & 2006 & 0.67 & 1.66 & no & \cite{ICES-WGNPBW-2007.pdf} \\ 
  ICES & Atlantic cod Irish Sea & \textit{Gadus morhua} & 1968-2006 & 2006 & 0.15 & 0.56 & no & \cite{ICES-WGNSDS-2007.pdf} \\ 
  ICES & Atlantic cod West of Scotland & \textit{Gadus morhua} & 1977-2006 & 2006 & 0.12 & 0.42 & no & \cite{ICES-WGNSDS-2007.pdf} \\ 
  ICES & Haddock West of Scotland & \textit{Melanogrammus aeglefinus} & 1977-2006 & 2006 & 0.58 & 0.73 & no & \cite{ICES-WGNSDS-2007.pdf} \\ 
  ICES & European Plaice Irish Sea & \textit{Pleuronectes platessa} & 1962-2006 & 2006 & 1.07 & 0.23 & no & \cite{ICES-WGNSDS-2007.pdf} \\ 
  ICES & common European sole Irish Sea & \textit{Solea vulgaris} & 1968-2006 & 2006 & 0.36 & 1.16 & no & \cite{ICES-WGNSDS-2007.pdf} \\ 
  ICES & Atlantic cod North Sea & \textit{Gadus morhua} & 1962-2007 & 2006 & 0.19 & 0.80 & no & \cite{ICES-WGNSSK-2007.pdf} \\ 
  ICES & Haddock ICES IIIa and North Sea & \textit{Melanogrammus aeglefinus} & 1963-2006 & 2006 & 0.62 & 0.25 & no & \cite{ICES-WGNSSK-2007.pdf} \\ 
  ICES & Haddock Rockall Bank & \textit{Melanogrammus aeglefinus} & 1990-2007 & 2006 & 1.10 & 0.52 & no & \cite{ICES-WGNSDS-2007.pdf} \\ 
  ICES & Norway pout North Sea & \textit{Trisopterus esmarkii} & 1983-2007 & 2006 & 0.90 & 0.33 & no & \cite{ICES-WGNSSK-2007.pdf} \\ 
  ICES & Pollock ICES IIIa, VI and North Sea & \textit{Pollachius virens} & 1964-2006 & 2006 & 0.57 & 0.97 & no & \cite{ICES-WGNSSK-2007.pdf} \\ 
  ICES & Sandeel North Sea & \textit{Ammodytes marinus} & 1983-2007 & 2007 & 0.92 & 0.24 & no & \cite{ICES-WGNSSK-2007.pdf} \\ 
  ICES & Whiting ICES IIIa, VIId and North Sea & \textit{Merlangius merlangus} & 1979-2006 & 2006 & 0.33 & 1.04 & no & \cite{ICES-WGNSSK-2007.pdf} \\ 
  ICES & Haddock ICES VIIb-k & \textit{Melanogrammus aeglefinus} & 1993-2006 & 2006 & 1.37 & 0.41 & no & \cite{ICES-WGSSDS-2007.pdf} \\ 
  ICES & European Plaice ICES VIIf-g & \textit{Pleuronectes platessa} & 1976-2006 & 2006 & 0.65 & 0.41 & no & \cite{ICES-WGSSDS-2007.pdf} \\ 
  ICES & European Plaice ICES VIIe & \textit{Pleuronectes platessa} & 1975-2006 & 2006 & 0.51 & 1.39 & no & \cite{ICES-WGSSDS-2007.pdf} \\ 
  ICES & common European sole Celtic Sea & \textit{Solea vulgaris} & 1970-2006 & 2006 & 0.90 & 0.95 & no & \cite{ICES-WGSSDS-2007.pdf} \\ 
  ICES & common European sole Western English Channel & \textit{Solea vulgaris} & 1968-2006 & 2006 & 0.51 & 1.74 & no & \cite{ICES-WGSSDS-2007.pdf} \\ 
  ICES & Whiting ICES VIIe-k & \textit{Merlangius merlangus} & 1982-2007 & 2006 & 0.44 & 1.25 & no & \cite{ICES-WGSSDS-2007.pdf} \\ 
  IOTC & Bigeye tuna Indian Ocean & \textit{Thunnus obesus} & 1957-2006 & 2004 & 1.23 & 0.97 & yes & \cite{JENSEN_BIGEYEIO-2007.pdf} \\ 
  IPHC & Pacific halibut North Pacific & \textit{Hippoglossus stenolepis} & 1988-2009 & 2008 & 0.54 & 2.01 & no & \cite{hare-clark08.pdf} \\ 
  MFish & Australian salmon New Zealand & \textit{Arripis trutta} & 1975-2006 & 2006 & 1.64 & 0.33 & yes & \cite{NA} \\ 
  MFish & Orange roughy New Zealand Mid East Coast & \textit{Hoplostethus atlanticus} & 1981-2004 & 2004 & 1.20 & 0.35 & yes & \cite{NA} \\ 
  MFish & Black oreo West end of Chatham Rise & \textit{Allocyttus niger} & 1973-2007 & 2007 & 0.99 & 0.82 & yes & \cite{NA} \\ 
  MFish & Smooth oreo Chatham Rise & \textit{Pseudocyttus maculatus} & 1979-2006 & 2006 & 2.25 & 0.38 & yes & \cite{NA} \\ 
  MFish & Smooth oreo West end of Chatham Rise & \textit{Pseudocyttus maculatus} & 1973-2004 & 2004 & 1.25 & 0.53 & yes & \cite{NA} \\ 
  MFish & Hoki Eastern New Zealand & \textit{Macruronus novaezelandiae} & 1972-2007 & 2007 & 1.11 & 0.33 & no & \cite{FAR0804hok07.pdf} \\ 
  MFish & Hoki Western New Zealand & \textit{Macruronus novaezelandiae} & 1972-2007 & 2007 & 0.51 & 0.57 & no & \cite{FAR0804hok07.pdf} \\ 
  MFish & New Zealand snapper New Zealand Area 8 & \textit{Chrysophrys auratus} & 1931-2005 & 2005 & 0.35 & 2.50 & yes & \cite{NA} \\ 
  MFish & Trevally New Zealand Areas TRE 7 & \textit{Pseudocaranx dentex} & 1944-2005 & 2005 & 1.44 & 0.83 & yes & \cite{NA} \\ 
  MFish & Red rock lobster New Zealand area CRA1 & \textit{Jasus edwardsii} & 1945-2001 & 2001 & 1.14 & 0.88 & no & \cite{NA} \\ 
  MFish & Red rock lobster New Zealand area CRA2 & \textit{Jasus edwardsii} & 1945-2001 & 2001 & 0.53 & 2.12 & no & \cite{NA} \\ 
  MFish & Red rock lobster New Zealand area CRA4 & \textit{Jasus edwardsii} & 1945-2005 & 2005 & 0.67 & 1.33 & no & \cite{NA} \\ 
  MFish & Red rock lobster New Zealand area CRA5 & \textit{Jasus edwardsii} & 1945-2002 & 2002 & 0.59 & 1.68 & no & \cite{NA} \\ 
  MFish & Red rock lobster New Zealand area CRA7 & \textit{Jasus edwardsii} & 1976-2005 & 2005 & 0.73 & 0.43 & no & \cite{NA} \\ 
  MFish & Red rock lobster New Zealand area CRA8 & \textit{Jasus edwardsii} & 1976-2005 & 2005 & 0.69 & 0.49 & no & \cite{NA} \\ 
  MFish & common gemfish New Zealand & \textit{Rexea solandri} & 1952-2007 & 2006 & 1.64 & 0.43 & yes & \cite{NA} \\ 
  MFish & New Zealand ling New Zealand Areas LIN 3 and 4 & \textit{Genypterus blacodes} & 1972-2007 & 2007 & 3.07 & 0.09 & yes & \cite{NA} \\ 
  MFish & New Zealand ling New Zealand Areas LIN 5 and 6 & \textit{Genypterus blacodes} & 1972-2007 & 2007 & 3.96 & 0.10 & yes & \cite{NA} \\ 
  MFish & New Zealand ling New Zealand Area LIN 6b & \textit{Genypterus blacodes} & 1980-2006 & 2006 & 2.19 & 0.11 & yes & \cite{NA} \\ 
  MFish & New Zealand ling New Zealand Area LIN 72 & \textit{Genypterus blacodes} & 1972-2007 & 2007 & 2.49 & 0.32 & yes & \cite{NA} \\ 
  MFish & New Zealand ling New Zealand Area LIN 7WC - WCSI & \textit{Genypterus blacodes} & 1972-2008 & 2008 & 2.21 & 0.13 & yes & \cite{NA} \\ 
  MFish & Southern blue whiting Campbell Island Rise & \textit{Micromesistius australis} & 1979-2006 & 2006 & 1.15 & 0.92 & yes & \cite{NA} \\ 
  MFish & Southern hake Chatham Rise & \textit{Merluccius australis} & 1975-2006 & 2006 & 1.77 & 0.12 & yes & \cite{NA} \\ 
  MFish & Southern hake Sub-Antarctic & \textit{Merluccius australis} & 1975-2007 & 2007 & 2.91 & 0.11 & yes & \cite{NA} \\ 
  MFish & New Zealand abalone species New Zealand Area PAU 5A & \textit{Haliotis iris} & 1964-2006 & 2006 & 0.72 & 2.83 & no & \cite{07-09-FAR.pdf} \\ 
  MFish & New Zealand abalone species New Zealand Area PAU 5B & \textit{Haliotis iris} & 1963-2007 & 2007 & 1.02 & 0.59 & no & \cite{08-05-FAR.pdf} \\ 
  MFish & New Zealand abalone species New Zealand Area PAU 5D & \textit{Haliotis iris} & 1964-2006 & 2006 & 0.44 & 2.10 & no & \cite{07-09-FAR.pdf} \\ 
  MFish & New Zealand abalone species New Zealand Area PAU 7 & \textit{Haliotis iris} & 1964-2008 & 2008 & 0.87 & 0.94 & no & \cite{NA} \\ 
  NAFO & American Plaice NAFO-3LNO & \textit{Hippoglossoides platessoides} & 1955-2007 & 2006 & 0.08 & 0.77 & no & \cite{NAFO-GrandBanks-AmPlaice-2007.pdf} \\ 
  NAFO & American Plaice NAFO-3M & \textit{Hippoglossoides platessoides} & 1960-2007 & 2007 & 0.13 & 0.00 & no & \cite{NAFO-AMPL3M-2008.pdf} \\ 
  NAFO & Atlantic cod NAFO 3NO & \textit{Gadus morhua} & 1953-2007 & 2006 & 0.02 & 0.27 & no & \cite{NAFO-3NO-COD-2007.pdf} \\ 
  NAFO & Greenland halibut NAFO 23KLMNO & \textit{Reinhardtius hippoglossoides} & 1960-2006 & 2006 & 0.39 & 1.73 & no & \cite{NAFO-GHAL23KLMNO-2007.pdf} \\ 
  NAFO & Redfish species NAFO 3LN & \textit{Redfish species} & 1959-2007 & 2006 & 1.95 & 0.01 & no & \cite{NAFO-RED3LN-2007.pdf} \\ 
  NAFO & Redfish species NAFO 3LN & \textit{Redfish species} & 1959-2008 & 2008 & 1.88 & 0.04 & yes & \cite{NAFO-3LN-Redfishspp-2008.pdf} \\ 
  NAFO & Redfish species NAFO 3M & \textit{Redfish species} & 1989-2006 & 2006 & 0.93 & 0.15 & no & \cite{NAFO-RED3M-2007.pdf} \\ 
  NAFO & Yellowtail Flounder NAFO 3LNO & \textit{Limanda ferruginea} & 1960-2009 & 2007 & 1.62 & 0.15 & no & \cite{NAFO-YELL3LNO-2008.pdf} \\ 
  NMFS & Alaska plaice Bering Sea and Aleutian Islands & \textit{Pleuronectes quadrituberculatus} & 1972-2008 & 2008 & 2.46 & 0.07 & yes & \cite{AFSC-ALPLAICBSAI-2008-Alaska plaice BSAI.pdf} \\ 
  NMFS & Arrowtooth flounder Bering Sea and Aleutian Islands & \textit{Reinhardtius stomias} & 1970-2008 & 2008 & 1.28 & 0.31 & no & \cite{AFSC-ARFLOUNDBSAI-2007-Arrowtooth flounder BSAI.pdf} \\ 
  NMFS & Arrowtooth flounder Gulf of Alaska & \textit{Reinhardtius stomias} & 1958-2010 & 2007 & 1.33 & 0.28 & no & \cite{2008_SAFE_GOAatf.pdf} \\ 
  NMFS & Atka mackerel Bering Sea and Aleutian Islands & \textit{Pleurogrammus monopterygius} & 1976-2009 & 2008 & 1.50 & 0.55 & no & \cite{2008_SAFE_BSAIatka.pdf} \\ 
  NMFS & Cabezon Northern California & \textit{Scorpaenichthys marmoratus} & 1916-2005 & 2004 & 0.89 & 0.99 & no & \cite{2005-SAFE-WCcabezon.pdf} \\ 
  NMFS & Cabezon Southern California & \textit{Scorpaenichthys marmoratus} & 1932-2005 & 2004 & 0.81 & 0.53 & no & \cite{2005_SAFE_Wccabezon.pdf} \\ 
  NMFS & Dusky rockfish Gulf of Alaska & \textit{Sebastes variabilis} & 1973-2008 & 2007 & 1.54 & 0.54 & yes & \cite{AFSC-DUSROCKGA-2008-Dusky rockfish GA.pdf} \\ 
  NMFS & Flathead sole Bering Sea and Aleutian Islands & \textit{Hippoglossoides elassodon} & 1977-2008 & 2008 & 1.66 & 0.18 & no & \cite{2008_SAFE_BSAIflathead.pdf} \\ 
  NMFS & Greenland turbot Bering Sea and Aleutian Islands & \textit{Reinhardtius hippoglossoides} & 1960-2009 & 2009 & 1.48 & 0.05 & yes & \cite{2008_SAFE_BSAIturbot.pdf} \\ 
  NMFS & Northern rockfish Bering Sea and Aleutian Islands & \textit{Sebastes polyspinis} & 1974-2009 & 2008 & 1.07 & 0.13 & no & \cite{2008_SAFE_BSAInorthern.pdf} \\ 
  NMFS & Northern rockfish Gulf of Alaska & \textit{Sebastes polyspinis} & 1959-2008 & 2008 & 1.50 & 0.66 & yes & \cite{AFSC-NROCKGA-2008-Northern rockfish GA.pdf} \\ 
  NMFS & Northern rock sole Eastern Bering Sea and Aleutian Islands & \textit{Lepidopsetta polyxystra} & 1971-2008 & 2007 & 3.02 & 0.21 & yes & \cite{2008_SAFE_BSAIrocksole.pdf} \\ 
  NMFS & Pacific cod Bering Sea and Aleutian Islands & \textit{Gadus macrocephalus} & 1964-2008 & 2007 & 0.89 & 0.93 & no & \cite{AFSC-PCODBSAI-2008-Pacific cod BSAI.pdf} \\ 
  NMFS & Pacific cod Gulf of Alaska & \textit{Gadus macrocephalus} & 1964-2008 & 2007 & 1.07 & 0.84 & no & \cite{AFSC-PCODGA-2008-Pacific cod GA.pdf} \\ 
  NMFS & Pacific Ocean perch Eastern Bering Sea and Aleutian Islands & \textit{Sebastes alutus} & 1974-2009 & 2008 & 1.70 & 0.26 & no & \cite{2008_SAFE_BSAIpop.pdf} \\ 
  NMFS & Pacific ocean perch Gulf of Alaska & \textit{Sebastes alutus} & 1959-2008 & 2008 & 1.16 & 0.73 & yes & \cite{AFSC-POPERCHGA-2008-Pacific ocean perch GA.pdf} \\ 
  NMFS & Red king crab Bristol Bay & \textit{Paralithodes camtschaticus} & 1960-2008 & 2008 & 1.27 & 1.05 & yes & \cite{CRABSAFE2008.pdf} \\ 
  NMFS & Sablefish Eastern Bering Sea / Aleutian Islands / Gulf of Alaska & \textit{Anoplopoma fimbria} & 1956-2008 & 2008 & 1.05 & 0.66 & yes & \cite{AFSC-SABLEFEBSAIGA-2008-Sablefish EBS AI GA.pdf} \\ 
  NMFS & Snow crab Bering Sea & \textit{Chionoecetes opilio} & 1979-2008 & 2008 & 0.35 & 1.49 & no & \cite{CRABSAFE2008.pdf} \\ 
  NMFS & Walleye pollock Aleutian Islands & \textit{Theragra chalcogramma} & 1976-2008 & 2008 & 0.86 & 0.02 & yes & \cite{AFSC-WPOLLAI-2008-Walleye pollock AI.pdf} \\ 
  NMFS & Walleye pollock Eastern Bering Sea & \textit{Theragra chalcogramma} & 1963-2008 & 2008 & 0.68 & 0.85 & no & \cite{AFSC-WPOLLEBS-2008-Walleye pollock EBS.pdf} \\ 
  NMFS & Yellowfin sole Bering Sea and Aleutian Islands & \textit{Limanda aspera} & 1959-2008 & 2008 & 1.94 & 0.62 & yes & \cite{AFSC-YSOLEBSAI-2008-Yellowfin sole BSAI.pdf} \\ 
  NMFS & Atlantic croaker Mid-Atlantic Coast & \textit{Micropogonias undulatus} & 1973-2002 & 2002 & 1.42 & 0.27 & yes & \cite{2004_ASMFC_AtlCroak.pdf} \\ 
  NMFS & Northern shrimp Gulf of Maine & \textit{Pandalus borealis} & 1960-2009 & 2008 & 1.58 & 0.56 & no & \cite{2008ShrimpAssessment.pdf} \\ 
  NMFS & American Plaice NAFO-5YZ & \textit{Hippoglossoides platessoides} & 1960-2008 & 2007 & 0.55 & 0.30 & no & \cite{ .pdf} \\ 
  NMFS & Bluefish Atlantic Coast & \textit{Pomatomus saltatrix} & 1981-2007 & 2007 & 0.81 & 1.25 & no & \cite{final-2005-SAW-41-assessment.pdf} \\ 
  NMFS & Black sea bass Mid-Atlantic Coast & \textit{Centropristis striata} & 1968-2007 & 2007 & 1.21 & 0.67 & no & \cite{DataPoorReviewPanelReportFinal-1-20-09.pdf} \\ 
  NMFS & Atlantic cod Georges Bank & \textit{Gadus morhua} & 1960-2008 & 2007 & 0.18 & 0.72 & no & \cite{NMFS-GB-Gadusmorhua-2008.pdf} \\ 
  NMFS & Atlantic cod Gulf of Maine & \textit{Gadus morhua} & 1893-2008 & 2007 & 1.46 & 0.29 & no & \cite{NMFS-GOM-Gadusmorhua-2008.pdf} \\ 
  NMFS & Haddock NAFO-5Y & \textit{Melanogrammus aeglefinus} & 1956-2008 & 2007 & 0.36 & 1.21 & no & \cite{NMFS-GOM-Melanogrammusaeglefinus-2008.pdf} \\ 
  NMFS & Monkfish Gulf of Maine / Northern Georges Bank & \textit{Lophius americanus} & 1964-2006 & 2006 & 1.73 & 0.38 & no & \cite{crd0721.pdf} \\ 
  NMFS & Monkfish Southern Georges Bank / Mid-Atlantic & \textit{Lophius americanus} & 1964-2006 & 2006 & 1.72 & 0.30 & no & \cite{Monkfish2007NEFSCAssessment.pdf} \\ 
  NMFS & Sea scallop Georges Bank & \textit{Placopecten magellanicus} & 1964-2006 & 2006 & 1.59 & 0.78 & no & \cite{SeaScallop2007.pdf} \\ 
  NMFS & Sea scallop Mid-Atlantic Coast & \textit{Placopecten magellanicus} & 1964-2006 & 2006 & 1.00 & 0.36 & no & \cite{SeaScallop2007.pdf} \\ 
  NMFS & Spiny dogfish Atlantic Coast & \textit{Squalus acanthias} & 1962-2006 & 2005 & 1.61 & 0.15 & no & \cite{spinydogfish2006.pdf} \\ 
  NMFS & Atlantic surfclam Mid-Atlantic Coast & \textit{Spisula solidissima} & 1965-2008 & 1994 & 1.85 & 0.00 & no & \cite{Surfclam2007.pdf} \\ 
  NMFS & Tilefish Mid-Atlantic Coast & \textit{Lopholatilus chamaeleonticeps} & 1973-2008 & 2005 & 0.72 & 0.61 & no & \cite{Tilefish2005.pdf} \\ 
  NMFS & Weakfish Atlantic Coast & \textit{Cynoscion regalis} & 1980-2008 & 2008 & 0.06 & 1.49 & no & \cite{NEFSC-Weakfish-2009.pdf} \\ 
  NMFS & White hake Georges Bank / Gulf of Maine & \textit{Urophycis tenuis} & 1963-2007 & 2007 & 0.35 & 0.80 & yes & \cite{WhiteHake2008.pdf} \\ 
  NMFS & Winter Flounder NAFO-5Z & \textit{Pseudopleuronectes americanus} & 1982-2007 & 2006 & 1.41 & 0.25 & no & \cite{garm3k.pdf} \\ 
  NMFS & Winter Flounder Southern New England-Mid Atlantic & \textit{Pseudopleuronectes americanus} & 1940-2007 & 2007 & 0.17 & 1.10 & no & \cite{NMFS-SNEMATL-Pseudopleuronectesamercianus-2008.pdf} \\ 
  NMFS & Witch Flounder NAFO-5Y & \textit{Glyptocephalus cynoglossus} & 1982-2008 & 2007 & 0.30 & 1.45 & yes & \cite{NA} \\ 
  NMFS & Yellowtail flounder Cape Cod / Gulf of Maine & \textit{Limanda ferruginea} & 1935-2008 & 2007 & 0.25 & 1.73 & yes & \cite{NMFS-CCGOM-Limandaferruginea-2008.pdf} \\ 
  NMFS & Yellowtail flounder Georges Bank & \textit{Limanda ferruginea} & 1935-2008 & 2007 & 0.22 & 1.14 & yes & \cite{NMFS-GB-Limandaferruginea-2008.pdf} \\ 
  NMFS & Yellowtail Flounder Southern New England-Mid Atlantic & \textit{Limanda ferruginea} & 1935-2008 & 2007 & 0.13 & 1.61 & yes & \cite{NMFS-SNEMATL-Limandaferruginea-2008.pdf} \\ 
  NMFS & Atlantic menhaden Atlantic & \textit{Brevoortia tyrannus} & 1940-2005 & 2005 & 0.47 & 0.97 & no & \cite{Atl.Menhaden-ASMFC-2006.pdf} \\ 
  NMFS & Pacific sardine North Pacific & \textit{Sardinops sagax} & 1981-2008 & 2006 & 1.73 & 0.37 & no & \cite{2008 pac sardine.pdf} \\ 
  NMFS & Arrowtooth flounder Pacific Coast & \textit{Reinhardtius stomias} & 1916-2007 & 2007 & 3.81 & 0.21 & yes & \cite{NWFSC-ARFLOUNDPCOAST-2007-Arrowtooth flounder.pdf} \\ 
  NMFS & Blackgill rockfish  Pacific Coast & \textit{Sebastes melanostomus} & 1950-2005 & 2004 & 0.80 & 1.20 & no & \cite{2005-SAFE-Wcblackgill.pdf} \\ 
  NMFS & Black rockfish Northern Pacific Coast & \textit{Sebastes melanops} & 1914-2006 & 2006 & 1.37 & 0.57 & no & \cite{NWFSC-BLACKROCKNPCOAST-2007-Black rockfish NOR WA.pdf} \\ 
  NMFS & Black rockfish Southern Pacific Coast & \textit{Sebastes melanops} & 1915-2007 & 2007 & 2.23 & 0.33 & yes & \cite{NWFSC-BLACKROCKSPCOAST-2007-Black rockfish OR CA.pdf} \\ 
  NMFS & Blue rockfish California & \textit{Sebastes mystinus} & 1916-2007 & 2007 & 0.75 & 1.19 & yes & \cite{NWFSC-BLUEROCKCAL-2007-Blue rockfish CA.pdf} \\ 
  NMFS & Bocaccio Southern Pacific Coast & \textit{Sebastes paucispinis} & 1951-2006 & 2006 & 0.32 & 0.10 & yes & \cite{NWFSC-BOCACCSPCOAST-2007 Bocaccio.pdf} \\ 
  NMFS & Chilipepper Southern Pacific Coast & \textit{Sebastes goodei} & 1892-2007 & 2006 & 1.43 & 0.04 & no & \cite{NWFSC-CHILISPCOAST-2007-Chilipepper CA OR.pdf} \\ 
  NMFS & Cowcod Southern California & \textit{Sebastes levis} & 1900-2007 & 2007 & 0.09 & 0.07 & yes & \cite{NWFSC-COWCODSCAL-2007-Cowcod CA.pdf} \\ 
  NMFS & Canary rockfish Pacific Coast & \textit{Sebastes pinniger} & 1916-2007 & 2007 & 0.85 & 0.02 & yes & \cite{NWFSC-CROCKPCOAST-2007-Canary.pdf} \\ 
  NMFS & Darkblotched rockfish Pacific Coast & \textit{Sebastes crameri} & 1928-2007 & 2007 & 0.73 & 0.31 & yes & \cite{NWFSC-DKROCKPCOAST-2008-Darkblotched rockfish.pdf} \\ 
  NMFS & English sole Pacific Coast & \textit{Parophrys vetulus} & 1876-2007 & 2006 & 2.06 & 0.14 & no & \cite{NWFSC-ESOLEPCOAST-2007-EnglishSole.pdf} \\ 
  NMFS & Longnose skate Pacific Coast & \textit{Raja rhina} & 1915-2007 & 2007 & 1.56 & 0.40 & no & \cite{NWFSC-LNOSESKAPCOAST-2008-Longnose skate.pdf} \\ 
  NMFS & Longspine thornyhead Pacific Coast & \textit{Sebastolobus altivelis} & 1962-2005 & 2005 & 2.65 & 0.23 & yes & \cite{2005-SAFE-Longspine.pdf} \\ 
  NMFS & Pacific hake Pacific Coast & \textit{Merluccius productus} & 1966-2008 & 2008 & 1.61 & 0.73 & yes & \cite{NWFSC-PHAKEPCOAST-2008-Pacific-Hake-US-Canada.pdf} \\ 
  NMFS & Pacific ocean perch Pacific Coast & \textit{Sebastes alutus} & 1953-2007 & 2007 & 0.69 & 0.00 & yes & \cite{NWFSC-POPERCHPCOAST-2007-Pacific ocean perch.pdf} \\ 
  NMFS & Petrale sole Northern Pacific Coast & \textit{Eopsetta jordani} & 1910-2005 & 2004 & 0.96 & 1.26 & no & \cite{2004_SAFE_WCpetralesole.pdf} \\ 
  NMFS & Petrale sole Southern Pacific Coast & \textit{Eopsetta jordani} & 1874-2005 & 2004 & 0.63 & 0.61 & no & \cite{2004-SAFE-WCpetralesole.pdf} \\ 
  NMFS & Sablefish Pacific Coast & \textit{Anoplopoma fimbria} & 1900-2007 & 2007 & 0.84 & 1.09 & no & \cite{NWFSC-SABLEFPCOAST-2007-Sablefish.pdf} \\ 
  NMFS & Shortspine thornyhead Pacific Coast & \textit{Sebastolobus alascanus} & 1901-2005 & 2004 & 1.55 & 0.93 & no & \cite{2005-SST-assessment.pdf} \\ 
  NMFS & Widow rockfish Pacific Coast & \textit{Sebastes entomelas} & 1955-2006 & 2006 & 0.91 & 0.07 & no & \cite{NWFSC-WROCKPCOAST-2007-widow.pdf} \\ 
  NMFS & Yelloweye rockfish Pacific Coast & \textit{Sebastes ruberrimus} & 1923-2006 & 2006 & 0.38 & 0.34 & no & \cite{NWFSC-YEYEROCKPCOAST-2007-yelloweye.pdf} \\ 
  NMFS & Yellowtail rockfish Northern Pacific Coast & \textit{Sebastes flavidus} & 1967-2005 & 2005 & 0.75 & 0.51 & no & \cite{2005_SAFE_yellowtail.pdf} \\ 
  NMFS & Gag Gulf of Mexico & \textit{Mycteroperca microlepis} & 1963-2004 & 2004 & 1.00 & 2.44 & yes & \cite{JENSEN_GAGGM_2007.pdf} \\ 
  NMFS & Gag Southern Atlantic coast & \textit{Mycteroperca microlepis} & 1962-2005 & 2005 & 0.94 & 1.31 & yes & \cite{JENSEN_GAGSATLC_2006.pdf} \\ 
  NMFS & Greater amberjack Gulf of Mexico & \textit{Seriola dumerili} & 1986-2004 & 2004 & 0.46 & 1.52 & no & \cite{JENSEN_GRAMBERGM_2006.pdf} \\ 
  NMFS & King mackerel Gulf of Mexico & \textit{Scomberomorus cavalla} & 1992-2001 & 2001 & 1.51 & 0.44 & no & \cite{JENSEN_KMACKGMSATLC_2004.pdf} \\ 
  NMFS & King mackerel Southern Atlantic Coast & \textit{Scomberomorus cavalla} & 1981-2001 & 2001 & 1.38 & 0.56 & no & \cite{JENSEN_KMACKGMSATLC_2004.pdf} \\ 
  NMFS & Gulf menhaden Gulf of Mexico & \textit{Brevoortia patronus} & 1964-2004 & 2004 & 1.08 & 0.48 & no & \cite{GILROY-MENHADENGM-2007.pdf} \\ 
  NMFS & Red grouper Gulf of Mexico & \textit{Epinephelus morio} & 1986-2005 & 2005 & 1.00 & 0.88 & no & \cite{JENSEN_RGROUPGM-2006.pdf} \\ 
  NMFS & Red porgy Southern Atlantic coast & \textit{Pagrus pagrus} & 1972-2004 & 2004 & 0.61 & 0.39 & yes & \cite{JENSEN_RPORGYSATLC_2006.pdf} \\ 
  NMFS & Snowy grouper Southern Atlantic coast & \textit{Epinephelus niveatus} & 1961-2002 & 2002 & 0.19 & 3.08 & yes & \cite{2004-SEDAR-deepwatersnappergrouper.pdf} \\ 
  NMFS & Spanish mackerel Southern Atlantic Coast & \textit{Scomberomorus maculatus} & 1950-2008 & 2007 & 0.38 & 0.91 & yes & \cite{JENSEN_SPANMACKSATLC_2008.pdf} \\ 
  NMFS & Tilefish Southern Atlantic coast & \textit{Lopholatilus chamaeleonticeps} & 1961-2002 & 2002 & 0.94 & 1.55 & yes & \cite{2004-SEDAR-deepwatersnappergrouper.pdf} \\ 
  NMFS & Vermilion snapper Southern Atlantic coast & \textit{Rhomboplites aurorubens} & 1946-2008 & 2007 & 0.86 & 1.27 & yes & \cite{2008_SEDAR_VermillionSnapper_Satl.pdf} \\ 
  NMFS & Yellowtail snapper Southern Atlantic Coast and Gulf of Mexico & \textit{Ocyurus chrysurus} & 1962-2001 & 2001 & 1.14 & 0.61 & yes & \cite{2003_SEDAR_Yellowtailsnapper.pdf} \\ 
  NMFS & Dover sole Pacific Coast & \textit{Microstomus pacificus} & 1910-2005 & 2004 & 1.33 & 0.45 & no & \cite{2005-SAFE-WCdover.pdf} \\ 
  NMFS & Gopher rockfish Southern Pacific Coast & \textit{Sebastes carnatus} & 1965-2005 & 2004 & 0.88 & 0.62 & no & \cite{2005-SAFE-Wcgopher.pdf} \\ 
  NMFS & Pacific sardine Pacific Coast & \textit{Sardinops sagax} & 1981-2007 & 2006 & 1.36 & 0.41 & no & \cite{NOAA-TM-NMFS-SWFSC-413.pdf} \\ 
  NMFS & Starry flounder Northern Pacific Coast & \textit{Platichthys stellatus} & 1970-2005 & 2004 & 0.68 & 0.33 & no & \cite{2005-SAFE-WCstarryflounder.pdf} \\ 
  NMFS & Starry flounder Southern Pacific Coast & \textit{Platichthys stellatus} & 1970-2005 & 2004 & 1.15 & 0.12 & no & \cite{2005-SAFE-WCstarryflounder.pdf} \\ 
  RFFA & Walleye pollock Northern Sea of Okhotsk & \textit{Theragra chalcogramma} & 1985-1994 & 1992 & 1.11 & 0.87 & no & \cite{WPOLLNSO-1997-JENSEN.pdf} \\ 
  RFFA & Walleye pollock Western Bering Sea & \textit{Theragra chalcogramma} & 1994-2004 & 2004 & 2.16 & 0.26 & no & \cite{WPOLLWBS-2004-JENSEN.pdf} \\ 
  SPRFMO & Chilean jack mackerel Chilean EEZ and offshore & \textit{Trachurus murphyi} & 1975-2007 & 2006 & 0.52 & 1.20 & no & \cite{JENSEN-JACKMACKCH-2008.pdf} \\ 
  US State & American lobster Rhode Island & \textit{Homarus americanus} & 1959-2007 & 2006 & 0.53 & 0.67 & no & \cite{NA} \\ 
  US State & Tautog Rhode Island & \textit{Tautoga onitis} & 1959-2007 & 2006 & 0.84 & 0.59 & no & \cite{NA} \\ 
  US State & Winter flounder Rhode Island & \textit{Pseudopleuronectes americanus} & 1959-2007 & 2006 & 0.25 & 1.59 & no & \cite{NA} \\ 
  WCPFC & Albacore tuna South Pacific Ocean & \textit{Thunnus alalunga} & 1959-2006 & 2006 & 2.46 & 0.90 & yes & \cite{JENSEN_ALBWPO_2008.pdf} \\ 
  WCPFC & Bigeye tuna Western Pacific Ocean & \textit{Thunnus obesus} & 1952-2006 & 2006 & 1.06 & 1.38 & yes & \cite{SC4-SA-WP1-rev1-bigeye-tuna.pdf} \\ 
  WCPFC & Skipjack tuna Central Western Pacific & \textit{Katsuwonus pelamis} & 1972-2006 & 2006 & 4.38 & 0.30 & yes & \cite{SC4-SA-WP4-SKJ-Assessment-rev1-skipjack.pdf} \\ 
  WCPFC & Yellowfin tuna Central Western Pacific & \textit{Thunnus albacares} & 1952-2005 & 2005 & 1.22 & 0.80 & yes & \cite{WCPFC-SC3-SA-SWG-WP-01.pdf} \\ 
   \hline
\hline
\caption{Summary of the assessments used in this analysis and their estimated ratios of current biomass to the biomass at maximum sustainable yield and current harvest rate to the harvest rate that results is maximum sustainable yield. The estimated ratios were preferentially obtained directly from the assessment document or derived from surplus production model fits. When both an SSBmsy and Bmsy reference points are available, the SSB is chosen preferentially.}
\label{tab:crosshair}
\end{longtable}

%\end{landscape}

%%\bibliographystyle{plain}
%\bibliography{./srdb-references}

%%\section*{Supporting Information}

\subsection*{Entity relationship diagram}
\begin{figure}
\begin{center}
\includegraphics[width=15cm]{/home/srdbadmin/SQLpg/srdb/trunk/doc/srdb-ERD.pdf}
\end{center}
\caption{Entity relationship diagram of the RAM legacy database.}
\end{figure}


\begin{table}
\caption{Data used to generate Figure~\ref{fig:orca} - Summary of the assessments used in this analysis and the timespan of their different results. }
\begin{tabular}{| p{5cm} | p{3cm} | c | r  c  l | c | }\label{tab:timespan}
\textit{Fisheries stock} & \textit{Scientific name} & \textit{Timespan} & & \textit{Num. years available} & & \textit{Source} \\
 & & & Catch & SSB & R & \\
\hline \hline
blah & blah & 1970-2000 & 30 & 29 & 30 & (blah) \\
\end{tabular}
\end{table}


%\begin{table}
%\caption{Data used to generate Figures~\ref{fig:friedegg} and ~\ref{fig:friedeggmgmt} - Summary of the assessments used in this analysis and their estimated ratios of current biomass to the biomass at maximum sustainable yield and current harvest rate to the harvest rate that results is maximum sustainable yield. The estimated ratios were preferentially obtained directly from the assessment document or derived from surplus production model fits. When both an SSBmsy and Bmsy reference points are available, the SSB is chosen preferentially. }
%\begin{tabular}{| p{5cm} | p{3cm} | c | c | c | c | c |}\label{tab:crosshair}
%\textit{Fisheries stock} & \textit{Scientific name} & \textit{Current year} & \textit{B/Bmsy} & \textit{u/umsy} & \textit{From assessment?} & \textit{Source} \\
%\hline \hline
%blah & blah & 2000 & 1.0 & 1.0 & yes &  (blah) \\
%\end{tabular}
%\end{table}

% latex table generated in R 2.13.0 by xtable 1.5-6 package
% Wed Jun 22 11:39:32 2011
\begin{table}[ht]
\begin{center}
\begin{tabular}{ccc}
  \hline
 & SP U/Umsy $<$ 1 & SP U/Umsy $>$ 1 \\ 
  \hline
U/Umsy $<$ 1 &  20 &  14 \\ 
  U/Umsy $>$ 1 &   2 &   8 \\ 
  B/Bmsy $<$ 1 &  28 &   6 \\ 
  B/Bmsy $>$ 1 &  12 &  30 \\ 
   \hline
\end{tabular}
\caption{Contingency tables of stock status classification for biomass and exploitation reference points obtained from assessments and those derived from surplus production models. }
\label{tab:contingency}
\end{center}
\end{table}

\begin{table}[ht]
\begin{center}
\begin{tabular}{rllrrrl}
  \hline
 & stock & scientificname & currentyear & Bratio & Uratio & fromassessment \\
  \hline
1 & Alaska plaice Bering Sea and Aleutian Islands & Pleuronectes quadrituberculatus & 2008 & 2.46 & 0.05 & yes \\
  2 & Arrowtooth flounder Bering Sea and Aleutian Islands & Reinhardtius stomias & 2008 & 1.28 & 0.31 & no \\
  3 & Arrowtooth flounder Gulf of Alaska & Reinhardtius stomias & 2007 & 1.33 & 0.28 & no \\
  4 & Atka mackerel Bering Sea and Aleutian Islands & Pleurogrammus monopterygius & 2008 & 1.50 & 0.55 & no \\
  5 & Cabezon Northern California & Scorpaenichthys marmoratus & 2004 & 0.89 & 0.99 & no \\
  6 & Cabezon Southern California & Scorpaenichthys marmoratus & 2004 & 0.81 & 0.53 & no \\
  7 & Dusky rockfish Gulf of Alaska & Sebastes variabilis & 2007 & 1.54 & 0.54 & yes \\
  8 & Flathead sole Bering Sea and Aleutian Islands & Hippoglossoides elassodon & 2008 & 1.66 & 0.18 & no \\
  9 & Greenland turbot Bering Sea and Aleutian Islands & Reinhardtius hippoglossoides & 2009 & 1.48 & 0.05 & yes \\
  10 & Northern rockfish Bering Sea and Aleutian Islands & Sebastes polyspinis & 2008 & 1.07 & 0.13 & no \\
  11 & Northern rockfish Gulf of Alaska & Sebastes polyspinis & 2008 & 1.50 & 0.66 & yes \\
  12 & Northern rock sole Eastern Bering Sea and Aleutian Islands & Lepidopsetta polyxystra & 2007 & 3.02 & 0.21 & yes \\
  13 & Pacific cod Bering Sea and Aleutian Islands & Gadus macrocephalus & 2007 & 0.89 & 0.93 & no \\
  14 & Pacific cod Gulf of Alaska & Gadus macrocephalus & 2007 & 1.07 & 0.84 & no \\
  15 & Pacific Ocean perch Eastern Bering Sea and Aleutian Islands & Sebastes alutus & 2008 & 1.70 & 0.26 & no \\
  16 & Pacific ocean perch Gulf of Alaska & Sebastes alutus & 2008 & 1.16 & 0.73 & yes \\
  17 & Red king crab Bristol Bay & Paralithodes camtschaticus & 2007 & 0.00 & 1.38 & no \\
  18 & Sablefish Eastern Bering Sea / Aleutian Islands / Gulf of Alaska & Anoplopoma fimbria & 2008 & 0.72 & 0.90 & no \\
  19 & Snow crab Bering Sea & Chionoecetes opilio & 2008 & 0.29 & 1.61 & no \\
  20 & Walleye pollock Aleutian Islands & Theragra chalcogramma & 2008 & 0.86 & 0.02 & yes \\
  21 & Walleye pollock Eastern Bering Sea & Theragra chalcogramma & 2008 & 0.68 & 0.85 & no \\
  22 & Yellowfin sole Bering Sea and Aleutian Islands & Limanda aspera & 2008 & 1.94 & 0.62 & yes \\
  23 & Capelin Barents Sea & Mallotus villosus & 2006 & 0.17 & 0.00 & no \\
  24 & Atlantic cod coastal Norway & Gadus morhua & 2006 & 0.20 & 2.57 & no \\
  25 & Atlantic cod Northeast Arctic & Gadus morhua & 2006 & 0.56 & 1.42 & no \\
  26 & Greenland halibut Northeast Arctic & Reinhardtius hippoglossoides & 2006 & 0.23 & 1.38 & no \\
  27 & Golden Redfish Northeast Arctic & Sebastes norvegicus & 2006 & 0.00 & 4.25 & no \\
  28 & Haddock Northeast Arctic & Melanogrammus aeglefinus & 2006 & 1.10 & 1.06 & no \\
  29 & Pollock Northeast Arctic & Pollachius virens & 2006 & 1.70 & 0.60 & no \\
  30 & Atlantic croaker Mid-Atlantic Coast & Micropogonias undulatus & 2002 & 1.42 & 0.27 & yes \\
  31 & Northern shrimp Gulf of Maine & Pandalus borealis & 2008 & 1.58 & 0.56 & no \\
  32 & Antarctic toothfish Ross Sea & Dissostichus mawsoni & 2007 & 1.76 & 1.09 & no \\
  33 & Deepwater flathead Southeast Australia & Platycephalus conatus & 2006 & 1.33 & 0.61 & no \\
  34 & common gemfish Southeast Australia & Rexea solandri & 2007 & 0.01 & 0.59 & no \\
  35 & Jackass morwong Southeast Australia & Nemadactylus macropterus & 2007 & 0.25 & 1.80 & no \\
  36 & New Zealand ling Eastern half of Southeast Australia & Genypterus blacodes & 2007 & 0.60 & 2.20 & no \\
  37 & Orange roughy Cascade Plateau & Hoplostethus atlanticus & 2006 & 1.76 & 0.34 & no \\
  38 & Orange roughy Southeast Australia & Hoplostethus atlanticus & 2006 & 0.89 & 0.29 & no \\
  39 & Silverfish Southeast Australia & Seriolella punctata & 2006 & 1.15 & 0.79 & no \\
  40 & School whiting Southeast Australia & Sillago flindersi & 2007 & 1.10 & 0.82 & no \\
  41 & Tiger flathead Southeast Australia & Neoplatycephalus richardsoni & 2006 & 1.05 & 0.00 & no \\
  42 & Blue Warehou Eastern half of Southeast Australia & Seriolella brama & 2006 & 0.17 & 0.84 & no \\
  43 & Blue Warehou Western half of Southeast Australia & Seriolella brama & 2006 & 0.62 & 2.04 & no \\
  44 & Atlantic cod NAFO 5Zjm & Gadus morhua & 2002 & 0.00 & 0.74 & no \\
  45 & Haddock NAFO-4X5Y & Melanogrammus aeglefinus & 2003 & 0.28 & 0.45 & no \\
  46 & Haddock NAFO-5Zejm & Melanogrammus aeglefinus & 2002 & 0.15 & 0.86 & no \\
  47 & Atlantic cod NAFO 2J3KL inshore & Gadus morhua & 2005 & 1.60 & 0.14 & no \\
  48 & Atlantic cod NAFO 3Ps & Gadus morhua & 2004 & 0.43 & 0.43 & no \\
  49 & English sole Hecate Strait & Parophrys vetulus & 2001 & 1.23 & 0.37 & no \\
  50 & Pacific herring Central Coast & Clupea pallasii & 2007 & 0.30 & 0.11 & no \\
  51 & Pacific herring Prince Rupert District & Clupea pallasii & 2007 & 0.00 & 0.44 & no \\
  52 & Pacific herring Queen Charlotte Islands & Clupea pallasii & 2007 & 0.20 & 0.00 & no \\
  53 & Pacific herring Straight of Georgia & Clupea pallasii & 2007 & 0.91 & 0.40 & no \\
  54 & Pacific herring West Coast of Vancouver Island & Clupea pallasii & 2007 & 0.03 & 0.00 & no \\
  55 & Pacific cod Hecate Strait & Gadus macrocephalus & 2004 & 0.37 & 0.25 & no \\
  56 & Pacific cod West Coast of Vancouver Island & Gadus macrocephalus & 2001 & 0.28 & 0.61 & no \\
  57 & Rock sole Hecate Strait & Lepidopsetta bilineata & 2001 & 1.03 & 0.45 & no \\
  58 & Pollock NAFO-4VWX5Zc & Pollachius virens & 2006 & 0.56 & 0.30 & no \\
  59 & Atlantic cod NAFO 3Pn4RS & Gadus morhua & 2006 & 0.03 & 1.09 & no \\
  60 & Atlantic cod NAFO 4TVn & Gadus morhua & 2006 & 0.17 & 0.32 & no \\
  61 & Herring ICES 22-24-IIIa & Clupea harengus & 2006 & 0.73 & 1.02 & no \\
  62 & Herring Northern Irish Sea & Clupea harengus & 2006 & 0.72 & 0.34 & no \\
  63 & Herring North Sea & Clupea harengus & 2006 & 0.65 & 1.32 & no \\
  64 & Herring ICES VIa & Clupea harengus & 2006 & 0.18 & 1.59 & no \\
  65 & Herring ICES VIa-VIIb-VIIc & Clupea harengus & 2000 & 0.50 & 1.04 & no \\
  66 & Albacore tuna North Atlantic & Thunnus alalunga & 2005 & 0.81 & 1.49 & yes \\
  67 & Bluefin tuna Eastern Atlantic & Thunnus thynnus & 2007 & 0.34 & 9.38 & yes \\
  68 & Bluefin tuna Western Atlantic & Thunnus thynnus & 2007 & 0.57 & 1.33 & yes \\
  69 & Bigeye tuna Atlantic & Thunnus obesus & 2005 & 0.90 & 0.86 & no \\
  70 & Skipjack tuna Eastern Atlantic & Katsuwonus pelamis & 2006 & 1.71 & 0.27 & no \\
  71 & Skipjack tuna Western Atlantic & Katsuwonus pelamis & 2006 & 1.72 & 0.27 & no \\
  72 & Swordfish Mediterranean Sea & Xiphias gladius & 2006 & 0.94 & 1.27 & yes \\
  73 & Swordfish North Atlantic & Xiphias gladius & 2005 & 1.03 & 0.82 & no \\
  74 & Swordfish South Atlantic & Xiphias gladius & 2005 & 1.18 & 0.69 & no \\
  75 & Yellowfin tuna Atlantic & Thunnus albacares & 2006 & 1.07 & 0.81 & yes \\
  76 & Chilean jack mackerel Chilean EEZ and offshore & Trachurus murphyi & 2006 & 0.52 & 1.20 & no \\
  77 & Argentine anchoita Northern Argentina & Engraulis anchoita & 2007 & 1.37 & 0.17 & yes \\
  78 & Argentine anchoita Southern Argentina & Engraulis anchoita & 2007 & 3.13 & 0.04 & yes \\
  79 & Argentine hake Northern Argentina & Merluccius hubbsi & 2007 & 0.16 & 1.26 & yes \\
  80 & Argentine hake Southern Argentina & Merluccius hubbsi & 2008 & 0.34 & 1.49 & yes \\
  81 & Patagonian grenadier Southern Argentina & Macruronus magellanicus & 2006 & 1.82 & 0.60 & yes \\
  82 & Bigeye tuna Indian Ocean & Thunnus obesus & 2004 & 1.23 & 0.97 & yes \\
  83 & Pacific halibut North Pacific & Hippoglossus stenolepis & 2008 & 0.54 & 2.01 & no \\
  84 & Anchovy South Africa & Engraulis encrasicolus & 2006 & 0.97 & 0.36 & no \\
  85 & Shallow-water cape hake South Africa & Merluccius capensis & 2008 & 1.16 & 0.40 & no \\
  86 & Cape horse mackerel South Africa South coast & Trachurus capensis & 2007 & 1.47 & 0.76 & no \\
  87 & Kingklip South Africa & Engraulis encrasicolus & 2008 & 1.13 & 0.55 & no \\
  88 & Sardine South Africa & Sardinops sagax & 2006 & 0.75 & 0.55 & no \\
  89 & Southern spiny lobster South Africa South coast & Palinurus gilchristi & 2008 & 0.51 & 1.50 & no \\
  90 & American Plaice NAFO-3LNO & Hippoglossoides platessoides & 2006 & 0.02 & 1.05 & no \\
  91 & American Plaice NAFO-3M & Hippoglossoides platessoides & 2007 & 0.00 & 0.00 & no \\
  92 & Atlantic cod NAFO 3NO & Gadus morhua & 2006 & 0.00 & 0.38 & no \\
  93 & Greenland halibut NAFO 23KLMNO & Reinhardtius hippoglossoides & 2006 & 0.39 & 1.73 & no \\
  94 & Redfish species NAFO 3LN & Redfish species & 2008 & 1.88 & 0.04 & yes \\
  95 & Redfish species NAFO 3M & Redfish species & 2006 & 0.93 & 0.15 & no \\
  96 & Yellowtail Flounder NAFO 3LNO & Limanda ferruginea & 2007 & 1.62 & 0.15 & no \\
  97 & American Plaice NAFO-5YZ & Hippoglossoides platessoides & 2007 & 0.55 & 0.30 & no \\
  98 & Bluefish Atlantic Coast & Pomatomus saltatrix & 2007 & 0.81 & 1.25 & no \\
  99 & Black sea bass Mid-Atlantic Coast & Centropristis striata & 2007 & 1.21 & 0.67 & no \\
  100 & Atlantic cod Georges Bank & Gadus morhua & 2007 & 0.00 & 1.15 & no \\
  101 & Atlantic cod Gulf of Maine & Gadus morhua & 2007 & 1.46 & 0.29 & no \\
  102 & Haddock NAFO-5Y & Melanogrammus aeglefinus & 2007 & 0.00 & 1.86 & no \\
  103 & Monkfish Gulf of Maine / Northern Georges Bank & Lophius americanus & 2006 & 1.73 & 0.38 & no \\
  104 & Monkfish Southern Georges Bank / Mid-Atlantic & Lophius americanus & 2006 & 1.72 & 0.30 & no \\
  105 & Sea scallop Georges Bank & Placopecten magellanicus & 2006 & 1.59 & 0.78 & no \\
  106 & Sea scallop Mid-Atlantic Coast & Placopecten magellanicus & 2006 & 0.91 & 0.37 & no \\
  107 & Spiny dogfish Atlantic Coast & Squalus acanthias & 2005 & 1.61 & 0.15 & no \\
  108 & Atlantic surfclam Mid-Atlantic Coast & Spisula solidissima & 1994 & 1.85 & 0.00 & no \\
  109 & Tilefish Mid-Atlantic Coast & Lopholatilus chamaeleonticeps & 2005 & 0.72 & 0.61 & no \\
  110 & Weakfish Atlantic Coast & Cynoscion regalis & 2008 & 0.06 & 1.49 & no \\
  111 & White hake Georges Bank / Gulf of Maine & Urophycis tenuis & 2007 & 0.35 & 0.80 & yes \\
  112 & Winter Flounder NAFO-5Z & Pseudopleuronectes americanus & 2006 & 1.41 & 0.25 & no \\
  113 & Winter Flounder Southern New England-Mid Atlantic & Pseudopleuronectes americanus & 2007 & 0.07 & 1.44 & no \\
  114 & Witch Flounder NAFO-5Y & Glyptocephalus cynoglossus & 2007 & 0.30 & 1.45 & yes \\
  115 & Yellowtail flounder Cape Cod / Gulf of Maine & Limanda ferruginea & 2007 & 0.25 & 1.73 & yes \\
  116 & Yellowtail flounder Georges Bank & Limanda ferruginea & 2007 & 0.22 & 1.14 & yes \\
  117 & Yellowtail Flounder Southern New England-Mid Atlantic & Limanda ferruginea & 2007 & 0.13 & 1.61 & yes \\
  118 & Australian salmon New Zealand & Arripis trutta & 2006 & 1.64 & 0.33 & yes \\
  119 & Orange roughy New Zealand Mid East Coast & Hoplostethus atlanticus & 2004 & 1.20 & 0.35 & yes \\
  120 & Atlantic menhaden Atlantic & Brevoortia tyrannus & 2005 & 0.47 & 0.97 & no \\
  121 & Pacific sardine North Pacific & Sardinops sagax & 2006 & 1.73 & 0.37 & no \\
  122 & Arrowtooth flounder Pacific Coast & Reinhardtius stomias & 2007 & 3.81 & 0.21 & yes \\
  123 & Blackgill rockfish  Pacific Coast & Sebastes melanostomus & 2004 & 0.80 & 1.20 & no \\
  124 & Black rockfish Northern Pacific Coast & Sebastes melanops & 2006 & 1.37 & 0.57 & no \\
  125 & Black rockfish Southern Pacific Coast & Sebastes melanops & 2007 & 2.23 & 0.33 & yes \\
  126 & Blue rockfish California & Sebastes mystinus & 2007 & 0.75 & 1.19 & yes \\
  127 & Bocaccio Southern Pacific Coast & Sebastes paucispinis & 2006 & 0.32 & 0.10 & yes \\
  128 & Chilipepper Southern Pacific Coast & Sebastes goodei & 2006 & 1.43 & 0.04 & no \\
  129 & Cowcod Southern California & Sebastes levis & 2007 & 0.09 & 0.07 & yes \\
  130 & Canary rockfish Pacific Coast & Sebastes pinniger & 2007 & 0.85 & 0.02 & yes \\
  131 & Darkblotched rockfish Pacific Coast & Sebastes crameri & 2007 & 0.73 & 0.31 & yes \\
  132 & English sole Pacific Coast & Parophrys vetulus & 2006 & 2.06 & 0.14 & no \\
  133 & Longnose skate Pacific Coast & Raja rhina & 2007 & 1.56 & 0.40 & no \\
  134 & Longspine thornyhead Pacific Coast & Sebastolobus altivelis & 2005 & 2.65 & 0.23 & yes \\
  135 & Pacific hake Pacific Coast & Merluccius productus & 2007 & 0.42 & 1.67 & no \\
  136 & Pacific ocean perch Pacific Coast & Sebastes alutus & 2007 & 0.69 & 0.00 & yes \\
  137 & Petrale sole Northern Pacific Coast & Eopsetta jordani & 2004 & 0.96 & 1.26 & no \\
  138 & Petrale sole Southern Pacific Coast & Eopsetta jordani & 2004 & 0.63 & 0.61 & no \\
  139 & Sablefish Pacific Coast & Anoplopoma fimbria & 2007 & 0.84 & 1.09 & no \\
  140 & Shortspine thornyhead Pacific Coast & Sebastolobus alascanus & 2004 & 1.55 & 0.93 & no \\
  141 & Widow rockfish Pacific Coast & Sebastes entomelas & 2006 & 0.91 & 0.07 & no \\
  142 & Yelloweye rockfish Pacific Coast & Sebastes ruberrimus & 2006 & 0.38 & 0.34 & no \\
  143 & Yellowtail rockfish Northern Pacific Coast & Sebastes flavidus & 2005 & 0.75 & 0.51 & no \\
  144 & Capelin Iceland & Mallotus villosus & 2006 & 0.49 & 0.85 & no \\
  145 & Atlantic cod Faroe Plateau & Gadus morhua & 2006 & 0.26 & 1.52 & no \\
  146 & Atlantic cod Iceland & Gadus morhua & 2006 & 0.46 & 1.17 & no \\
  147 & Haddock Faroe Plateau & Melanogrammus aeglefinus & 2006 & 0.85 & 1.07 & no \\
  148 & Haddock Iceland & Melanogrammus aeglefinus & 2007 & 0.73 & 1.36 & no \\
  149 & Pollock Faroe Plateau & Pollachius virens & 2006 & 0.99 & 1.52 & no \\
  150 & Black oreo West end of Chatham Rise & Allocyttus niger & 2007 & 0.99 & 0.82 & yes \\
  151 & Smooth oreo Chatham Rise & Pseudocyttus maculatus & 2006 & 2.25 & 0.38 & yes \\
  152 & Smooth oreo West end of Chatham Rise & Pseudocyttus maculatus & 2004 & 1.25 & 0.53 & yes \\
  153 & Hoki Eastern New Zealand & Macruronus novaezelandiae & 2007 & 1.11 & 0.33 & no \\
  154 & Hoki Western New Zealand & Macruronus novaezelandiae & 2007 & 0.51 & 0.57 & no \\
  155 & New Zealand snapper New Zealand Area 8 & Chrysophrys auratus & 2005 & 0.35 & 2.50 & yes \\
  156 & Trevally New Zealand Areas TRE 7 & Pseudocaranx dentex & 2005 & 1.44 & 0.83 & yes \\
  157 & Red rock lobster New Zealand area CRA1 & Jasus edwardsii & 2001 & 1.14 & 0.88 & no \\
  158 & Red rock lobster New Zealand area CRA2 & Jasus edwardsii & 2001 & 0.53 & 2.12 & no \\
  159 & Red rock lobster New Zealand area CRA4 & Jasus edwardsii & 2005 & 0.67 & 1.33 & no \\
  160 & Red rock lobster New Zealand area CRA5 & Jasus edwardsii & 2002 & 0.59 & 1.68 & no \\
  161 & Red rock lobster New Zealand area CRA7 & Jasus edwardsii & 2005 & 0.73 & 0.43 & no \\
  162 & Red rock lobster New Zealand area CRA8 & Jasus edwardsii & 2005 & 0.69 & 0.49 & no \\
  163 & common gemfish New Zealand & Rexea solandri & 2006 & 1.64 & 0.43 & yes \\
  164 & New Zealand ling New Zealand Areas LIN 3 and 4 & Genypterus blacodes & 2007 & 3.07 & 0.09 & yes \\
  165 & New Zealand ling New Zealand Areas LIN 5 and 6 & Genypterus blacodes & 2007 & 3.96 & 0.10 & yes \\
  166 & New Zealand ling New Zealand Area LIN 6b & Genypterus blacodes & 2006 & 2.19 & 0.11 & yes \\
  167 & New Zealand ling New Zealand Area LIN 72 & Genypterus blacodes & 2007 & 2.49 & 0.32 & yes \\
  168 & New Zealand ling New Zealand Area LIN 7WC - WCSI & Genypterus blacodes & 2008 & 2.21 & 0.13 & yes \\
  169 & Southern blue whiting Campbell Island Rise & Micromesistius australis & 2006 & 1.15 & 0.92 & yes \\
  170 & Southern hake Chatham Rise & Merluccius australis & 2006 & 1.77 & 0.12 & yes \\
  171 & Southern hake Sub-Antarctic & Merluccius australis & 2007 & 2.91 & 0.11 & yes \\
  172 & New Zealand abalone species New Zealand Area PAU 5A & Haliotis iris & 2006 & 0.72 & 2.83 & no \\
  173 & New Zealand abalone species New Zealand Area PAU 5B & Haliotis iris & 2007 & 1.02 & 0.59 & no \\
  174 & New Zealand abalone species New Zealand Area PAU 5D & Haliotis iris & 2006 & 0.44 & 2.10 & no \\
  175 & New Zealand abalone species New Zealand Area PAU 7 & Haliotis iris & 2008 & 0.87 & 0.94 & no \\
  176 & American lobster Rhode Island & Homarus americanus & 2006 & 0.51 & 0.68 & no \\
  177 & Tautog Rhode Island & Tautoga onitis & 2006 & 0.84 & 0.59 & no \\
  178 & Winter flounder Rhode Island & Pseudopleuronectes americanus & 2006 & 0.03 & 2.35 & no \\
  179 & Gag Gulf of Mexico & Mycteroperca microlepis & 2004 & 1.25 & 1.84 & no \\
  180 & Gag Southern Atlantic coast & Mycteroperca microlepis & 2005 & 0.94 & 1.31 & yes \\
  181 & Greater amberjack Gulf of Mexico & Seriola dumerili & 2004 & 0.46 & 1.52 & no \\
  182 & King mackerel Gulf of Mexico & Scomberomorus cavalla & 2001 & 1.51 & 0.44 & no \\
  183 & King mackerel Southern Atlantic Coast & Scomberomorus cavalla & 2001 & 1.38 & 0.56 & no \\
  184 & Gulf menhaden Gulf of Mexico & Brevoortia patronus & 2004 & 1.08 & 0.48 & no \\
  185 & Red grouper Gulf of Mexico & Epinephelus morio & 2005 & 0.17 & 1.39 & no \\
  186 & Red porgy Southern Atlantic coast & Pagrus pagrus & 2004 & 0.61 & 0.39 & yes \\
  187 & Snowy grouper Southern Atlantic coast & Epinephelus niveatus & 2002 & 0.19 & 3.08 & yes \\
  188 & Spanish mackerel Southern Atlantic Coast & Scomberomorus maculatus & 2007 & 0.38 & 0.91 & yes \\
  189 & Tilefish Southern Atlantic coast & Lopholatilus chamaeleonticeps & 2002 & 0.94 & 1.55 & yes \\
  190 & Vermilion snapper Southern Atlantic coast & Rhomboplites aurorubens & 2007 & 0.86 & 1.27 & yes \\
  191 & Yellowtail snapper Southern Atlantic Coast and Gulf of Mexico & Ocyurus chrysurus & 2001 & 1.14 & 0.61 & yes \\
  192 & Walleye pollock Northern Sea of Okhotsk & Theragra chalcogramma & 1992 & 1.11 & 0.87 & no \\
  193 & Albacore tuna South Pacific Ocean & Thunnus alalunga & 2006 & 2.46 & 0.90 & yes \\
  194 & Bigeye tuna Western Pacific Ocean & Thunnus obesus & 2006 & 1.06 & 1.38 & yes \\
  195 & Skipjack tuna Central Western Pacific & Katsuwonus pelamis & 2006 & 4.38 & 0.30 & yes \\
  196 & Yellowfin tuna Central Western Pacific & Thunnus albacares & 2005 & 1.22 & 0.80 & yes \\
  197 & Dover sole Pacific Coast & Microstomus pacificus & 2004 & 1.33 & 0.45 & no \\
  198 & Gopher rockfish Southern Pacific Coast & Sebastes carnatus & 2004 & 1.69 & 0.08 & no \\
  199 & Pacific sardine Pacific Coast & Sardinops sagax & 2006 & 1.36 & 0.41 & no \\
  200 & Starry flounder Northern Pacific Coast & Platichthys stellatus & 2004 & 0.68 & 0.33 & no \\
  201 & Starry flounder Southern Pacific Coast & Platichthys stellatus & 2004 & 1.15 & 0.12 & no \\
  202 & Tasmanian giant crab Tasmania & Pseudocarcinus gigas & 2007 & 0.50 & 1.71 & no \\
  203 & Walleye pollock Western Bering Sea & Theragra chalcogramma & 2004 & 2.16 & 0.26 & no \\
  204 & Atlantic cod Baltic Areas 22 and 24 & Gadus morhua & 2006 & 0.31 & 1.55 & no \\
  205 & Atlantic cod Baltic Areas 25-32 & Gadus morhua & 2006 & 0.14 & 1.58 & no \\
  206 & Atlantic cod Kattegat & Gadus morhua & 2006 & 0.19 & 0.31 & no \\
  207 & Herring ICES 25-32 & Clupea harengus & 2006 & 0.69 & 0.79 & no \\
  208 & Herring ICES 30 & Clupea harengus & 2006 & 1.19 & 1.10 & no \\
  209 & Herring ICES 31 & Clupea harengus & 2006 & 0.27 & 1.65 & no \\
  210 & Herring Iceland (Summer spawners) & Clupea harengus & 2006 & 0.70 & 0.89 & no \\
  211 & Herring ICES 28 & Clupea harengus & 2006 & 1.21 & 0.87 & no \\
  212 & common European sole ICES Kattegat and Skagerrak & Solea vulgaris & 2006 & 1.25 & 0.54 & no \\
  213 & Sprat ICES Baltic Areas 22-32 & Sprattus sprattus & 2006 & 1.13 & 1.27 & no \\
  214 & Fourspotted megrim ICES VIIIc-IXa & Lepidorhombus boscii & 2006 & 0.00 & 1.61 & no \\
  215 & Hake Northeast Atlantic North & Merluccius merluccius & 2006 & 1.04 & 0.74 & no \\
  216 & Megrim ICES VIIIc-IXa & Lepidorhombus whiffiagonis & 2006 & 0.24 & 1.31 & no \\
  217 & common European sole Bay of Biscay & Solea vulgaris & 2006 & 0.62 & 1.13 & no \\
  218 & Mackerel ICES Northeast Atlantic & Scomber scombrus & 2006 & 0.98 & 0.73 & no \\
  219 & Whiting Northeast Atlantic & Micromesistius poutassou & 2006 & 0.67 & 1.66 & no \\
  220 & Atlantic cod Irish Sea & Gadus morhua & 2006 & 0.09 & 0.69 & no \\
  221 & Atlantic cod West of Scotland & Gadus morhua & 2006 & 0.10 & 0.45 & no \\
  222 & Haddock West of Scotland & Melanogrammus aeglefinus & 2006 & 0.58 & 0.73 & no \\
  223 & European Plaice Irish Sea & Pleuronectes platessa & 2006 & 1.07 & 0.23 & no \\
  224 & common European sole Irish Sea & Solea vulgaris & 2006 & 0.36 & 1.16 & no \\
  225 & Atlantic cod North Sea & Gadus morhua & 2006 & 0.19 & 0.80 & no \\
  226 & Haddock ICES IIIa and North Sea & Melanogrammus aeglefinus & 2006 & 0.62 & 0.25 & no \\
  227 & Haddock Rockall Bank & Melanogrammus aeglefinus & 2006 & 1.10 & 0.52 & no \\
  228 & Norway pout North Sea & Trisopterus esmarkii & 2006 & 0.90 & 0.33 & no \\
  229 & Pollock ICES IIIa, VI and North Sea & Pollachius virens & 2006 & 0.56 & 0.97 & no \\
  230 & Sandeel North Sea & Ammodytes marinus & 2007 & 0.92 & 0.24 & no \\
  231 & Whiting ICES IIIa, VIId and North Sea & Merlangius merlangus & 2006 & 0.33 & 1.04 & no \\
  232 & Haddock ICES VIIb-k & Melanogrammus aeglefinus & 2006 & 1.37 & 0.41 & no \\
  233 & European Plaice ICES VIIf-g & Pleuronectes platessa & 2006 & 0.65 & 0.41 & no \\
  234 & European Plaice ICES VIIe & Pleuronectes platessa & 2006 & 0.51 & 1.39 & no \\
  235 & common European sole Celtic Sea & Solea vulgaris & 2006 & 0.90 & 0.95 & no \\
  236 & common European sole Western English Channel & Solea vulgaris & 2006 & 0.51 & 1.75 & no \\
  237 & Whiting ICES VIIe-k & Merlangius merlangus & 2006 & 0.44 & 1.25 & no \\
   \hline
\end{tabular}
\caption{Data used to generate Figures~\ref{fig:friedegg} and ~\ref{fig:friedeggmgmt} - Summary of the assessments used in this analysis and their estimated ratios of current biomass to the biomass at maximum sustainable yield and current harvest rate to the harvest rate that results is maximum sustainable yield. The estimated ratios were preferentially obtained directly from the assessment document or derived from surplus production model fits. When both an SSBmsy and Bmsy reference points are available, the SSB is chosen preferentially.}
\label{tab:crosshair}
\end{center}
\end{table}


\end{document}

 % - open-ended, this is just the beginning, we want more assessments
