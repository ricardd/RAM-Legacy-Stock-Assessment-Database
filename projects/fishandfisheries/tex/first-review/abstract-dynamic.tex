\section*{Abstract}

Comparative analyses can provide novel insight into marine population dynamics and the status of fished species, but the world's main stock assessment database (the Myers Stock-Recruitment database) is now outdated.  
To facilitate new analyses, we developed a new stock assessment database, the RAM Legacy database, for commercially exploited marine fishes and invertebrates. 
Time series of total biomass, spawner biomass, recruits, fishing mortality, and catch form the core of the database. 
Assessments were assembled from 21 national and international management agencies for a total of 331 stocks (295 fish stocks representing 46 families, and 36 invertebrate stocks representing 12 families), including 9 of the world's 10 largest fisheries. Stock assessments were available from 27 Large Marine Ecosystems and 4 High Seas regions, and include the Atlantic, Pacific, Indian, Arctic and Antarctic Oceans. 
Most assessments came from the U.S., Europe, Canada, New Zealand, and Australia. Assessed marine stocks represent a small proportion of harvested fish taxa (18\%), and an even smaller proportion of marine fish biodiversity (1\%), but provide high quality abundance data for these intensively studied stocks. 
The database provides new insight into the status of exploited populations: 58\% of stocks with reference points (n=214) were estimated to be below the biomass that results in maximum sustainable yield ( ), and 30\% had exploitation levels estimated to be above the exploitation rate that results in maximum sustainable yield ( ).
We anticipate that the database will facilitate new research in population and fishery dynamics and life histories and encourage further data contributions from stock assessment scientists.

%Assessments were assembled for 331 stocks
%(295 fish populations representing
%46 families, and 36
%invertebrate populations representing 12
%families). Assessments were obtained from 21 national
%and international management institutions. Stocks
%present in the database come from 27 Large
%Marine Ecosystems. Assessed marine fish stocks
%comprise a relatively small proportion of harvested taxa
%(18\%), and an even smaller proportion of
%marine fish biodiversity (1\%). Reference
%points were available or could be calculated for about
%65\% of these stocks. The available data
%provide new insight into the status of exploited populations,
%58\% of stocks with reference points
%were estimated to be below $B_{msy}$, and
%30\% had exploitation levels
%estimated to be above $U_{msy}$. Temporal coverage of assessments is
%recent with 90\% of catch time-series ranging 1966-2007
%and 90\% of biomass time-series ranging 1972-2007.
