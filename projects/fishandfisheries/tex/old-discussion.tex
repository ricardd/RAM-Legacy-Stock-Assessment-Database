%%%%%%%%%%%%%%%%%%%%%%%%%%%%%%%%%%%
% OLD DISCUSSION
The RAM Legacy Database of available stock assessments for the world's
marine fisheries, although non-exhaustive, provides a basis to
evaluate the existing knowledge-base of exploited marine populations.
The stocks comprising the database are predominantly from developed
nations with properly identified fisheries management bodies and tend
to cover a recent time period. Few assessments extend back beyond 30
years from present. The taxonomic makeup of available assessments is a
very limited subset of the accepted taxonomic coverage of marine
species worldwide.

\subsection*{Biases}

Even in developed countries, however, not all stocks are assessed.
For example, in 2007, of the 528 fish and invertebrate stocks
recognized by the National Marine Fisheries Service (NMFS), only 179,
or slightly over one-third, were fully assessed
\citep{NMFS:2008:status}.  An assessment by the European Environment
Agency (EEA) in 2006 indicated that the percentage of commercial
landings obtained from assessed stocks ranged between 66-97 percent in
northern European waters and 30-77 percent in the Mediterranean
\citep{eea:2009:status}.  The New Zealand Ministry of Fisheries
reports the status of 117 stocks or sub-stocks out of a total of 628
stocks managed under New Zealand's Quota Management System
\citep{NZMF:2009}.  In Australia, 98 federally managed stocks have
been assessed \citep{Wilson:etal:2009:status} out of an unknown total.
The extent to which stocks are assessed elsewhere in the world is
currently unknown.


\subsubsection*{Geographic bias}

There are important geographic biases in the amount of assessments entered per LME

The question of geographic bias relates to whether: \begin{inparaenum}[1\upshape)] \item an
assessment is conducted on a stock; \item it is possible to access the
assessment; and \item the non-exhaustive collation we undertook may have
overlooked the assessment\end{inparaenum}. Whether an assessment is conducted for a
given stock depends upon a multitude of factors, including the
economic value of the stock and availability of fiscal resources to
collect the data required for an assessment. How accessible
assessments are for entry depends upon the transparency and access
policies of the relevant management agencies, which varies
geographically. Similarly, the legal context where fisheries are
prosecuted will strongly influence the requirement for conducting
stock assessments. In the United States, the Magnuson-Stevens Act
defines what stocks are to be monitored and managed, hence a large
number of the assessments in the RAM Legacy database are under the
jurisdiction of the US National Marine Fisheries Services.  Our
incomplete search for assessments could also give rise to geographic
biases, as concerted collation efforts have only been conducted in
those assessment-rich regions of It is hoped that readers of this
article can assist in correcting these biases by participating in
future updates of the RAM Legacy database.


Note Also: from Figure 4 SOFIA 2008 (FAO 2009) - The top 10
wild-caught marine fisheries producer countries are China, Peru,
U.S.A., Indonesia, Japan, Chile, India, Russia, Thailand, Philippines.
We only have assessments for U.S.A. and Russia (and only 2 for
Russia!). Most major fish producing countires do not do assessments or
do not make them accessible, still to be decided how to phrase this.


%%INTRODUCE LMES, LIMITATIONS AND ADDITIONS FOR OPEN OCEAN SPECIES. SAME FOR FAO AREAS 
%Large Marine Ecosystems were defined by NOAA \citep{NOAA:LME64:1998}.
%They encompass the continental shelves of the world's oceans and
%represent the most productive areas of the oceans.  However, they do
%not include a classification category for large migratory species such
%as tuna that also inhabit the open ocean and is not associated with
%given LMEs. Each species of highly migratory species is also observed
%in a number of LMEs. While the global coverage of LMEs is beneficial
%to classify assessments from around the world, it is debatable whether
%once can use each LME as an independent unit of comparison. 

%There are important geographic biases in the amount of assessments
%entered per LME (Figure~\ref{fig:lmes}). A large proportion of
%assessments come from North America, Europe, Australia, New Zealand
%and the High Seas. Few assessments are entered from regions such as
%Southeast Asia, South America, and the Indian Ocean (outside
%Australian waters). The question of geographic bias relates to
%whether: \begin{inparaenum}[1\upshape)] \item an assessment is
%  conducted on a stock; \item it is possible to access the assessment;
%  and \item the non-exhaustive collation we undertook may have
%  overlooked the assessment \end{inparaenum}. Whether an assessment is
%conducted for a given stock depends upon a multitude of factors,
%including the economic value of the stock and availability of fiscal
%resources to collect the data required for an assessment. How
%accessible assessments are for entry depends upon the transparency and
%access policies of the relevant management agencies, which varies
%geographically. Similarly, the legal context where fisheries are
%prosecuted will strongly influence the requirement for conducting
%stock assessments. In the United States, the Magnuson-Stevens Act
%defines what stocks are to be monitored and managed, hence a large
%number of the assessments in the RAM Legacy database are under the
%jurisdiction of the US National Marine Fisheries Services.

%Our incomplete search for assessments could also give rise to
%geographic biases, as concerted collation efforts have only been
%conducted in those assessment-rich regions of Figure~\ref{fig:lmes}.
%It is hoped that readers of this article can assist in correcting
%these biases by participating in future updates of the RAM Legacy
%database. 

\subsubsection*{Taxonomic bias}

Taxonomic biases in those species that are assessed include ...

%The accepted taxonomic coverage of fish and elasmobranch species from
%FishBase includes 12339 species from 54 different orders. In comparsion,
%the SAUP data is from XX species and XX orders.  Taxonomic biases in
%those species that are assessed include ...
%% Note that this analysis does not account for discarding or unreported catches \citep{Pauly:etal:2002:nature, Pitcher:etal:2002:fnf}.

%Global fish catches represent XX\% of the accepted taxonomic coverage
%for fish species.


%The assessed stocks present in the RAM Legacy database represent XX\%
%of the global catch for fish species.


\subsection*{Caveats and limitations}
Assessment outputs e.g. biomass timeseries, are model estimates, not
raw data. The uncertainty associated with these estimates should be
carried forth in subsequent analyses. The RAM Legacy database
structure allows for estimates of uncertainty (standard errors, 95\%
credible/confidence intervals), however these estimates are often
missing from assessments either because they aren't produced by the
assessment model (e.g. non-bootstrapped VPA assessments) or the focus
of the assessment document was on central tendency (e.g. mean
biomass), not the associated uncertainty. Note that this view is
changing with the advent of MCMC aproaches to Bayesian inference for
assessments, bootstrap methods, statistical catch-at-age models
\citep{admb} and a general focus on uncertainty
\citep{Walters:Maguire:1996:reviews}. As with any analysis, clearer
inference on the strength of a signal is available when all
uncertainty in the data is carried forth. This represents a difficulty
for large-scale analyses of fisheries data in that in an ideal world
one would access the raw data per sub-unit (e.g. stock) and carry
forth the uncertainty at all levels of the analysis. In the case of
assessments, the raw data is typically catch-at-age matrices and
potentially survey indices. To understand the fleet characteristics
and survey stratification schema for each stock in a potentially
global meta-analysis would be extremely time consuming and
error-prone. So, the expert opinion of those researchers most familiar
with the data, stock assessment authors, is used but without
accompanying uncertainty estimates the strength of conclusions drawn
may be weakened.

%delta method approximations 

The original database developed by Dr. Myers was used to address a
variety of ecological question derived from stock-recruit
relationships. This was possible because of the timeseries of stock
and recruitment generated in assessment using VPA-type models. With
the increasing use of statistical catch-at-age/length models, an
underlying stock-recruit relationship is assumed to exist. Estimating
the parameters of a stock-recruit relationship using timeseries from
such models does not constitute a valid statistical procedure.  More
generally, the increasing use of Bayesian methods that incorporate
prior information poses a challenge for meta-analysis of such model
outputs.

Point data are stored in the database with an associated unit, value
and year. We expect to also include age-varying and length-varying
data such as maturity ogive and age-disaggregated natural mortality in
subsequent releases of the RAM Legacy database. In addition to the
initial aim of providing reliable access to timeseries information
about stocks, we hope to also stimulate research in the relationships
of life-history characteristics and their relation to exploitation.

%Assessment outputs e.g. biomass time series, are estimates, not raw
%data. The uncertainty associated with these estimates should be
%carried forth in subsequent analyses. The RAM Legacy database
%structure allows for estimates of uncertainty (standard errors, 95\%
%credible/confidence intervals), however these estimates are only
%occassionally provided because they aren't produced by the assessment
%model (e.g. non-bootstrapped VPA assessments) or the focus of the
%assessment document was on central tendency (e.g. mean biomass), not
%the associated uncertainty. Note that this view is changing with the
%advent of MCMC aproaches to Bayesian inference for assessments,
%bootstrap methods, statistical catch-at-age models and a focus on
%uncertainty \citep{Walters:Maguire:1996:reviews}. As with any
%analysis, clearer inference on the strength of a signal is available
%when all uncertainty in the data is carried forth. This represents a
%difficulty for large-scale analyses of fisheries data in that in an
%ideal world one would access the raw data per sub-unit (e.g. stock)
%and carry forth the uncertainty at all levels of the analysis. In the
%case of assessments, the raw data is typically catch-at-age matrices
%and potentially survey indices. To understand the fleet
%characteristics and survey stratification schema for each stock in a
%potentially global meta-analysis would be extremely time consuming and
%error-prone. So, the expert opinion of those researchers most familiar
%with the data, stock assessment authors, is used but without
%accompanying uncertainty estimates the strength of conclusions drawn
%may be weakened.

%The original database developed by Dr. Myers was used to address a
%variety of ecological question derived from stock-recruit
%relationships. This was possible because of the time series of stock
%and recruitment generated in assessment using VPA-type models. With
%the increasing use of statistical catch-at-age/length models, an
%underlying stock-recruit relationship is assumed to exist. Estimating
%the parameters of a stock-recruit relationship using time series from
%such models does not constitute a valid statistical procedure. 

%Point data are stored in the database with an associated unit, value
%and year. We expect to also include age-varying and length-varying
%data such as maturity ogive and age-disaggregated natural mortality in
%subsequent releases of the RAM Legacy database. In addition to the
%initial aim of providing reliable access to time series information
%about stocks, we hope to also stimulate research in the relationships
%of life-history characteristics and their relation to exploitation.

% \subsection*{Using a Relational Database Management System}


%Housing assessments in a Relational Database Management System (RDBMS)
%allows mutliple users to concurrently access and extract subsets of
%persistent data in an efficient and reproducible manner. With the
%development of Application Programming Interfaces (APIs) that allow
%analytical softwares to directly communicate and extract data from the
%database, a common data environment is established, independent of
%ones choice of analytical software e.g. (SAS:SAS ACCESS, Matlab:
%Matlab/Database, R:RDBI/RODBC, Perl:DBI, etc.). In all these
%applications the same SQL query will extract the same data. A data
%product tailored to a specific project can be generated and stored as
%a dynamic (continually updated) ``view'' within the database. These
%are typically spreadsheet-like results of an expansive query of the
%relevant tables that can be readily read into all commonly-used
%analytical softwares. In contrast, manipulating flat text files or spreadsheets for
%importing into a specific analytical software runs the risk of losing
%data integrity and becomes impractical with large, non-rectangular,
%datasets. Full benefits of the database can be realized through the
%development of tailored queries to suit one's analyses. Such queries
%require an understanding of the structure of the various data within
%assessments. RDBMSs form the server back-end to a great many
%applications of interest to ecologists, including web-clients and GIS
%softwares. 

\subsection*{Future Development}
It is anticipated that the RAM Legacy database will continue to grow
with hitherto unentered stocks and updated assessments for already
included stocks. The ultimate goal is to provide a data standard for
researchers to use results from multiple regions to assist in their
own applied and fundamental research in population ecology, fisheries
science, and conservation biology. The development of a standard for
assessment reporting would assist in realizing this goal. For example,
ICES assessments have a very regular standard, including agreed-upon
reference points and regular estimate reporting. This makes the
process of data collation much more routine than unstandardized
documents where the recorder trawls through a report for information.
Certainly different stocks and regions require different formats but
the basic output tables, consisting of total and spawning biomass,
recruitment, catch/landings, estimated fishing mortality over
vulnerable age groups, associated measures of uncertainty, and
commonly-used reference points would streamline the process immensely.
A process whereby the assessment spreadsheets are filled out at each
assessment meeting would be the least error prone method. In return,
the assessment scientists can access results for a global collation of
assessments to further their own research initiatives in population
assessment and management. Other products include
management-agency-level reports containing summaries of all stocks
within their remit. Future versions to the database will also include
timelines of management actions per stock. The RAM Legacy database
contains the corresponding species codes to the Sea Around Us Project
and FishBase, thus facilitating researchers's use of a global
fisheries data ``toolkit'' to address questions on the relationships
between life history attributes and resulting population dynamics in
an exploited setting.

%It is anticipated that the RAM Legacy database will continue to grow
%with hitherto unentered and updated assessments. The ultimate goal is
%to provide a data standard for researchers to use results from
%multiple regions to assist in their own applied and fundamental
%research in population ecology. The development of a standard for
%assessment reporting would assist in realizing this goal. For example,
%ICES assessments have a very regular standard, including agreed-upon
%reference points and regular estimate reporting. This makes the
%process of data collation much more routine than unstandardized
%documents where the recorder trawls through a report for information.
%Certainly different stocks and regions require different formats but
%the basic output tables, consisting of total and spawning biomass,
%recruitment, catch/landings, estimated fishing mortality over
%vulnerable age groups, associated measures of uncertainty, and
%commonly-used reference points would streamline the process immensely.
%A process whereby the assessment spreadsheets are filled out at each
%assessment meeting would be the least error prone method. In return,
%the assessment scientists can access results for a global collation of
%assessments to further their own research initiatives in population
%assessment and management. Other products include
%management-agency-level reports containing summaries of all stocks
%within their remit. Future versions to the database will also include
%timelines of management actions per stock. The RAM Legacy database
%contains the corresponding species codes to the Sea Around Us Project
%and FishBase, thus facilitating researchers's use of a global
%fisheries data ``toolkit'' to address questions on the relationships
%between life history attributes and resulting population
%dynamics in an exploited setting.

%, particularly providers, \citep[for survey data]{Ricard:etal:2009:ices}

%- Weather forecasting and climatology note don't
%use these terms (central repository)

